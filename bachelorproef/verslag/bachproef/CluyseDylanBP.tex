%===============================================================================
% LaTeX sjabloon voor de bachelorproef toegepaste informatica aan HOGENT
% Meer info op https://github.com/HoGentTIN/latex-hogent-report
%===============================================================================

\documentclass[dutch,dit,thesis]{hogentreport}

% TODO:
% - If necessary, replace the option `dit`' with your own department!
%   Valid entries are dbo, dbt, dgz, dit, dlo, dog, dsa, soa
% - If you write your thesis in English (remark: only possible after getting
%   explicit approval!), remove the option "dutch," or replace with "english".

\usepackage{lipsum} % For blind text, can be removed after adding actual content
\usepackage{array}
\usepackage{listings}

\usepackage{rotating}
\usepackage{tikz}

\lstdefinelanguage{Python}{
	keywords=[1]{request},
	keywordstyle=[1]\color{Bittersweet},
	keywords=[2]{form}, % ML-typen
	keywordstyle=[2]\color{purple},	
	keywords=[3]{}, 
	keywordstyle=[3]\color{violet},
	keywords=[4]{reader, writer},
	keywordstyle=[4]\color{RoyalBlue},
	keywords=[5]{},
	keywordstyle=[5]\color{Aquamarine}\bfseries,
	keywords=[6]{setup\_scholars\_teachers, teaching\_tool},
	keywordstyle=[6]\color{OliveGreen}\bfseries,
	keywords=[7]{def, function, for, in},
	keywordstyle=[7]\color{PineGreen},
	keywords=[8]{}, %input
	keywordstyle=[8]\color{Periwinkle},
	identifierstyle=\color{black},
	sensitive=false,
	comment=[l]{//},
	morecomment=[s]{/*}{*/},
	commentstyle=\color{red}\ttfamily,
	stringstyle=\color{Sepia}\ttfamily,
	morestring=[b]',
	morestring=[b]"
}

\lstdefinelanguage{javascript}{
	keywords=[1]{addeventlistener, getelementbyid, queryselectorall, foreach},
	keywordstyle=[1]\color{Bittersweet},
	keywords=[2]{const, let, var, typeof, new, true, false, catch, function, return, null, catch, switch, var, if, in, while, do, else, case, break}, % ML-typen
	keywordstyle=[2]\color{purple},	
	keywords=[3]{document, element, elements}, 
	keywordstyle=[3]\color{violet},
	keywords=[4]{nouns, adj, verb, other},
	keywordstyle=[4]\color{RoyalBlue},
	keywords=[5]{},
	keywordstyle=[5]\color{Aquamarine}\bfseries,
	keywords=[6]{setup\_scholars\_teachers, teaching\_tool},
	keywordstyle=[6]\color{OliveGreen}\bfseries,
	keywords=[7]{def, function, for, in},
	keywordstyle=[7]\color{PineGreen},
	keywords=[8]{}, %input
	keywordstyle=[8]\color{Periwinkle},
	identifierstyle=\color{black},
	sensitive=false,
	comment=[l]{//},
	morecomment=[s]{/*}{*/},
	commentstyle=\color{red}\ttfamily,
	stringstyle=\color{Sepia}\ttfamily,
	morestring=[b]',
	morestring=[b]"
}

\lstset{ %
	backgroundcolor=\color{white},   
	basicstyle=\footnotesize,        
	breakatwhitespace=false,         
	breaklines=true,                 
	captionpos=b,                    
	commentstyle=\color{commentsColor}\textit,
	deletekeywords={...},            % if you want to delete keywords from the given language
	escapeinside={\%*}{*)},          % if you want to add LaTeX within your code
	extendedchars=true,              % lets you use non-ASCII characters; for 8-bits encodings only, does not work with UTF-8
	frame=tb,	                   	   % adds a frame around the code
	keepspaces=true,                 % keeps spaces in text, useful for keeping indentation of code (possibly needs columns=flexible)
	keywordstyle=\color{keywordsColor},       % keyword style
	language=Python,                 % the language of the code (can be overrided per snippet)
	otherkeywords={*,...},           % if you want to add more keywords to the set
	numbers=left,                    % where to put the line-numbers; possible values are (none, left, right)
	numbersep=5pt,                   % how far the line-numbers are from the code
	numberstyle=\tiny\color{commentsColor}, % the style that is used for the line-numbers
	rulecolor=\color{black},         % if not set, the frame-color may be changed on line-breaks within not-black text (e.g. comments (green here))
	showspaces=false,                % show spaces everywhere adding particular underscores; it overrides 'showstringspaces'
	showstringspaces=false,          % underline spaces within strings only
	showtabs=false,                  % show tabs within strings adding particular underscores
	stepnumber=1,                    % the step between two line-numbers. If it's 1, each line will be numbered
	stringstyle=\color{stringColor}, % string literal style
	tabsize=2,	                   % sets default tabsize to 2 spaces
	title=\lstname,                  % show the filename of files included with \lstinputlisting; also try caption instead of title
	columns=fixed                    % Using fixed column width (for e.g. nice alignment)
}

%% Pictures to include in the text can be put in the graphics/ folder
\graphicspath{{graphics/}}

%% For source code highlighting, requires pygments to be installed
%% Compile with the -shell-escape flag!
\usepackage[section]{minted}
\usemintedstyle{solarized-light}
\definecolor{bg}{RGB}{253,246,227} %% Set the background color of the codeframe

%% Change this line to edit the line numbering style:
\renewcommand{\theFancyVerbLine}{\ttfamily\scriptsize\arabic{FancyVerbLine}}

%% Macro definition to load external java source files with \javacode{filename}:
\newmintedfile[javacode]{java}{
    bgcolor=bg,
    fontfamily=tt,
    linenos=true,
    numberblanklines=true,
    numbersep=5pt,
    gobble=0,
    framesep=2mm,
    funcnamehighlighting=true,
    tabsize=4,
    obeytabs=false,
    breaklines=true,
    mathescape=false
    samepage=false,
    showspaces=false,
    showtabs =false,
    texcl=false,
}

% Other packages not already included can be imported here

%%---------- Document metadata -------------------------------------------------
% TODO: Replace this with your own information
\author{Dylan Cluyse}
\supervisor{Mevr. L. De Mol}
\cosupervisor{J. Decorte; J. Van Damme;}
\title[]%
    {Scholieren met dyslexie van de derde graad middelbaar onderwijs ondersteunen bij het begrijpend lezen van wetenschappelijke artikelen via geautomatiseerde en gepersonaliseerde tekstvereenvoudiging}
\academicyear{\advance\year by -1 \the\year--\advance\year by 1 \the\year}
\examperiod{2}
\degreesought{\IfLanguageName{dutch}{Professionele bachelor in de toegepaste informatica}{Bachelor of applied computer science}}
\partialthesis{false} %% To display 'in partial fulfilment'
%\institution{Internshipcompany BVBA.}

%% Add global exceptions to the hyphenation here
\hyphenation{back-slash}

%% The bibliography (style and settings are  found in hogentthesis.cls)
\addbibresource{bachproef.bib}            %% Bibliography file
\addbibresource{../voorstel/voorstel.bib} %% Bibliography research proposal
\defbibheading{bibempty}{}

%% Prevent empty pages for right-handed chapter starts in twoside mode
\renewcommand{\cleardoublepage}{\clearpage}

\renewcommand{\arraystretch}{1.2}

%% Content starts here.
\begin{document}

%---------- Front matter -------------------------------------------------------

\frontmatter

\hypersetup{pageanchor=false} %% Disable page numbering references
%% Render a Dutch outer title page if the main language is English
\IfLanguageName{english}{%
    %% If necessary, information can be changed here
    \degreesought{Professionele Bachelor toegepaste informatica}%
    \begin{otherlanguage}{dutch}%
       \maketitle%
    \end{otherlanguage}%
}{}

%% Generates title page content
\maketitle
\hypersetup{pageanchor=true}

%%=============================================================================
%% Voorwoord
%%=============================================================================

\chapter*{\IfLanguageName{dutch}{Woord vooraf}{Preface}}%
\label{ch:voorwoord}

Deze scriptie en het bijhorende onderzoek zou niet tot stand zijn gekomen zonder de waardevolle bijdragen van diverse individuen die mij hebben ondersteund en gestimuleerd tijdens mijn onderzoek. Ik wil graag mijn oprechte dank betuigen aan deze personen.

\medspace

Ten eerste, wil ik mijn promotor Lena De Mol bedanken voor haar uitmuntende begeleiding tijdens het onderzoek. Haar affiniteit voor technologie, taal en onderwijs vormde een perfecte match met het onderzoeksgebied van deze scriptie. Daarnaast wil ik graag Johan Decorte en Jana Van Damme bedanken voor hun deskundige inbreng op de vakgebieden machinaal leren en logopedie. Elke wekelijkse sessie met Johan bracht nieuwe inzichten in hoe ik het technologische component van mijn onderzoek kon aanpakken. Dit heeft mijn ambitie alleen maar vergroot. Jana ben ik dankbaar voor haar begeleiding en follow-up op het gebied van logopedie. Haar expertise heeft mijn horizon verbreed binnen dit vakgebied. Verder wil ik Emmanuel Vercruysse en Johannes Nijs van Hogeschool Vives en Sofie Smet en Sophie Vyncke van Arteveldehogeschool bedanken voor hun bijdragen aan de referentieteksten voor het experiment. Alle lectoren hebben mij met veel plezier geholpen door deze taken op zich te nemen, ondanks hun drukke agenda’s. Tot slot, wil ik mijn goede vriendin Lobke bedanken voor haar constante steun en aanmoediging tijdens het hele onderzoeksproces, en ook mijn grootste steunpunt die ik tijdens het schrijven van deze scriptie heb kunnen vinden.

\medspace

Ik wil graag benadrukken dat deze personen van onschatbare waarde zijn geweest voor het succes van mijn onderzoek en het eindresultaat. Hun inzet en toewijding hebben ertoe bijgedragen dat ik deze scriptie met trots kan presenteren.
%%=============================================================================
%% Samenvatting
%%=============================================================================

% TODO: De "abstract" of samenvatting is een kernachtige (~ 1 blz. voor een
% thesis) synthese van het document.
%
% Een goede abstract biedt een kernachtig antwoord op volgende vragen:
%
% 1. Waarover gaat de bachelorproef?
% 2. Waarom heb je er over geschreven?
% 3. Hoe heb je het onderzoek uitgevoerd?
% 4. Wat waren de resultaten? Wat blijkt uit je onderzoek?
% 5. Wat betekenen je resultaten? Wat is de relevantie voor het werkveld?
%
% Daarom bestaat een abstract uit volgende componenten:
%
% - inleiding + kaderen thema
% - probleemstelling
% - (centrale) onderzoeksvraag
% - onderzoeksdoelstelling
% - methodologie
% - resultaten (beperk tot de belangrijkste, relevant voor de onderzoeksvraag)
% - conclusies, aanbevelingen, beperkingen
%
% LET OP! Een samenvatting is GEEN voorwoord!

%%---------- Nederlandse samenvatting -----------------------------------------
%
% TODO: Als je je bachelorproef in het Engels schrijft, moet je eerst een
% Nederlandse samenvatting invoegen. Haal daarvoor onderstaande code uit
% commentaar.
% Wie zijn bachelorproef in het Nederlands schrijft, kan dit negeren, de inhoud
% wordt niet in het document ingevoegd.

\IfLanguageName{english}{%
\selectlanguage{dutch}
\chapter*{Samenvatting}
\lipsum[1-4]
\selectlanguage{english}
}{}

%%---------- Samenvatting -----------------------------------------------------
% De samenvatting in de hoofdtaal van het document

\chapter*{\IfLanguageName{dutch}{Samenvatting}{Abstract}}

Ingewikkelde woordenschat en zinsbouw hinderen scholieren met dyslexie in de derde graad van het middelbaar onderwijs bij het begrijpend lezen van wetenschappelijke artikelen. Gepersonaliseerde \textit{manual text simplification} (MTS) is een gekende en bewezen techniek die deze scholieren helpt met hun leesbegrip. De techniek vraagt echter veel tijd en energie en is daardoor minder inzetbaar. Artificiële intelligentie maakt de automatisering van het proces mogelijk en vermindert de werkdruk bij leraren en scholieren. Dit onderzoek achterhaalt met welke technologische en logopedische aspecten AI-ontwikkelaars rekening moeten houden bij de ontwikkeling van een AI-toepassing voor geautomatiseerde en gepersonaliseerde tekstvereenvoudiging. Hiervoor stelt het onderzoek de volgende onderzoeksvraag op: "Hoe kan een wetenschappelijk artikel automatisch worden vereenvoudigd, gericht op de unieke noden van scholieren met dyslexie in de derde graad middelbaar onderwijs?". Een requirementsanalyse achterhaalt de benodigde functionaliteiten om gepersonaliseerde en geautomatiseerde tekstvereenvoudiging mogelijk te maken. Vervolgens wijst de vergelijkende studie uit welk taalmodel ontwikkelaars kunnen inzetten voor de taak van gepersonaliseerde en geautomatiseerde tekstvereenvoudiging. De requirementsanalyse wijst uit dat bestaande toepassingen om wetenschappelijke artikelen te vereenvoudigen, gemaakt zijn voor een centrale doelgroep en geen rekening houden met de unieke noden van een scholier met dyslexie in de derde graad middelbaar onderwijs. Hoewel toepassingen voor gepersonaliseerde automatische tekstvereenvoudiging mogelijk zijn, is het essentieel dat ontwikkelaars zich meer inzetten om de bekende uitdagingen van dyslexie en verwante taalstoornissen bij scholieren weg te werken. 

%---------- Inhoud, lijst figuren, ... -----------------------------------------

\tableofcontents

% In a list of figures, the complete caption will be included. To prevent this,
% ALWAYS add a short description in the caption!
%
%  \caption[short description]{elaborate description}
%
% If you do, only the short description will be used in the list of figures

\listoffigures

% If you included tables and/or source code listings, uncomment the appropriate
% lines.
\listoftables

\listoflistings

% Als je een lijst van afkortingen of termen wil toevoegen, dan hoort die
% hier thuis. Gebruik bijvoorbeeld de ``glossaries'' package.
% https://www.overleaf.com/learn/latex/Glossaries
\glossary{}

%---------- Kern ---------------------------------------------------------------

\mainmatter{}

% De eerste hoofdstukken van een bachelorproef zijn meestal een inleiding op
% het onderwerp, literatuurstudie en verantwoording methodologie.
% Aarzel niet om een meer beschrijvende titel aan deze hoofdstukken te geven of
% om bijvoorbeeld de inleiding en/of stand van zaken over meerdere hoofdstukken
% te verspreiden!

%%=============================================================================
%% Inleiding
%%=============================================================================

\chapter{\IfLanguageName{dutch}{Inleiding}{Introduction}}%
\label{ch:inleiding}

Lezen is een dagelijkse activiteit voor iedereen. Deze vaardigheid strekt zich uit tot elk aspect van het leven. Dit geldt des te meer in het onderwijs, waar leerkrachten divers leesmateriaal gebruiken om lesinhouden op een authentieke manier over te brengen. Zo zetten leerkrachten in de derde graad van het middelbaar onderwijs wetenschappelijke artikelen in als leesvoer. Toch brengt hun leesgraad een nieuwe uitdaging mee voor zowel scholieren als leerkrachten. 

\medspace

Om scholieren attent te maken van wetenschappelijk onderzoek, lanceerde het Amerikaanse onderwijs het C.R.E.A.T.E.\footnote{https://teachcreate.org/}-initiatief. Het zet twaalf- tot achttienjarige scholieren aan om wetenschappelijke artikelen te lezen in plaats van enkel boeken. Zo begrijpen ze hoe wetenschappers onderzoek plannen, uitvoeren, en resultaten analyseren en interpreteren. Hoewel er geen vergelijkbare Vlaamse initiatieven bestaan, benadrukken de lerarenopleidingen toch het gebruik van afwisselende leerstof in klas. Andere initiatieven, zoals het M-decreet en de leerplannen van het katholiek\footnote{https://pro.katholiekonderwijs.vlaanderen/basisoptie-stem/ondersteunend-materiaal} en het gemeenschapsonderwijs\footnote{https://g-o.be/stem/}, adviseren Vlaamse leerkrachten om hun lessen op een toegankelijke manier aanbieden. Zo nemen zij alle scholieren ongeacht eventuele leesachterstand mee in het verhaal.

\medspace

Met een jaarlijks budget van 32 miljoen euro is Vlaanderen een pionier in Europa op het gebied van artificiële intelligentie (AI) op de werkvloer \autocite{Crevits2022}. Zo stampte de Vlaamse overheid verschillende  projecten uit de grond om Vlaamse AI ontwikkelingen te ondersteunen en AI softwarebedrijven te inspireren. Het amai!-project\footnote{https://amai.vlaanderen/} brengt AI softwarebedrijven uit diverse domeinen samen, waaronder het onderwijs. Zij doelen op een automatisering van processen om de werkdruk bij leerkrachten te verminderen. Dit gebeurt onder andere door \textit{real-time} ondertiteling in de klas en een taalassistent voor leerkrachten in meertalige klasgroepen.


\section{\IfLanguageName{dutch}{Probleemstelling}{Problem Statement}}%
\label{sec:probleemstelling}

De geletterdheid van scholieren bevindt zich op een kritieke punt. Elke drie jaar nemen experts uit 79 geïndustrialiseerde landen de PISA-test af bij middelbare scholen om de leesvaardigheid en wetenschappelijke geletterdheid van 15-jarige scholieren te meten. Uit de PISA-test van 2018 blijkt dat deze doelgroep in Vlaanderen zich echter negatief uit over leesplezier en daarmee de slechtst scorende doelgroep is van alle bevraagde landen. Zoals aangegeven in figuur \ref{img:oeso-leesplezier}, beschouwt bijna de helft van de bevraagden begrijpend lezen als tijdverspilling. Slechts 17\% beschouwt lezen als een hobby. Dit is een dalende trend, want voordien lag dit cijfer hoger dan 20\%. Lezen kan daarmee een obstakel vormen bij deze doelgroep.

\begin{figure}[H]
	\begin{center}
		\includegraphics[width=\linewidth]{img/oeso-graphic-leesplezier.png}
	\end{center}
	\caption{Het leesplezier bij 15-jarigen volgens de PISA-test \autocite{DeMeyer2019}.}
	\label{img:oeso-leesplezier}
\end{figure}

Begrijpend lezen valt niet te omzeilen in onze huidige samenleving, maar het leesbegrip verschilt sterk onder studenten in het middelbaar onderwijs. Zo benadrukt \textcite{Vlaanderen2020} dat begrijpend lezen een essentiële vaardigheid is, ook voor vakken buiten Nederlands. Bij wiskunde is begrijpend lezen van cruciaal belang bij complexe vraagstukken. Ook helpt begrijpend lezen studenten om STEM-vakjargon beter te begrijpen.

\medspace 

In het bijzonder hebben scholieren met dyslexie problemen met begrijpend lezen. Onderzoeken van \textcite{Bonte2020, VanDerMeer2022} schatten dat ongeveer 15\% van de Vlaamse scholieren in het middelbaar onderwijs een vorm van dyslexie heeft. Scholieren met dyslexie ervaren moeite en hinder bij het lezen en spellen. Ondanks de bestaande ondersteuning blijven ze toch nog steeds de negatieve impact van hun leerstoornis ervaren. De gevolgen hiervan  kunnen zich doorzetten na het middelbaar onderwijs \autocite{Lissens2020}. Leesvaardigheid blijft daarmee cruciaal voor succes op school en in het werkveld. Scholieren met dyslexie hebben moeilijkheden met deze vaardigheid, wat kan leiden tot onzekerheid en stress. Daarnaast zijn vooroordelen nog steeds een probleem en kunnen ze leiden tot stigmatisering. Echter toont onderzoek aan dat scholieren met dyslexie een sterke doorzettingsvermogen hebben en goede probleemoplossers zijn \autocite{Ghesquiere2018, Lissens2020, Bonte2020}. 

\medspace

Het leerplan voor STEM-vakken stimuleert het gebruik van wetenschappelijke artikelen, maar houdt niet altijd rekening met de bijhorende en complexe leesgraad. De ingewikkelde woordenschat en syntax in wetenschappelijke artikelen kunnen een hindernis vormen voor de begrijpelijkheid van een tekst, aldus \textcite{PlavenSigray2017}. Wetenschappelijke artikelen handmatig vereenvoudigen kan planning, tijd en energie van leerkrachten in het middelbaar onderwijs opslorpen. Het Vlaamse onderwijssysteem staat onder druk en docenten hebben moeite om boven water te blijven. 

\medspace

AI-technologieën zijn vandaag voldoende hoogstaand om tekstvereenvoudiging te automatiseren en om een baanbrekende oplossing te bieden aan het middelbaar onderwijs \autocite{Belpaeme2018}. Het onderwijst past echter zelden soortgelijke technologieën toe. Er is terughoudendheid door enerzijds ouders van leerlingen volgens \textcite{Martens2021a}, anderzijds door de weinige ontwikkelingen in schoolgerelateerde AI-software. Dit terwijl AI-ondersteuning in het onderwijs wel degelijk een positief effect heeft \autocite{Kraft2020}. Er is nood aan een intuïtieve en gebruikersvriendelijke toepassing die taalmodellen of API's kan integreren en aanpassen naargelang de specifieke behoeften van een student met dyslexie. Zo kan dit enerzijds de werkdruk bij leerkrachten verminderen, en anderzijds scholieren in de derde graad ondersteunen bij het lezen van complexe wetenschappelijke artikelen.

\section{\IfLanguageName{dutch}{Onderzoeksvraag}{Research question}}%
\label{sec:onderzoeksvraag}

Dit onderzoek beschrijft het gebruik van AI in de vorm van tekstvereenvoudiging, als advies voor implementatie in het onderwijs. Specifiek om scholieren met dyslexie in de derde graad van het middelbaar onderwijs te ondersteunen bij het begrijpend lezen van wetenschappelijke artikelen. Hiervoor stelt het onderzoek de volgende onderzoeksvraag op: 

\begin{itemize}
	\item Hoe kan een wetenschappelijk artikel automatisch vereenvoudigd worden, gericht op de unieke noden van scholieren met dyslexie in de derde graad middelbaar onderwijs?
\end{itemize}

De oplossingen voor de volgende deelvragen vormen een globaal antwoord op de onderzoeksvraag:

\begin{enumerate}
	% 1
	\item Welke specifieke noden hebben scholieren met dyslexie van de derde graad middelbaar onderwijs bij het begrijpen van complexere teksten? Aanvullend hierop: 
	\begin{itemize}
		\item Wat zijn de specifieke kenmerken van wetenschappelijke artikelen?
	\end{itemize} 
	% 2
	\item Welke aanpakken zijn er voor tekstvereenvoudiging?
	\begin{itemize}
		\item Hoe verloopt de handmatige vereenvoudiging van teksten voor scholieren met dyslexie?
		\item Welke toepassingen, tools en modellen zijn er beschikbaar om Nederlandse geautomatiseerde tekstvereenvoudiging met AI mogelijk te maken?
		\item Hoe is de combinatie van geautomatiseerde tekstvereenvoudiging met gepersonaliseerde  tekstvereenvoudiging mogelijk?
	\end{itemize}
	%4 
	\item Welke functies ontbreken AI-toepassingen om geautomatiseerde tekstvereenvoudiging mogelijk te maken voor scholieren met dyslexie in de derde graad middelbaar onderwijs? 
	\begin{itemize}
		\item Welke manuele methoden voor tekstvereenvoudiging ontbreken in deze tools?
	\end{itemize}
	%3 
	\item Met welke valkuilen bij taalverwerking met AI moeten ontwikkelaars rekening houden?
	% 5
	\item Welk taalmodel of LLM is geschikt voor de ATS van wetenschappelijke artikelen voor scholieren met dyslexie in de derde graad van het middelbaar onderwijs, met dezelfde of gelijkaardige kwaliteiten als gepersonaliseerde MTS?
	% 6
	\item Wat zijn de nodige stappen bij de ontwikkeling van een intuïtieve lokale webtoepassing die zowel scholieren met dyslexie als leerkrachten helpt?
\end{enumerate}


\section{\IfLanguageName{dutch}{Onderzoeksdoelstelling}{Research objective}}%
\label{sec:onderzoeksdoelstelling}

Het onderzoek achterhaalt de technologische en logopedische aspecten waarmee ontwikkelaars rekening meoten houden bij AI-tekstvereenvoudiging. Het resultaat dient als een houvast om hen te begeleiden tijdens de ontwikkeling van deze applicaties voor gepersonaliseerde en geautomatiseerde tekstvereenvoudiging. Verder ontwikkelt het onderzoek een soortgelijke toepassing in het bijzonder voor scholieren met dyslexie in de derde graad van het middelbaar onderwijs. Dit resulteert in een uitgewerkt prototype voor tekstvereenvoudiging, genaamd \textit{Pentimentor}. \textit{Pentimentor} heeft voornamelijk twee functies. Allereerst kan \textit{Pentimentor} wetenschappelijke artikelen vereenvoudigen op basis van de specifieke behoeften van scholieren met dyslexie in de derde graad van het middelbaar onderwijs. Daarnaast biedt \textit{Pentimentor} een geautomatiseerde benadering om wetenschappelijke artikelen op een gepersonaliseerde manier te vereenvoudigen door het gebruik van aanpasbare parameters. Tot slot geeft \textit{Pentimentor} de eindgebruiker het vereenvoudigd artikel terug in Word-formaat. 

\section{\IfLanguageName{dutch}{Opzet van deze bachelorproef}{Structure of this bachelor thesis}}%
\label{sec:opzet-bachelorproef}

De rest van deze scriptie is als volgt opgebouwd:

\begin{itemize}
	\item Hoofdstuk~\ref{ch:stand-van-zaken} geeft een overzicht van de stand van zaken binnen het onderzoeksdomein, op basis van een literatuurstudie.
	\item Hoofdstuk~\ref{ch:methodologie} licht de methodologie toe. Het onderzoek vermeldt de gebruikte onderzoekstechnieken om een antwoord te kunnen formuleren op de onderzoeksdeelvragen. 
	\item Hoofdstuk~\ref{ch:resultaten} bevat de resultaten voor alle onderzoekstechnieken.
	\item Hoofdstuk~\ref{ch:conclusie} geeft de uiteindelijke conclusie en beantwoordt daarmee de onderzoeksvraag.
	\item Tot slot geeft Hoofdstuk~\ref{ch:discussie} verdere aanbevelingen en aanzet voor toekomstig onderzoek binnen de bestudeerde domeinen. 
\end{itemize}

\chapter{\IfLanguageName{dutch}{Stand van zaken}{State of the art}}%
\label{ch:stand-van-zaken}

\section{Inleiding}

Het onderzoek start met een uitgebreide literatuurstudie over de benodigde kennis binnen het logopedisch, taal- en informaticavakdomein om geautomatiseerde en gepersonaliseerde vereenvoudigde teksten te verkrijgen van wetenschappelijke artikelen. Om een toepassing voor gepersonaliseerde en geautomatiseerde tekstvereenvoudiging van wetenschappelijke artikelen  op maat van deze doelgroep aan te reiken, is het van cruciaal belang om de noden van scholieren met dyslexie in de derde graad van het middelbaar onderwijs te benoemen. Het onderzoek benoemt bewezen noden met behulp van een literatuurstudie. Daarnaast kaart het de huidige problemen bij wetenschappelijke artikelen aan. Wetenschappelijke artikelen vereenvoudigen op maat voor scholieren met dyslexie kan volgens taalexperten op verschillende manieren. Het is belangrijk om stil te staan bij de bestaande en reeds bewezen technieken voor \textit{manual text simplification}. Vervolgens komen technieken voor \textit{automated text simplification} aan bod. Zowel de nodige informatie van taalverwerking met AI, als de huidige AI-technologieën voor tekstvereenvoudiging zijn gegeven. Ten slotte zijn AI-technologieën hoogstaand en ontwikkelaars maken deze alsmaar robuuster, maar het is cruciaal om bij dit onderzoek aandacht te besteden aan de mogelijke problemen die AI-ontwikkelaars moeten vermijden of waarvan zij zichzelf attent op moeten maken. 

\section{Specifieke noden en richtpunten}

Om wetenschappelijke artikelen specifiek voor scholieren met dyslexie te vereenvoudigen, moet het onderzoek stilstaan bij de unieke noden van scholieren met dyslexie in de derde graad van het middelbaar onderwijs. Daarnaast moet het onderzoek stilstaan bij de moeilijkheden tijdens het begrijpend lezen van wetenschappelijke artikelen. Deze sectie bespreekt eerst welke technieken en methoden er bestaan om scholieren met dyslexie te ondersteunen tijdens het begrijpend lezen van teksten. Daarna worden de belemmeringen en moeilijkheden van wetenschappelijke artikelen aangekaart. Deze sectie beantwoordt de volgende twee onderzoeksvragen: 

\begin{itemize}
	\item Welke specifieke noden hebben scholieren met dyslexie van de derde graad middelbaar onderwijs bij het begrijpen van complexere teksten?
	\item Wat zijn de specifieke kenmerken van wetenschappelijke artikelen?
\end{itemize}

\subsection{Specifieke noden van scholieren met dyslexie in de derde graad van het middelbaar onderwijs.}

Leesvaardigheid is geen aangeboren vaardigheid, maar iets dat mensen zelf moeten aanleren \autocite{Bonte2020, VanDerMeer2022}. Hoewel dit proces vlot kan verlopen, kunnen mensen met dyslexie benadeeld worden tijdens dit proces.  Dyslexie wordt gekenmerkt door beperkt lezen en kan het voorlezen traag, radend en letter-voor-letter maken. Mensen met dyslexie kunnen tijdens het begrijpend lezen verschillende drempels ervaren. Tabel \ref{table:dyslexia-hurdles} somt deze noden op.

\begin{center}
	\begin{table}[H]
	\begin{tabular}{ | m{9cm} | m{6cm} | } 
		\hline
		\textbf{Kenmerk} & \textbf{Bron} \\ 
		\hline
		Trage woordbenoeming &  \autocite{Bonte2020} \\
		\hline
		Problemen bij het leesbegrip & \autocite{Gala2016, Bonte2020} \\ 
		\hline
		Hardnekkig letter-voor-letter lezen & \autocite{Bonte2020, Zhang2021} \\ 
		\hline
		Problemen met woordherkenning -en herinnering & \autocite{Bonte2020} \\
		\hline
		Moeite bij homofonische of pseudo-homofonische woordenschat & \autocite{Zhang2021} \\
		\hline
		Moeite bij onregelmatige lettergreepcombinaties & \textcite{Gala2016} \\
		\hline
	\end{tabular}
	\caption{Unieke drempels bij scholieren met dyslexie tijdens het begrijpend lezen.}
	\label{table:dyslexia-hurdles}
	\end{table}
\end{center}

De digitalisering evolueerde de voorbije twintig jaar in stijgende lijn en scholieren in de tweede en derde graad zijn, door het gebruik van smartphones en laptops, hier het meeste vatbaar op \autocite{Fernando2021}. Verder omschrijft dit artikel een checklist van technische elementen waaraan een webpagina of toepassing moet voldoen om een leesbare ervaring te voorzien voor scholieren met dyslexie. Tabel \ref{table:dyslexia-necessaries} toont deze noden.

\begin{center}
		\begin{table}[H]
	\begin{tabular}{ | m{9cm} | m{6cm} | } 
		\hline
		\textbf{Kenmerk} & \textbf{Bron} \\
		\hline
		Zachtgele, -groene -of bruine achtergrondkleur 	& \textcite{Santana2012, Rello2017} 	\\ \hline
		Grotere lettergrootte dan 14pt gebruiken		& \autocite{Rello2015} 					\\ \hline
		Woord- en karakterspatiëring verbreden 			& \textcite{Santana2012, Rello2013b} 	\\ \hline
		Consistente lay-out toepassen					& \autocite{Rello2015, Fernando2021} 	\\  \hline
		Waarschuwingen geven omtrent formulieren, sessies (login) & \autocite{Fernando2021}  	\\ \hline
		Duidelijk zichtbare koppen- of headingstructuur & \autocite{Rello2012a} 				\\ \hline
		Duidelijke symbolen of \textit{icons} gebruiken & \autocite{Rello2012} 					\\ \hline
		Inhoud visueel groeperen 						& \autocite{Rello2015, Fernando2021}	\\ \hline
		Huidige positie benadrukken 					& \autocite{Fernando2021} 				\\ \hline
		
	\end{tabular}
	\caption{Noden en oplossingen om webpagina's beter af te stemmen op de mogelijke noden van scholieren met dyslexie.}
	\label{table:dyslexia-necessaries}
	\end{table}
\end{center}

\subsection{Specifieke kenmerken van wetenschappelijke artikelen}

Wetenschappelijke artikelen zijn van cruciaal belang voor het verspreiden van nieuwe kennis en onderzoeksresultaten. Toch blijven ze voor velen een mysterieus en ontoegankelijk gebied, omwille van de complexiteit van de inhoud en het technische jargon dat onmisbaar lijkt te zijn \autocite{Ball2017}. Dit kan het begrip van de artikelen bemoeilijken, vooral bij begrijpend lezen. Daarmee vormt er zich een extra obstakel bij het implementeren van wetenschappelijke artikelen als bron van kennis tijdens de les. Zo volgen wetenschappelijke artikelen IMRAD, een uniform formaat voor gepubliceerde wetenschappelijke artikelen, dat bestaat uit vijf hoofdstukken: samenvatting, inleiding, methodologie, resultaten en discussie. Hoewel het middelbaar en hoger onderwijs deze artikelen gebruiken als leermiddel, is  de inhoud van een hoger niveau en voornamelijk gericht op mensen uit het vakgebied. \textcite{Pain2016, CAS2021} benadrukken de complexiteit van wetenschappelijke artikelen en de volgende aspecten waarom ze moeilijk te interpreteren zijn. Tabel \ref{table:scientific-paper-struggles} somt deze factoren op. Hoewel wetenschappelijke artikelen over een grote drempel bezitten, betrekken ze jongeren met wetenschappelijk onderzoek en leren ze een kritische vaardigheid. 

\begin{center}
	\begin{table}[H]
	\begin{tabular}{| m{4cm} | m{8cm} | m{3cm} | }
		\hline
		\textbf{Probleem} & \textbf{Oplossing} & Bron \\
		\hline
		Veel informatie in een compact formaat of \textit{high information density} & Extra uitleg schrijven bij woorden of compacte zinnen schrijven. & \autocite{Matarese2013, PlavenSigray2017} \\
		\hline
		Hoog gebruik van meerlettergrepige woorden & Synoniemen met minder lettergrepen gebruiken. & \autocite{Siddharthan2006} \\
		\hline
		Wetenschappelijk jargon & Rekening houden met een doelgroep buiten het vakgebied door eenvoudigere synoniemen te schrijven. Indien deze niet beschikbaar zijn, kan er extra uitleg als alternatief worden gegeven. & \autocite{PlavenSigray2017} \\
		\hline
		Complexe concepten & Paragrafen herschrijven zodat ze eerst uitleg geven op een high-level niveau. Vervolgens lagen van complexiteit toevoegen om de lezer te begeleiden doorheen de methodologie, discussie en conclusie van het wetenschappelijk artikel. & \autocite{Pain2016} \\ 
		\hline
		Cijfermateriaal bij resultaten & De interpretatie van percentages of cijfermateriaal schrijven. Zoals 'ongeveer een kwart van de bevolking' in plaats van '24.97\% van de bevolking'. & \autocite{PlavenSigray2017} \\
		\hline
	\end{tabular}
	\caption{Complexe leesfactoren van een wetenschappelijk artikel.}
	\label{table:scientific-paper-struggles}
	\end{table}
\end{center}

Scholieren kunnen over verschillende achtergrondkennis beschikken, wat invloed kan hebben op het tekstbegrip tijdens het begrijpend lezen \autocite{DeMeyer2019}. Zo kunnen scholieren met een achtergrond in fysica sneller de draad oppikken bij het lezen van fysica-gerelateerde artikelen dan scholieren met een economische achtergrond. Dit maakt het moeilijk om de leesbaarheid van een tekst objectief te beoordelen. Het jargon kan voor de ene groep scholieren makkelijk zijn, en voor de andere groep moeilijk.

\medspace

Onderzoeken en ontwikkelaars zorgen voor de ontwikkeling van geautomatiseerde berekening van leesmetrieken en leesgraadsscores. Zo kunnen ontwikkelaars dit doen met python-libraries via commandline of CLI-toepassingen, waaronder Textstat\footnote{https://pypi.org/project/textstat/} en Readability\footnote{https://pypi.org/project/readability/}. Tabel \ref{table:readability-scores} toont drie prevalente leesgraadscores weer. Daarnaast kunnen toepassingen, zoals Textinspector\footnote{https://textinspector.com/} of Charactercalculator\footnote{https://charactercalculator.com/}, analytisch inzicht geven in de complexiteit van Engelstalige teksten. Deze toepassingen kunnen echter geen Nederlandstalige teksten of leesmetrieken analyseren. Hoewel deze leesmetrieken een beknopte analyse kunnen vormen voor taaldeskundigen, toch benadrukken onderzoekers dat deze leesgraadscores geen rekening houden met de achtergrondkennis van mogelijke lezers, aldus \autocite{Cantos2019}. Recent onderzoek van \textcite{Crossley2019} achterhaalt de mogelijkheid om geautomatiseerde taalverwerking te gebruiken voor leesmetrieken die wel rekening houden met de doelgroep. Hoewel deze modellen potentieel tonen, kunnen ontwikkelaars deze nog niet gebruiken \autocite{Crossley2019}.

\begin{center}
	\begin{table}[H]
	\begin{tabular}{ | m{5cm} | m{10cm} | } 
		\hline
		\textbf{Score} & \textbf{Uitleg} \\ 
		\hline
		Flesch Reading Ease (FRE) & Deze leesgraadscore berekent de moeilijkheidsgraad op zinbasis. Hoe hoger de score, hoe 'eenvoudiger' de zin \autocite{Cantos2019, Readable2021}. \\
		\hline
		Gunning FOG (FOG) & In tegenstelling tot FRE, berekent FOG de moeilijkheidsgraad op basis van de volledige tekst \autocite{Cantos2019}. \\
		\hline
		Complexe woordenlijst volgens \textit{Dale-Chall Index} (DCI) & Deze lijst omvat woorden die experimenten bij Amerikaanse tieners als complex omschrijven. De DCI werkt per leeftijdscategorie \autocite{Cantos2019}. \\
		\hline
	\end{tabular}
	\caption{Leesgraadscores volgens onderzoek van \textcite{Cantos2019}.}
	\label{table:readability-scores}
	\end{table}
\end{center}

Divers onderzoek van de afgelopen tien jaar wijzen uit dat wetenschappelijke teksten steeds complexer worden. Dat maakt deze teksten voor niet-experten en niet-doctoraatsstudenten minder toegankelijk, vanwege het gebruik van technisch jargon en ingewikkelde zinsstructuren \autocite{Ball2017, PlavenSigray2017, Jones2019}. Deze trend begon volgens onderzoek al in de tweede helft van de twintigste eeuw \autocite{Hayes1992}.

\medspace

Volgens onderzoek van \textcite{PlavenSigray2017} maken wetenschappers en onderzoekers onbewust wetenschappelijke artikelen moeilijker om te lezen. Uit een vergelijkende studie tussen abstracten en de rest van de inhoud van wetenschappelijke tijdschriften blijkt dat abstracts het meest complexe deel van een artikel vormen. De evolutie van de leesbaarheid wordt weergegeven in figuur \ref{img:fre-ndc}. Deze figuur toont de FRE (links) en NDC (rechts) scores. Zo schat het onderzoek dat 22\% van alle wetenschappelijke artikelen op het niveau van een masterstudent in het Engels geschreven zijn, tegenover 14\% in 1960. Deze trend is belangrijk om op te volgen in de komende decennia, omdat het een obstakel kan vormen voor toekomstige generaties.

\begin{figure}[H]
	\includegraphics[width=\linewidth]{img/fre-ndc.png}
	\caption{De toename van benodigde leesgraad voor het lezen van wetenschappelijke artikelen. Bron: \autocite{PlavenSigray2017}}
	\label{img:fre-ndc}
\end{figure}

Onbegrijpelijke en ontoegankelijke zinsstructuren hinderen ook vakexperten. Zo toonde onderzoek van \textcite{McNutt2014} aan dat begrip van de methodologie en resultaten cruciaal is in het kader van reproduceerbaarheid; enkel zo kunnen wetenschappers op correcte wijze een studie reproduceren en wetenschappelijke inzichten bevestigen of met verdere resultaten verrijken. Experimenten van \textcite{Hubbard2017} wijzen namelijk uit dat het net vooral de methodologie en resultaten van een wetenschappelijk artikel zijn die een hoge leesgraad vergen. In deze context zijn de onderzoeken van \textcite{Hartley1999} en \textcite{Snow2010} relevant waarin ze aantonen dat het herschrijven van abstracts de begrijpbaarheid ervan kan verhogen.

\medspace

Volgens \textcite{Hollenkamp2020} moeten vereenvoudigde of samengevatte wetenschappelijke artikelen drie vragen kunnen beantwoorden: waarom werd het onderzoek uitgevoerd, wat zijn de experimenten en wat zijn de conclusies van de onderzoekers? Dit omvat de achtergrondinformatie, hypotheses, methoden, resultaten, implicaties, beperkingen en aanbevelingen. De tekst omzetten in een ander formaat zoals post-it notes, tabelvorm of opsommingen, maakt het beter begrijpbaar \autocite{Rijkhoff2022}. 

\medspace

Wetenschappelijke artikelen zijn voornamelijk in pdf-formaat terug te vinden. Dit formaat valt eenvoudig in te lezen met python-pakketten, zoals PDFMiner of PyPDF. Wel ondervinden ontwikkelaars soms problemen bij het inlezen van pdf-bestanden, aldus \textcite{Lee2021}. Tools kunnen niet alle tekstinhoud uit een pdf extraheren. Als oplossing kunnen ontwikkelaars gebruik maken van \textit{optical character recognition} of OCR. Ondertussen bestaan er python-bibliotheken die deze technologie met een eenvoudige implementatie kunnen uitwerken, namelijk EasyOCR\footnote{https://pypi.org/project/easyocr/} en Tesserat \autocite{Lee2021}.

\medspace

Bovendien bestaat de uitvoer van deze artikelen uit louter losse tekstblokken. Het systeem is niet in staat om automatisch te identificeren welke delen titels, afbeeldingen of tekstblokken zijn. Een mogelijke oplossing hiervoor is het gebruik van \textit{LayoutParser}\footnote{https://pypi.org/project/layoutparser/}. Dit is een \textit{deep learning} of DL-model dat zorgvuldige \textit{Document Image Analysis} uitvoert. Met behulp van \textit{LayoutParser}, in samenwerking met het Detectron2\footnote{https://ai.meta.com/blog/-detectron2-a-pytorch-based-modular-object-detection-library-/}-algoritme, is het mogelijk om de tekstinhoud van wetenschappelijke artikelen te extraheren. Figuur \ref{img:layoutparser} geeft een voorbeeld weer waarbij het model de tekstblokken van drie verschillende soorten documenten, waaronder een wetenschappelijk artikel links in beeld, markeert en classificeert. Met reeds vermelde OCR-technieken kan het systeem deze omkaderde tekst inlezen en gebruiken in dergelijke toepassingen \autocite{Shen2021}.

\begin{figure}[H]
	\includegraphics[width=\linewidth]{img/layoutparser.png}
	\caption{LayoutParser toepassen op drie verschillende documenten. De kaders geven verschillende geclassificeerde tekstblokken aan. Bron: \autocite{Shen2021}.}
	\label{img:layoutparser}
\end{figure} 

\subsubsection{Conclusie}

In deze eerste sectie van de literatuurstudie zocht het onderzoek naar de specifieke behoeften van scholieren met dyslexie in de derde graad van het middelbaar onderwijs. Verder stond het stil bij de moeilijkheden die zij ervaren bij het begrijpend lezen van wetenschappelijke artikelen. Zo wijst de literatuurstudie de volgende technieken en methoden om hen te ondersteunen bij het begrijpend lezen. Tabellen \ref{table:dyslexia-necessaries} en \ref{table:dyslexia-hurdles} geven een overzicht van deze technieken en methoden. Daarnaast omschrijven onderzoeken ook specifieke kenmerken van wetenschappelijke artikelen die het begrip ervan bemoeilijken, opgesomd in tabel \ref{table:scientific-paper-struggles}. Tot slot kunnen ontwikkelaars en vakexperten de complexiteit van een tekst berekenen met bestaande leesgraadscores, zoals beschreven in tabel \ref{table:readability-scores}.

\section{Aanpakken voor tekstvereenvoudiging}

\subsection{Manuele tekstvereenvoudiging}
% Hoe worden teksten handmatig vereenvoudigd voor scholieren met dyslexie? 

Voor sommige lezers kan het begrijpend lezen van een complexe tekst echter een uitdaging zijn, zoals scholieren met dyslexie. \textit{Manual tekst simplification} of MTS kan deze groep helpen \autocite{Siddharthan2014}. De techniek van MTS gebruikt eenvoudige woordenschat en zinsstructuren en maakt structurele aanpassingen (SA) om de tekst vlotter leesbaar te maken. MTS is het proces dat het technische leesniveau en het woordgebruik van een geschreven tekst vermindert. Dit resulteert tot een betere leeservaring zonder het verlies van de kerninhoud tijdens het lezen van de tekst. Tabel \ref{table:manual-simplification} toont een overzicht van bewezen MTS-technieken, zonder een specifiek toespitsing op een doelgroep.

\begin{center}
		\begin{table}[H]
			\begin{tabular}{ | m{2.5cm} | m{8cm} | m{4.5cm} | } 
			\hline
			\textbf{Type vereenvoudiging} & \textbf{Techniek} & \textbf{Bron} \\ \hline
			
			LS & Moeilijke woorden vervangen door eenvoudigere synoniemen & \autocite{Crossley2012, Rello2013c, Siddharthan2014} \\ 
				& Woorden- en synoniemenlijst maken & \autocite{Siddharthan2006, Bosmans2022b} \\
				& Dubbelzinnige woorden vervangen & \\
				& Idiomen vervangen & \autocite{Siddharthan2006} \\ 
				& Regelmatige lettergreepcombinaties gebruiken & \autocite{Gala2016} \\
				& Rekening houden met het gekende jargon van de doelgroep & \autocite{JavoureyDrevet2022} \\
			\hline
			SS & Tangconstructies aanpassen & \autocite{Bosmans2022c} \\
			& Zinnen langer dan tien woorden inkorten & \autocite{Siddharthan2014} \\
			& Verwijswoorden aanpassen & \autocite{Bosmans2022a} \\
			& Voorzetseluitdrukkingen aanpassen & \autocite{Rello2013d} \\
			& Samengestelde werkwoorden aanpassen & \autocite{Bosmans2022b} \\
			& Actieve stem gebruiken & \autocite{Ruelas2020} \\
			& Onregelmatige werkwoorden vermijden & \autocite{Rello2013d, Gala2016} \\
			\hline
			SA & Marges aanpassen & \autocite{Rello2013d} \\
			& Lettertype -en grootte aanpassen & \autocite{Rello2012a} \\
			& Woord- en karakterspatiëring aanpassen & \autocite{Rello2012a} \\
			& Herschrijven als opsomming of tabelvorm & \autocite{Rello2015} \\
			\hline
		\end{tabular}
		\caption{Drie algemene technieken voor MTS bij een algemene doelgroep.}
		\label{table:manual-simplification}
	\end{table}
\end{center}

\subsection{Bevoordelende effecten van MTS bij scholieren met dyslexie}

Onderzoek toont aan dat vereenvoudigde teksten het leesbegrip en woordherkenning van kinderen met dyslexie significant kunnen verbeteren \autocite{RiveroContreras2021}. Bovendien blijkt uit experimenten dat frequent woordgebruik de decodeertijd bij mensen met dyslexie significant vermindert, en dat teksten met verminderde lexicale complexiteit minder leesfouten opleveren voor mensen met dyslexie \autocite{Rello2013a, Gala2016}. De studie van \textcite{Gala2016} benadrukt ook moeilijkheden van kinderen met dyslexie bij het lezen van woorden met onregelmatige lettergreepcombinaties. Mensen zonder dyslexie bereiken doelwaarden onder optimale omstandigheden, zoals aangegeven door de richting van de pijl op figuur \ref{img:readability-mean-fixation-duration}. Het gebruik van veelvoorkomende woorden vermindert de decodeertijd en verbetert het leesbegrip voor mensen met dyslexie.

\begin{figure}
\includegraphics[width=\linewidth]{img/readability-mean-fixation-duration.png}
\caption{De gemeten \textit{mean fixation duration} tijdens het begrijpend lezen van teksten uit het onderzoek van \textcite{Rello2013a}.}
\label{img:readability-mean-fixation-duration}
\end{figure}

Hoewel onderzoeken de positieve effecten van lexicale vereenvouding voor lezers met dyslexie onderstrepen, is er relatief weinig onderzoek gedaan naar de effecten van syntactische vereenvoudiging op kinderen en scholieren met dyslexie. In het experiment van \textcite{Linderholm2000} had het aanpassen van causale structuren een significant effect op het leesbegrip en de foutenmarge van de bevraagden met een lage leesgraad. Het onderzoek van \textcite{Leroy2013} onderzoek het effect van herstelde coherentieonderbrekingen en plaatste tekst in een logische volgorde. Zo konden zowel vaardige als minder vaardige lezers profiteren van de revisies. Verbaal parafraseren had geen significant effect op lezers met dyslexie, volgens \textcite{Rello2013c}. De bevraagden waren tijdens het onderzoek tussen de 13 en 37 jaar oud, met een gemiddelde leeftijd van 21 jaar. Het tekstformaat bleef ongewijzigd, maar lettertypes werden aangepast.

\medspace

Het onderzoek van \textcite{Nandhini2013} experimenteerde met een andere vorm van aanpassingen om de leesbaarheid van teksten te verhogen, namelijk gepersonaliseerde samenvattingen. Het experiment in het onderzoek maakt gebruik van onaangepaste zinnen uit de oorspronkelijke tekst die op maat van de lezer zijn gepresenteerd en herstructureert deze volgens de oorspronkelijke tekst. Door de belangrijkste zinnen onaangepast te laten en de structuur aan te passen, is de tekst toegankelijker voor de lezer. Hoewel de onderzoekers de rusulterende logische structuur in twijfel trokken, was de leesbaarheid van teksten bij de deelnemers significant beter dan bij de oorspronkelijke tekst, zonder negatieve effecten op het leesbegrip.

\medspace

Tot slot hebben onderzoeken aangetoond dat scholieren met dyslexie gevoeliger zijn voor veranderingen in visuele parameters, zoals lettertype, karakterafstand, tekst- en achtergrondkleur en grijswaarden. Minimalistische ontwerpen met pictogrammen behoren tot de aanbeveling om de leesbaarheid te verbeteren, evanals lettergrootte groter dan 14pt en een \textit{sans-serif}, \textit{monospaced} of \textit{roman} lettertype \autocite{Rello2013b}. Volgens \textcite{Rello2015, Bezem2016, Rello2017} zijn lichtgrijze achtergronden met zwart lettertype op een gele achtergrond, of zachtgele, -groene of lichtblauwe achtergrondkleuren de beste kleurencombinaties. Het gebruik van lettertypen zoals OpenDys heeft geen effect op lezers met of zonder dyslexie, terwijl cursieve lettertypen worden afgeraden, aldus \textcite{Rello2013b, Rello2015}.

\medspace

Tabel \ref{table:benefits-mts} somt de bewezen strategieën op samen met de bewezen voordelen tijdens het begrijpend lezen.

\begin{center}
	\begin{table}[H]
	\begin{tabular}{ | m{5cm} | m{5cm} | m{5cm} | } 
	\hline
	\textbf{Techniek} & \textbf{Bewezen voordeel} & \textbf{Bron}\\
	\hline
	Frequent woordgebruik & Lagere decodeertijd & \autocite{Rello2013a, Gala2016} \\
	& Beter leesbegrip & \autocite{Rello2013a, Gala2016} \\
	\hline	
	Verwerpen van onregelmatige lettergrepen & Verminderde decodeertijd & \autocite{Gala2016} \\
	& Beter leesbegrip & \autocite{Gala2016} \\	
	\hline
	Causale structuren aanpassen & Beter leesbegrip & \autocite{Linderholm2000} \\
	& Minder fouten bij het begrijpend lezen & \autocite{Linderholm2000} \\
	\hline	
	Tekstgebeurtenissen in een tijdsafhankelijke volgorde plaatsen & Beter leesbegrip bij het reviseren & \autocite{Leroy2013} \\
	\hline
	Coherentieonderbrekingen herstellen & Beter leesbegrip bij het reviseren & \autocite{Leroy2013} \\
	\hline
	Gepersonaliseerde samenvatting & Betere leesbaarheid & \autocite{Nandhini2013} \\
	\hline
	Zachtkleurige achtergrond & Betere leesbaarheid & \autocite{Rello2015} \\
	\hline
	Niet-cursieve, sans-serif lettertypen & Betere leesbaarheid & \autocite{Rello2013b} \\
	\hline 
	Lettertype groter dan 14pt & Betere leesbaarheid & \autocite{Rello2013b} \\
	\hline
	\end{tabular}
	\caption{Bewezen voordelen van MTS op mensen met dyslexie tijdens het begrijpend lezen.}
	\label{table:benefits-mts}
	\end{table}
\end{center}

\subsection{Aanpak voor ATS.}
% Welke toepassingen, tools en modellen zijn er beschikbaar om Nederlandse geautomatiseerde tekstvereenvoudiging met AI mogelijk te maken? 

De laatste evolutie in machinaal leren biedt een mogelijkheid om dit proces te automatiseren. \textit{Automatic text simplification} of ATS is een onderdeel van natuurlijke taalverwerking binnen machinaal leren (ML). Dit omvat technieken zoals tekstanalyse, taalherkenning -en generatie, spraakherkenning -en synthese en semantische analyse. NLP stelt computers in staat om menselijke taal te begrijpen en te communiceren op een natuurlijke manier. De begrippen die volgen worden behandeld in \textcite{Sohom2019, Eisenstein2019} en zijn cruciaal voor de daaropvolgende concepten.

\medspace

Zo dient tokenisatie om zinnen op basis van tokens te splitsen. Er zijn vier manieren om tokens in een tekst te splitsen en zo een woordenschat op te bouwen, namelijk op woord-, karakter-, subwoord- en zinniveau, volgens onderzoek van \textcite{Menzli2023}. Bij karaktertokenisatie neemt de inputlengte toe, maar dit heeft volgens \textcite{Ribeiro2018} weinig bruikbare \textit{use cases}. Het opsplitsen van zeldzame woorden in kleinere stukken om een woordenschat op te bouwen biedt voordelen ten opzichte van woordtokenisatie, aldus \autocite{Iredale2022}.

\medspace

In NLP baseert het lemmatiseren zich op stemming, een NLP-taak waarbij deze de stam van een woord neemt, maar ook rekening houdt met de betekenis van elk woord. Er zijn Nederlandstalige modellen beschikbaar voor lemmatiseren, zoals JohnSnow\footnote{https://nlp.johnsnowlabs.com/2020/05/03/lemma\_nl.html}. Omgekeerd lemmatiseren werkt door een afgeleide vorm van de stam te bepalen, bijvoorbeeld enkelvoud of meervoud voor zelfstandige naamwoorden als ’hond’ \autocite{Eisenstein2019}. Bij het parsen krijgt elk woord of zinsdeel een label toegewezen, zoals zelfstandig naamwoord, bijwoord, werkwoord, bijzin of stopwoord. De identificatie van zinsdelen heet chunking. Parsing is vatbaar op ambiguïteit omdat bijvoorbeeld ‘een plant’ niet gelijk is aan de vervoeging van het werkwoord ’planten’ \autocite{Eisenstein2019}.

\medspace

Om de betekenis van elk woord in een tekst te begrijpen, moet een machine in staat zijn om de betekenis achter elk token te begrijpen. Dit is waar \textit{sequence labeling} om de hoek komt kijken, volgens \textcite{Eisenstein2019}. Elk woord in een tekst krijgt een label voor \textit{Part-of-Speech} (PoS) of \textit{Named-Entity-Recognition} (NER). Deze fase van NLP achterhaalt de structuur van een tekst. PoS-tagging richt zich op de grammaticale categorieën van woorden, terwijl NER-labeling zich richt op het herkennen van specifieke entiteiten in een tekst. Bij PoS-tagging worden de woorden in een zin geanalyseerd. Elk woord krijgt een koppeling aan een grammaticale categorie, zoals een zelfstandig naamwoord, werkwoord, bijvoeglijk naamwoord of bijwoord. \textit{PoS-tagging} helpt bij het begrijpen van de syntactische structuur van een zin en is nuttig bij parsing en machinevertaling. Een voorbeeld van PoS-tagging is te zien in figuur \ref{fig:pos-labeling} en is afkomstig uit \textcite{Bilisci2021}.

\begin{center}
	\begin{figure}[H]
		\includegraphics[width=15cm]{img/poslabeling.png}
		\caption{Voorbeeld van PoS-labeling uit het artikel van \textcite{Bilisci2021}.}
		\label{fig:pos-labeling}
	\end{figure}
\end{center}

Met NER-labeling kunnen systemen zo namen van personen, organisaties en locaties herkennen en classificeren. Het haalt specifieke informatie uit een tekst, zoals het identificeren van de namen van personen, plaatsen of bedrijven die in nieuwsartikelen, of het extraheren van belangrijke data of getallen uit financiële rapporten, aldus \textcite{Jurafsky2014}. Er zijn vier vormen van NER-labeling, zoals beschreven door \textcite{Li2018}: \textit{dictionary-based}, \textit{rule-based}, \textit{ML-based} en \textit{deep learning-based}. De eerste twee maken gebruik van vooraf gedefinieerde woordenboeken en regels, terwijl de laatste twee gebruik maken van statistische of neurale netwerken om te leren hoe entiteiten te herkennen. Elke vorm maakt gebruik van representaties om entiteiten te modelleren. \textcite{Poel2008} hebben een neuraal netwerkmodel onderzocht voor PoS-tagging van Nederlandstalige teksten. Het model behaalde een nauwkeurigheid van 97,88\% voor bekende woorden en 41,67\% voor onbekende woorden en maakte gebruik van de Corpus Gesproken Nederlands (CGN) als trainingsdata. In de verwerking van tekst maken NLP-systemen gebruik van embeddings om woorden numeriek te representeren. Traditionele word embeddings bouwen een woordenschat op zonder de betekenis ervan in context te begrijpen. Contextuele word embeddings begrijpen wel de context van een woord \autocite{Eisenstein2019}. 

\section{De verschillende soorten ATS}

Tekstvereenvoudiging kan bijdragen aan het begrijpen van complexe informatie. Zoals onderzocht door \textcite{Siddharthan2014}, zijn er vier soorten transformaties bij ge- automatiseerde tekstvereenvoudiging, waaronder lexicale vereenvoudiging, waarbij eenvoudigere synoniemen de complexe woorden vervangen. Dit heet \textit{lexical simplification} (LS) of lexicale vereenvoudiging. Bijvoorbeeld, ‘klevend’ kan ‘adhesief’ vervangen. \textcite{Kandula2010} noemt twee manieren om lexicale vereenvoudiging te bewerkstelligen: het vervangen van het woord door een synoniem of het genereren van extra uitleg. De zinsstructuur blijft hetzelfde en de kerninhoud en benadrukking van de tekst blijven behouden. Het doel van lexicale vereenvoudiging is om de moeilijkheidsgraad van de woordenschat in een zin of tekst te verlagen.

\medspace

Diverse onderzoeken hebben aangetoond dat lexicale vereenvoudiging een belangrijke bijdrage kan leveren aan het begrijpen van complexe informatie, en in dit kader wordt de pipeline zoals weergegeven in figuur \ref{img:pipeline-lexical-simplification} vaak gebruikt, bijvoorbeeld in onderzoeken van \textcite{Paetzold2016, Bingel2018, Bulte2018}. Deze pipeline omvat bij de vermelde onderzoeken telkens minstens vier stappen, waarbij de eerste stap \textit{Complex Word Identification} (CWI) is, een gesuperviseerde NLP-taak die moeilijke woorden of \textit{multi-word expressions} (MWE) in een tekst identificeert \autocite{Shardlow2013, Gooding2019}. De LS, waarbij eenvoudigere synoniemen de moeilijkere woorden vervangen, komt na CWI. Hier kunnen ook verklarende beelden of definities komen \autocite{Zeng2005, Kandula2010}. Een goede uitvoering is van cruciaal belang bij CWI, omdat een lage recall van dit component zal resulteren in een uitvoertekst zonder vereenvoudiging van moeilijke woorden zoals opgemerkt door \textcite{Shardlow2013}. Er zijn verschillende manieren geïdentificeerd om substitutiegeneratie uit te voeren, zoals opgesomd in tabel \ref{table:lexical-databases}. Recenter onderzoek, zoals dat van \textcite{Zhou2019}, gebruikt ook een extra \textit{Substitution Ranking} (SR) stap om substituties te rangschikken op basis van relevantie. 

\begin{figure}[H]
	\includegraphics[width=15cm]{img/lexical-simplification-pipeline.png}
	\caption{Een pipeline voor LS volgens \textcite{Althunayyan2021}.}
	\label{img:pipeline-lexical-simplification}
\end{figure}

\begin{center}
\begin{table}[H]
	\begin{tabular}{ | m{7cm} | m{7cm} | } 
		\hline
		\textbf{Databank} & \textbf{Ondersteunde talen} \\
		\hline
		Engels & WordNet \\
		& SWORDS \\
		& LSBert \\
		\hline
		Nederlands & Celex \\
		& NT2Lex \\
		& Cornetto \\
		\hline
		Meertalig (Engels, Duits, Spaans en Portugees) & PHOR-in-One \\
		\hline	
	\end{tabular}
	\caption{Beschikbare Nederlandstalige, Engelstalige en meertalige lexicale databanken anno mei 2023.}
	\label{table:lexical-databases}
\end{table}
\end{center}

\textit{Syntactic simplification} (SS) of syntactische vereenvoudiging is een techniek om de complexiteit van een zin te verminderen. Het past de grammatica en zinsstructuur van de tekst aan. Dit gebeurt door het combineren van twee zinnen tot één eenvoudigere zin of door de syntax te vereenvoudigen, terwijl de semantiek bewaren blijft. \textcite{Kandula2010} onderzochten de ontwikkeling van dergelijk model voor medische informatie. Het model bestaat uit drie modules, die zinnen met meer dan tien woorden vereenvoudigen en eventueel vervangen door kortere zinnen. Het omvat een \textit{PoS Tagger}, een Grammar Simplifier en een Output Validator als onderdelen van de architectuur.

\begin{enumerate}
	\item Ten eerste wordt de \textit{PoS Tagger}-functie uit het open-source pakket OpenNLP\footnote{https://opennlp.apache.org/} gebruikt.
	\item Vervolgens splitst de \textit{Grammar Simplifier}-module lange zinnen in kortere zinnen door middel van het identificeren van POS-patronen en het toepassen van transformatieregels.
	\item Tot slot controleert de \textit{Output Validator}-module de grammatica en leesbaarheid van de output van de Grammar Simplifier.
\end{enumerate}  

ATS is geen nieuw concept. Volgens onderzoeken van \textcite{Canning2000, Siddharthan2006} zijn de eerste aanpakken op ATS gebouwd op rule-based modellen. Deze modellen bewerken de syntax door zinnen te splitsen, te verwijderen of de volgorde van de zinnen in een tekst aan te passen. LS komt hier niet aan de pas. Recentere onderzoeken van \textcite{Coster2011, Bulte2018} verduidelijken hoe ontwikkelaars LS en SS kunnen combineren.

\medspace

Vroegere onderzoeken tonen aan dat geautomatiseerde tekstvereenvoudiging al geruime tijd bestaat. Zo hebben \textcite{Canning2000} en \textcite{Siddharthan2006} onderzocht dat de eerste methoden gebaseerd waren op \textit{rule-based} modellen die de syntaxis van de tekst bewerken door zinnen te splitsen, te verwijderen of te herschikken. Lexicale vereenvoudiging speelde hierbij geen rol. Enkel de recentere onderzoeken van  Pas bij recentere onderzoeken, zoals die van \textcite{Coster2011, Bulte2018} tonen de mogelijkheid aan om LS en SS te combineren.

\medspace

Om wetenschappelijke artikelen toegankelijker en begrijpelijker te maken, is het van belang om de kernpunten van een artikel op een duidelijke en beknopte manier samen te vatten. Hoewel samenvatten niet gericht is op het vereenvoudigen van de tekst, is het wel een techniek die noodzakelijk is om de semantiek achter een tekst met zo min mogelijk woord- of tekens te kunnen begrijpen. Full-text-search en gepersonaliseerde informatiefiltering benadrukken het belang van deze op maat gemaakte samenvattingen. Een samenvattingssysteem bestaat uit drie fases: analyse van de brontekst, identificatie van de kernpunten en samenvoeging van deze kernpunten tot één overzichtelijke tekst. Het machinaal samenvatten van teksten kan op twee manieren: door extractie en door abstractie \autocite{Hahn2000, Dubay2004}.

\medspace

Het proces van extraherend samenvatten markeert de belangrijkste zinnen in een tekst en herschrijft ze. Dit kan echter leiden tot onsamenhangende uitvoertekst, zoals \textcite{Khan2014} heeft aangetoond. Er zijn verschillende methoden beschikbaar om de kernzinnen te bepalen, zoals woordfrequentie, zinpositie -en gelijkenissen, de \textit{cue}-methode, titels, \textit{proper nouns}, woordgebruik en de afstand tussen \textit{text unit entities}, aldus \textcite{Khan2014}. Verschillende technieken voor het extraherend samenvatten van teksten, waaronder graafgebaseerde methoden, maximal marginal relevance (MMR) en metaheuristiek gebaseerde ES, zijn onderzocht door \textcite{Verma2020}. Tabel \ref{table:extractive-summarization} omschrijft deze drie methoden verder.

\begin{center}
	\begin{table}[H]
	\begin{tabular}{ | m{4cm} | m{12cm} | } 
		\hline
		MMR-gebaseerde ES & Deze techniek gebruikt de maximaal marginale relevantiescore (MMR) om de relevantie en diversiteit van gemarkeerde zinnen te bepalen. Dit voorkomt dat geselecteerde zinnen elkaar niet overlappen in inhoud en relevantie. Deze methode kan leiden tot betere samenvattingen, maar vereist meer rekenkracht en tijd dan de andere twee technieken. \\
		\hline
		Graafgebaseerde ES & Deze techniek vertegenwoordigt een document als een graaf van zinnen. Deze vorm gebruikt algoritmen om de belangrijkste zinnen te bepalen en redundantie te vermijden. Dit kan zowel voor lange wetenschappelijke artikelen als korte nieuwsartikelen goede resultaten opleveren \autocite{McDonald2007, Lin2010}. \\ 
		\hline
		Metaheuristiek-gebaseerde ES & Deze techniek maakt gebruik van optimalisatie-algoritmen om de belangrijkste zinnen in een tekst te vinden \autocite{Premjith2015, Verma2020}. Evaluatiefuncties kunnen echter afhankelijk van de gebruikte criteria in een lokaal optimum vastlopen \autocite{Rani2021}. \\
		\hline
	\end{tabular}
	\caption{Drie manieren om extraherende samenvatting mogelijk te maken volgens \textcite{Verma2020}.}
	\label{table:extractive-summarization}
	\end{table}
\end{center}

Vooroordelen of \textit{bias} kan de extraherende samenvatting van nieuwsartikelen beïnvloeden, zo blijkt uit experimenten uitgevoerd door \textcite{McKeown1999}. Deze vorm van samenvatten neemt de zinnen over zoals ze zijn. \textcite{Hahn2000} bouwden verder op deze experimenten door het combineren van \textit{knowledge-rich} en \textit{knowledge-poor} methoden, wat resulteerde in significante verbeteringen. Bij het extraherend samenvatten is het van belang om de meest relevante tekstgedeeltes te selecteren, meestal in de vorm van zinnen. Om de lexicale en statistische relevantie van een zin te kunnen bepalen, noemen \textcite{Hahn2000} twee methoden:

\begin{itemize}
	\item Het lineaire gewichtsmodel, waarbij elke teksteenheid wordt gewogen op basis van factoren zoals de positie van de zin en het aantal keren dat deze voorkomt.
	\item Het gewichtsmodel op basis van de statistische relevantie van een eenheid, waarbij rekening wordt gehouden met de aanwezigheid van woorden in titels.
\end{itemize}

Om de nauwkeurigheid van modellen te verbeteren, ontwikkelden \textcite{Nallapati2017} \textit{SummaRuNNer}. Dit model kan teksten extraherend samenvatten door een neuraal netwerk. Zo gebruikt het \textit{PyTorch} en bestaat het uit drie modellen: een recurrent neuraal netwerk, een convolutioneel recurrent neuraal netwerk en een \textit{hiërarchical attention network}.

\medspace

Extraherende samenvattingen kunnen leiden tot een onsamenhangende tekst. Abstraherende samenvatting kan een oplossing bieden, omdat het rekening houdt met de samenhang van een tekst. Onderzoek van \textcite{Gupta2019} wijst twee benaderingen voor abstraherende samenvatting: semantisch-gebaseerd en structuurgebaseerd. De structuurgebaseerde methode gebruikt regels om belangrijke informatie in de tekst te vinden, maar dit kan leiden tot samengevatte zinnen van lage taalkundige kwaliteit en grammaticale fouten. De semantisch-gebaseerde benadering daarentegen gebruikt de betekenis van de tekst om korte en duidelijke samenvattingen te maken met minder redundante zinnen en een betere taalkundige kwaliteit. Een extra parsingfase kan van pas komen volgens de onderzoeken. Onderzoeken van \textcite{Suleiman2020, Cao2022} wijzen \textit{deep learning}-methoden uit om automatisch abstraherende samenvattingen te genereren. Zo kunnen RNN's, CNN's en Seq2Seq dienen om abstraherende samenvatting mogelijk te maken. 

\medspace

Ontwikkelaars moeten een andere aanpak gebruiken wanneer zij een systeem willen ontwikkelen voor \textit{long text summarization} of LTS, zoals bij boeken of wetenschappelijke tijdschriften. Zo kan het opsplitsen van de tekst leiden tot het breken van samenhangende paragrafen. Dat kan later resulteren in redundante tekst in het samengevatte document. Zo raadden onderzoeken van \textcite{Hsu2018, Huang2019} aan om zowel extraherend als abstraherende samenvatting toe te passen. Om deze reden zou een \textit{hybrid summarization pipeline} twee fasen moeten bevatten: een inhoudselectiefase waarbij het systeem kernzinnen extraheert, gevolgd door een parafraserende fase.

\medspace

Tot slot moeten ontwikkelaars rekening houden met de doelgroep wanneer zij een systeem of model uitkiezen voor tekstvereenvoudiging of samenvatting. Zo moeten ontwikkelaars rekening houden met de individuele behoeften en uitdagingen van elke scholier, volgens \textcite{Gooding2022}. Dyslexie kan zich namelijk op verschillende manieren uiten bij verschillende scholieren, waarbij bijkomende symptomen niet altijd van invloed zijn op de spellingprestaties van een scholier. Om deze reden is het belangrijk om een toepassing te ontwerpen die rekening houdt met de diversiteit van dyslexie.

\section{Beschikbare tools en taalmodellen}
\label{sec:beschikbare-tools-en-taalmodellen}

Het kan moeilijk zijn voor scholieren met dyslexie om goed te lezen en te schrijven. Gelukkig zijn er verschillende softwareprogramma's en tools beschikbaar om hen te ondersteunen. Deze sectie gaat in op de nationale en internationale software die specifiek is ontworpen voor scholieren met dyslexie om hen te helpen bij het lezen van teksten.  Er zal voornamelijk worden gekeken naar beschikbare software in Vlaamse middelbare scholen, chatbots zoals Bing Chat en ChatGPT, en software die speciaal is ontwikkeld om dyslexie bij het lezen te ondersteunen. Deze sectie beantwoordt de volgende deelvraag: 

\begin{itemize}
	\item Welke toepassingen, tools en modellen zijn er beschikbaar om Nederlandstalige geautomatiseerde tekstvereenvoudiging met AI mogelijk te maken?
\end{itemize}

Scholieren met dyslexie krijgen in het middelbaar onderwijs enkel ondersteuning in de vorm van voorleessoftware \autocite{DeCraemer2018, OnderwijsVlaanderen2023}. Het ministerie van Onderwijs in Vlaanderen biedt licenties aan voor verschillende softwarepakketten zoals SprintPlus, Kurzweil3000, Alinea Suite, IntoWords en TextAid, die scholieren kunnen gebruiken om zinnen te markeren en deze vervolgens samen te vatten. Het samenvatten gebeurt echter op een manier waarbij de zinnen lexicaal en syntactisch identiek blijven. Helaas bieden deze softwarepakketten geen functie voor het vereenvoudigen van teksten. Volgens \textcite{Tops2018} is het belangrijk om deze software zo vroeg mogelijk in de schoolcarrière te introduceren, zodat scholieren er snel vertrouwd mee raken en het optimaal kunnen gebruiken in verdere studies. Hoewel \textcite{Tops2018} de handige aspecten van deze software benadrukt, is het te laat om deze software pas in het hoger onderwijs te introduceren.

\medspace

Momenteel beschikken de voorleessoftware beperkte LS-functionaliteiten. Dit onderstreept de noodzaak aan nieuwe erkende tools die tekstvereenvoudiging in het onderwijs mogelijk maken. Tools zoals Simplish en Rewordify kunnen een oplossing bieden. Hoewel Simplish oorspronkelijk Engelstalig is, kan het inmiddels vereenvoudigde teksten genereren van Nederlandstalige teksten. Deze functionaliteit is echter enkel betalend voor niet-Engelstalige teksten. Vervolgens kan Rewordify enkel Engelstalige teksten vereenvoudigen. Tot slot vindt de literatuurstudie weinig online \textit{proof-of-concepts} terug. Daarnaast bieden de toepassingen geen transparantie over hun gebruikte taalmodel, waardoor ontwikkelaars de logica en werking van deze toepassingen niet kunnen reproduceren. 

\medspace

Voor samenvatting zijn er echter meer tools beschikbaar. Enkele voorbeelden hiervan zijn Resoomer, Paraphraser, Editpad, Scribbr en Quillbot. Al zijn er onderzoeken over ATS-technieken voor scholieren met dyslexie, het aantal onderzoeken over samenvatten voor deze doelgroep is schaars. Zoals eerder aangehaald is er wel onderzoek gedaan naar de verschillende manieren om een tekst samen te vatten, maar er is geen toepassing of onderzoek dat dit concreet uitwerkt. \textcite{Sanja2021} wijzen erop dat toepassingen voor tekstvereenvoudiging regelmatig als \textit{showcase} van de technologie ontwikkeld worden en zelden tot weinig rekening houden met gepersonaliseerde samenvatting om zo rekening te houden met de verschillende noden.

\medspace

Er zijn weinig toepassingen beschikbaar om wetenschappelijke artikelen te vereenvoudigen, maar er bestaan gratis en betalende toepassingen. Zo reiken SciSpace\footnote{https://typeset.io/} en Scholarcy\footnote{https://www.scholarcy.com/} ATS specifiek voor wetenschappelijke artikelen aan. Hero vloeide verder uit het onderzoek van \textcite{Bingel2018}, waarbij de onderzoekers een toepassing voor gepersonaliseerde ATS voor kinderen met dyslexie ontwikkelden. Hoewel Hero een oplossing aanreikt, slaagt de toepassing er niet in om wetenschappelijke artikelen te vereenvoudigen. Wel kunnen scholieren deze browserextensie gebruiken voor selecte Engelstalige nieuwssites. Tabel \ref{table:overview-tools} geeft een overzicht van prevalente tools die momenteel tekstvereenvoudiging aanbieden.

\begin{center}
	\begin{table}[H]
		\begin{tabular}{ | m{4cm} | m{11cm} | } 
		\hline
		\textbf{Tool} & \textbf{Algemene functionaliteit} \\
		\hline
		Sprintplus & \\
		Kurzweil3000 & \\
		Alinea Suite & Voorleessoftware met ondersteuningsmogelijkheden voor moeilijke woordenschat \\
		IntoWords & \\
		TextAid & \\
		\hline
		Resoomer &  \\
		Paraphraser & \\
		Editpad & Samenvattingstool \\
		Scribbr & \\
		Quillbot & \\
		\hline
		SciSpace & Samenvattingstool specifiek voor wetenschappelijke artikelen. \\
		Scholarcy & \\
		\hline
		Simplish & Tool voor tekstvereenvoudiging met bijhorende analyse\\
		Rewordify & \\
		\hline
		\end{tabular}
	\caption{Overzicht van gekende voorleessoftware, tekstvereenvoudigings- en samenvattingstools die intuïtief zijn ontwikkeld voor de eindgebruiker (leerkracht of scholier).}
	\label{table:overview-tools}
	\end{table}
\end{center}

Ontwikkelaars kunnen ook zelf aan de slag gaan. Zo beschikt HuggingFace (HF) een breed scala aan API's en tools die gemakkelijk te downloaden en trainen zijn voor \textit{pretrained} modellen voor veelvoorkomende NLP-taken, zoals \textit{text classification}, taalmodellering en samenvatting. Tekstvereenvoudiging is in mindere mate aanwezig. Verder kunnen ontwikkelaars deze modellen \textit{finetunen} op specifieke datasets om modellen te bouwen voor gepersonaliseerde NLP-taken. Tot slot geeft tabel \ref{table:huggingface-models} HF-taalmodellen met hun respectievelijke casus. De volgende taalmodellen neemt de tabel op: Google Pegasus\footnote{https://ai.googleblog.com/2020/06/pegasus-state-of-art-model-for.html}, Longformer Encoder-Decoder\footnote{https://huggingface.co/docs/transformers/model\_doc/led}, Simplification\footnote{https://huggingface.co/haining/scientific\_abstract\_simplification}, BART Large Scientific Summarisation\footnote{https://huggingface.co/sambydlo/bart-large-scientific-lay-summarisation}, T5 finetuned text simplification model\footnote{https://huggingface.co/husseinMoh/t5-small-finetuned-text-simplification} en Keep It Simple\footnote{https://huggingface.co/philippelaban/keep\_it\_simple}

\begin{center}
	\begin{table}[H]
	\begin{tabular}{ | m{4cm} | m{12cm} | } 
		\hline
		\textbf{Taalmodel} & \textbf{Specifieke casus} \\ \hline
		Google Pegasus & Samenvattingstaken voor kort tot middelgrote documenten. \\
		\hline
		Longformer Encoder-Decoder (LED) & Samenvatting van lange wetenschappelijke artikelen \\
		\hline
		Haining Scientific Abstract Simplification & Het lexicaal vereenvoudigen van wetenschappelijke artikelen. \\
		\hline
		BART Large Scientific Summarisation & Het samenvatten van wetenschappelijke artikelen. \\
		\hline
		T5 finetuned text simplification model & Tekstvereenvoudiging voor algemeen gebruik. \\
		\hline
		Keep It Simple & Ongesuperviseerde tekstvereenvoudiging met ATS. \\
		\hline
	\end{tabular}
		\caption{Beschikbare en ge-finetunede HF-taalmodellen.}
		\label{table:huggingface-models}
	\end{table}
\end{center}

Dankzij de sterke evolutie in data en AI, kon de grootte van deze taalmodellen sterk vergroten. Zo is GPT-3 een opkomend \textit{Large Language Model} of LLM. OpenAI ontwikkelde dit taalmodel en paste daarvoor een tweestapsleerparadigma toe \autocite{Radford2019, Li2022}. 

\begin{itemize}
	\item Allereerst komt er een ongesuperviseerde training aan bod met een \textit{language modelleing goal}. Ontwikkelaars trainden dit model op niet-gecategoriseerde data van het internet met datasets zoals Common Crawl, WebText2, Books1, Books2 en Wikipedia.
	\item Tenslotte finetunen de ontwikkelaars dit verder. Om een correcte respons van het model te krijgen, passen de ontwikkelaars \textit{reinforcement learning} toe.
\end{itemize}

GPT-3 beschikt over meerdere versies, waaronder GPT-3.5 die als engine dient voor ChatGPT. Omdat onbegrijpelijke en ontoegankelijke zinsstructuren niet alleen voor leken, maar ook voor vakexperten een obstakel vormen, is het belangrijk om te benadrukken dat GPT-3.5 gericht is op conversationele doeleinden, terwijl GPT-3 in het algemeen bedoeld is om met hoogstens één prompt te werken \autocite{McNutt2014, Hubbard2017}. Verder reikt OpenAI documentatie uit voor het GPT-3 taalmodel. Daarin vermelden ze vier \textit{engines}, namelijk Davinci, Curie, Babbage en Ada. In maart 2023 kwam daar een vijfde engine bij, namelijk GPT-3 Turbo die fungeert als achterliggende engine voor Chat-GPT. Davinci-003 is het meest geavanceerde model, geschikt voor taken zoals het schrijven van essays en het genereren van code, en levert de meest menselijke antwoorden. Curie is goed in nuance, maar minder menselijk dan Davinci, terwijl Ada en Babbage minder krachtig zijn en beter geschikt zijn voor eenvoudige taken zoals het aanvullen van tekst en sentimentanalyse \autocite{Brockman2023}. Deze engines gebruiken dezelfde set hyperparameters, die ontwikkelaars kunnen aanpassen. Tabel \ref{table:gpt-3-parameters} somt deze parameters verder op.

\begin{center}
	\begin{table}[H]
	\begin{tabular}{ | m{2.5cm} | m{7cm} | m{4.5cm} | }
		\hline
		\textbf{Parameter} & \textbf{Omschrijving} & \textbf{Mogelijke waarden} \\ \hline
		\textit{model} & De GPT-3 engine die ontwikkelaars kunnen gebruiken. & davinci, curie, babbage, ada, text-davinci-002, text-curie-001, text-babbage-001, text-ada-001, davinci-codex \\ \hline
		\textit{temperature} & De gulzigheid van het generatief model. Een lagere waarde kan voorspelbare tekst teruggeven. Hogere waarden daarentegen kunnen onverwachtse tekst teruggeven, wat beter werkt bij creatieve toepassingen. & Een kommagetal tussen 0 en 1. \\ \hline
		\textit{max tokens} & Het maximaal aantal tokens (woorden of subwoorden) dat het generatief model kan teruggeven. & Een getal tussen 1 and 2048. \\ \hline
		\textit{top-p} & Vergelijkbaar met temperature, maar deze waarde onderhoudt de \textit{probability distribution} voor \textit{common tokens}. Hoe lager de waarde, hoe hoger de woordfrequentie van de gegenereerde tekst. Toepassingen gericht op nauwkeurigheid maken beter gebruik van hoge waarden. & Een kommagetal tussen 0 en 1. \\
		\hline
		stop & Een tekstwaarde (woord/symbool) tot waar het model zal genereren. & String-waarden \\
		\hline
		\textit{presence penalty} & Factor die bepaalt hoe regelmatig woorden voorkomen & Kommagetallen tussen 0 en 1 \\
		\hline
	\end{tabular}
		\caption{Tabel met alle GPT-3 parameters.}
		\label{table:gpt-3-parameters}
	\end{table}
\end{center}

Hoewel onderzoeken rond GPT-3 nog volop in ontwikkeling zijn, bestaan er vergelijkende onderzoeken naar de mogelijkheden van dit LLM. Onderzoek van \textcite{Binz2023} wijst uit dat de mean-regret score kan dienen als maatstaf om de menselijkheid van antwoorden te beoordelen. Deze studie wees verder uit dat deze modellen capabel zijn om menselijke antwoorden te produceren, zoals geïllustreerd in figuur \ref{img:mean-regret-chatgpt}. Uit het experiment van \textcite{Goyal2022} blijkt dat \textit{zero-shot} samenvattingen met GPT-3 beter presteren dan gefinetunede modellen. \textcite{Li2022} benadrukt dat GPT-3 overkill is voor sentimentanalyse. Daarvoor haalt het onderzoek aan om een kleinschaliger taalmodel te gebruiken. Daarnaast beschikken LLM's over een grotere ecologische voetafdruk, waarvoor onderzoeken van \textcite{Strubell2019, Simon2021} deze praktijk ook afraden. Uiteindelijk bestaan er al enkele tools die gebruikmaken van de GPT-3 API, waaronder Jasper AI en ChatSonic zoals aangehaald in het onderzoek van \textcite{Mottesi2023}. Experten zoals \textcite{Roose2023, Garg2022} halen het GPT-3 model en ChatGPT aan als de toekomst voor gepersonaliseerde en adaptieve uitleg aan scholieren. Bing Chat biedt een extra dat revolutionair kan zijn bij het opzoeken van uitleg voor zoektermen, zonder het verlies aan bronvermelding. Dankzij de personalisering van de prompts biedt GPT-3 mogelijkheden aan voor toepassingen in het onderwijs, aldus \textcite{Roose2023, Garg2022}.

\begin{figure}[H]
	\begin{center}
		\includegraphics[width=\linewidth]{img/chatgpt-engines-mean-regret.png}
		\caption{Een experiment op de \textit{mean-regret} van GPT-3 engines uit \textcite{Binz2023}.}
		\label{img:mean-regret-chatgpt}
	\end{center}
\end{figure}

Met neurale netwerken kan het taalmodel patronen in de input herkennen. Deze patronen dienen om voorspellingen te maken over de output \autocite{Liu2020}. Via prompts hebben mensen toegang tot krachtige taalmodellen, zoals GPT-3 of BERT \autocite{McFarland2023, Harwell2023}. Figuur \ref{img:prompt-engineering} illustreert de werking van deze vaardigheid. Onderzoek van \textcite{Liu2020} benadrukt het belang van goed opgestelde prompts. Hiervoor kunnen eindgebruikers de technieken in tabel \ref{table:techniques-for-good-prompts}. Zo moeten deze werk kunnen produceren op maat van het doel. Zo benadrukt de onderzoeker dat een prompt concreet moet zijn. Bij het opstellen van een prompt voor een zoekopdracht is het cruciaal om voldoende parameters op te nemen om te voorkomen dat het model te algemeen blijft en afwijkt van de intentie van de gebruiker. Effectieve prompt engineering voor AI leidt tot hoogwaardige trainingsgegevens, waardoor het AI-model nauwkeurige voorspellingen en beslissingen kan maken \autocite{Liu2020}.

\begin{table}
	\begin{tabular}{| m{5cm} | m{5cm} | }
		\hline
		\textbf{Prompttechniek} & \textbf{Bron} \\ \hline
		Duidelijke scope & \autocite{McFarland2023} \\ \hline
		Specifieke sleutelwoorden & \autocite{McFarland2023} \\ \hline
		De context waarin de vraag zich afspeelt. & \autocite{McFarland2023} \\ \hline
		Gepersonaliseerde keuzes & \autocite{McFarland2023} \\ \hline
	\end{tabular}
	\caption{Technieken voor concrete en goed opgestelde prompts.}
	\label{table:techniques-for-good-prompts}
\end{table}

\begin{figure}[H]
	\begin{center}
		\includegraphics[width=\linewidth]{img/prompt-engineering-medium.png}
	\end{center}
	\caption{De werking van \textit{prompt engineering} volgens \textcite{McFarland2023}}
	\label{img:prompt-engineering}
\end{figure}

Om prompts te kunnen hergebruiken, bouwden ontwikkelaars \textit{prompt patterns} op. Deze patronen bieden herbruikbare oplossingen voor veelvoorkomende problemen in een bepaalde context, vooral bij de interactie met LLM's. \textcite{White2023} benoemt vier \textit{prompt patterns}:

\begin{itemize}
	\item	\textit{Intent prompts} waarbij een LLM een instructie krijgt met een specfiek verwacht antwoord.
	\item	\textit{Restriction prompts} die het antwoord van een LLM inperkt. Deze pattern is noodzakelijk om een LLM binnen de lijnen te houden.
	\item 	\textit{Contextualization prompts} verzekeren dat de output van een LLM relevant is. De prompt bevat de context.
	\item	\textit{Expansion/reduction prompts} genereren een beknopte output met voldoende details. 
\end{itemize}

BERT is een meertalige LLM die gebruikmaakt van \textit{contextual word embeddings}. Onderzoekers trainden het model op 110 miljoen parameters uit 104 verschillende talen, waaronder Nederlands. Voor de Nederlandse taal zijn er twee varianten van BERT beschikbaar: RobBERT en BERTje. GPT-3 en BERT beschikken over een verschillende architectuur. Zo is volgens \textcite{Mottesi2023} GPT-3 een autoregressief model die alleen rekening houdt met de linkercontext bij het voorspellen of genereren van tekst. BERT daarentegen is een bidirectioneel model, waarbij het model zowel de linker- als de rechtercontext in overweging neemt. De bidirectionele werking van BERT is geschikt voor sentimentanalyse, waarbij begrip van de volledige zincontext noodzakelijk is. Omdat GPT-3 toegang heeft tot meer informatie (45 TB) dan BERT (3 TB), kan dit een voordeel bieden tijdens taalbewerkingen zoals vereenvoudigen, parafraseren of vertalen. Tot slot verschillen de LLM's ook qua grootte. Hoewel beide modellen erg groot zijn, is GPT-3 aanzienlijk groter dan BERT, mede door de uitgebreide trainingsdatasetgrootte \autocite{Brown2020}. Er is echter een nieuw generatief taalmodel genaamd LLaMa, dat sterker is dan GPT-3 en vergelijkbare modellen, terwijl het slechts tien keer minder parameters gebruikt. Helaas is LLaMa momenteel nog niet beschikbaar als online webtoepassing of API \autocite{Hern2023, Touvron2023}. Figuur \ref{img:graph-language-models} toont de evolutie van \textit{pre-trained} taalmodellen. Zo volgt de LLM-performantie ten opzichte van het aantal parameters van een LLM een lineaire functie.

\begin{figure}[H]
	\begin{center}
		\includegraphics[width=\linewidth]{img/graph-language-models.png}
		\caption{De evolutie van LLM's. Bron: \autocite{Simon2021}}
		\label{img:graph-language-models}
	\end{center}
\end{figure}

Microsoft en OpenAI werken nauw samen. Zo gebruikt Bing Chat ook het GPT-3 taalmodel. Met de Prometheus-technologie kan deze chatbot verder bouwen en verwijzingen bieden naar andere websites \autocite{Ribas2023}. Zo maakt Bing Chat verwijzingen naar bestaande online documenten dankzij de Bing index-, ranking- en antwoordresultaten. Prometheus combineert dit met het redeneervermogen van OpenAI’s GPT-modellen. Via de \textit{Bing Orchestrator} genereert Prometheus iteratief een set \textit{internal queries} om zo binnen de gegeven gesprekscontext nauwkeurige antwoorden op gebruikersvragen te bieden \autocite{Ribas2023}. Bing Chat is beschikbaar als webpagina en browserextensie voor Microsoft Edge. Het gebruik van extraherende en abstraherende samenvattingen maakt de chatbot interessant, al bestaat er nog te weinig onderzoek naar de credibiliteit en correctheid van de verstrekte verwijzingen. Daarnaast beschikt Bing Chat niet over een API, waardoor ontwikkelaars geen CLI-toepassingen met deze technologie kunnen maken.

\section{De valkuilen bij AI en NLP.}

AI en ML zijn volop in groei. NLP gebruikt AI en ML om menselijke taal te verwerken, terwijl NLU deze technologieën gebruikt om menselijke taal te begrijpen. Hoewel deze technologieën veelbelovend zijn, moeten AI-ontwikkelaars rekening houden met veelvoorkomende en genegligeerde uitdagingen en valkuilen \autocite{Sciforce2020, Roldos2020, Khurana2022}. Deze sectie beantwoordt de volgende onderzoeksvraag: 

\begin{itemize}
	\item Met welke valkuilen bij taalverwerking met AI moeten ontwikkelaars rekening houden?
\end{itemize}

Het ontwikkelen van NLP- en NLU-toepassingen is duur en vormt een obstakel voor IT-professionals vanwege factoren zoals het gebrek aan NLP-expertise, de kwaliteit en kwantiteit van data, de integratie en deployment van modellen en de transparantie van modellen \autocite{IBM2022}. Bij de ontwikkeling en finetuning van een NLP-toepassing met AI verkiezen software-ontwikkelaars \textit{black-box} modellen. Hoewel het verschil in nauwkeurigheid minimaal is, wordt de afweging gemaakt op basis van de transparantie van het model. Na een transformatie wordt niet aangegeven waarom specifieke transformaties zijn uitgevoerd, zoals het vervangen van een woord door een eenvoudiger synoniem. \textit{White-box} taalmodellen zijn schaars \autocite{Punardeep2020}.

\medspace 

Homoniemen kunnen \textit{sequence labeling} bemoeilijken, zoals beschreven in onderzoek door \textcite{Roldos2020}. Een voorbeeld hiervan is het woord 'bank', dat voor de machine niet duidelijk aangeeft of het gaat om de financiële instelling of het meubelstuk. Methoden zoals \textit{Word Sense Disambiguation} (WSD), \textit{PoS-tagging} en \textit{contextual embeddings} kunnen de betekenis van woorden bepalen op basis van de context \autocite{Eisenstein2019, Liu2020}. Het verbeteren van NLP-systemen met synoniemen en antoniemen kan worden bereikt door het gebruik van \textit{candidate generation} en synoniemherkenning, terwijl meertalige transformers zoals BERT een oplossing bieden voor de beperkte toepassingen in andere talen dan het Engels \autocite{Dandekar2016, Roldos2020}.

\medspace

Verschillende studies tonen aan dat MTS en ATS gelijke kansen kunnen bieden aan iedereen. Bij de ontwikkeling van gepersonaliseerde ATS-toepassingen moeten ontwikkelaars de ethische overwegingen en de behoeften van de eindgebruiker in overweging nemen, zoals beschreven in onderzoeken van \textcite{Niemeijer2010, Xu2015, Gooding2022}. Om de eindgebruiker meer controle te geven, moet deze eindgebruiker kunnen kiezen welke delen van de tekst het systeem moet vereenvoudigen.

\medspace

Hoewel iedereen kan converseren met een chatbot, vereist het verkrijgen van gepaste en verwachte antwoorden een doordachte input. Onnauwkeurige prompts of een gebrek aan trainingsdata kunnen leiden tot onjuiste output. Door het gebruik van conditionele expressies of het finetunen van hyperparameters kan echter de betrouwbaarheid van de antwoorden worden vergroot \autocite{Miszczak2023, Jiang2023}.

\medspace

Het beoordelen van vereenvoudigde teksten vereist de nodige opvolging van ontwikkelaars in vergelijking met andere ML- of NLP-taken. Evaluatiemetrieken zoals ROUGE en BLEU zijn beperkt omdat ze geen rekening houden met de semantiek tussen een referentietekst en een vereenvoudigde of samengevatte tekst. Om dit probleem op te lossen, beveelt \textcite{Fabbri2020} aan dat ontwikkelaars menselijke evaluatie inschakelen om de vereenvoudigde tekst van een taalmodel te beoordelen. De onderzoekers dringen aan op verdere studie naar nieuwe standaarden en beste praktijken voor betrouwbare menselijke beoordeling. Bovendien moeten de doelgroepen voor wie de tekst wordt vereenvoudigd, nauw betrokken worden bij het proces \autocite{Iskender2021}. Tabel \ref{table:summary-hurdles} somt de aangehaalde struikelblokken bij de ontwikkeling van NLP-toepassingen op.

\begin{center}
	\begin{table}[H]
	\begin{tabular}{ | m{4cm} | m{10cm} | }
		\hline
		\textbf{Probleem} & \textbf{Oplossing} \\
		\hline
		Dure ontwikkeling en onderhoud van taalmodellen & Voorkeur voor black-box modellen bij ontwikkeling en finetuning. API's kunnen als alternatief dienen op zelf-gehoste taalmodellen. \\
		\hline
		Homoniemen kunnen \textit{sequence labeling} bemoeilijken & Word Sense Disambiguation, PoS-tagging en contextual embeddings. \\
		\hline
		Paternalisme & Ontwikkeling van gepersonaliseerde tekstvereenvoudigingstoepassingen moeten de eindgebruiker meer controle geven, zoals het kiezen welke delen van de tekst vereenvoudigd moeten worden, het gebruik van synoniemen of het markeren van zinnen die moeilijk te begrijpen zijn. \\
		\hline
		Onnauwkeurige prompts & Gebruik van conditionele expressies bij prompts of one-shot summary uitvoeren. \\
		\hline
		Onnauwkeurige evaluatie van tekstvereenvoudiging & Menselijke evaluatie toepassen of gebruik maken van ROUGE-L metrieken die wel de semantiek in achting nemen. \\
		\hline
	\end{tabular}
	\caption{Samenvattend schema met vaak voorkomende struikelblokken bij NLP-toepassingen.}
	\label{table:summary-hurdles}
	\end{table}
\end{center}

\section{Conclusie}

Tot slot beantwoordt de literatuurstudie de onderzoeksdeelvragen. Zo is begrijpend lezen voor scholieren met dyslexie in de derde graad van het middelbaar niet enkel moeizaam, maar gaat verder dan dat. Deze scholieren kunnen moeite ondervinden bij het ontcijferen en automatiseren van woordherkenning. MTS en aangepaste opmaakopties bieden bewezen voordelen voor scholieren met dyslexie. Het gebruik van vakjargon, ingewikkelde woordenschat en moeizame syntax sluiten een algemeen publiek uit en maken het enkel mogelijk voor wetenschappelijk geletterden om deze artikelen te lezen. Zo kunnen MTS-technieken, besproken in \ref{table:benefits-mts}, scholieren met dyslexie helpen bij het begrijpend lezen van wetenschappelijke artikelen. Tabel \ref{table:scientific-paper-struggles} somt alle complexe leesfactoren op, alsook hoe lezers deze moeilijke materie kunnen aanpakken. Deze twee tabellen moeten dienen als aftoetscriteria bij de ontwikkeling van een prototype voor gepersonaliseerde ATS. 

\medspace

Er zijn taalmodellen beschikbaar in de vorm van API's of open-source software die deze transformaties kunnen uitvoeren. De overheid leent voornamelijk leessoftware uit, maar LLM's bieden mogelijkheden aan voor gepersonaliseerde ATS. Zo kunnen de achterliggende taalmodellen van ChatGPT en Bing Chat specifieke vragen verwerken. Het onderzoek moet verschillende prompts kunnen vergelijken bij het uittesten van dit taalmodel. Verschillende LS, SS en SA-technieken moeten in deze vergelijkende studie aan bod komen. 

\medspace

Gerichte menselijke vergelijkingen en leesbaarheidsformules dienen als evaluatie voor de vereenvoudigde teksten door ATS. De python-bibliotheek \textit{readability} biedt de leesgraadscores, zoals weergegeven in \ref{table:readability-scores}, aan voor ontwikkelaars. Zo kan dit onderzoek de uitvoer van verschillende teksten vergelijken met een objectieve maatstaf.

\medspace

Tot slot geeft tabel \ref{table:summary-hurdles} een overzicht van struikelblokken waarmee ontwikkelaars rekening moeten houden. In het kader van ATS voor wetenschappelijke artikelen, moet het taalmodel gebruikmaken of voldoende getraind zijn op data van wetenschappelijke artikelen en vereenvoudigde versies van diezelfde wetenschappelijke artikelen. Als het onderzoek gebruik maakt van een promptgebaseerd model, dan moeten deze prompts op maat gemaakt zijn voor scholieren met dyslexie in de derde graad van het middelbaar onderwijs. Tot slot moet dergelijk prototype ook de gebruiker de vrijheid krijgen om teksten te markeren waarvan deze persoon een vereenvoudigde versie wilt.

%%=============================================================================
%% Methodologie
%%=============================================================================

\chapter{\IfLanguageName{dutch}{Methodologie}{Methodology}}%
\label{ch:methodologie}

% Door de vergaarde kennis uit de literatuurstudie in hoofdstuk \ref{ch:stand-van-zaken} over de mogelijke noden van scholieren met dyslexie, de complexiteit van wetenschappelijke artikelen, de technieken voor MTS en ATS, en de bijhorende valkuilen bij taalverwerking met AI, kunnen onderzoeksmethoden worden toegepast om een antwoord te vinden op de onderzoeksvraag. 

De vergaarde kennis uit de literatuurstudie dient als vertrekpunt voor het verdere onderzoek. Zo kan het de onderzoeksvraag beantwoorden met drie onderzoeksstappen. Eerst staat het onderzoek stil bij de nodige functionaliteiten om gepersonaliseerde ATS te kunnen verwezenlijken. Vervolgens achterhaalt het een geschikt taalmodel voor gepersonaliseerde ATS. Tot slot ontwikkelt het onderzoek Pentimentor, ofwel het prototype voor ATS van wetenschappelijke artikelen. Dit moet een eenduidige manier voor scholieren met dyslexie in de derde graad van het middelbaar onderwijs aanreiken om wetenschappelijke artikelen te vereenvoudigen. Zo doelt dit onderzoek om de haalbaarheid voor een toepassing voor gepersonaliseerde ATS te achterhalen met Pentimentor. 

\section{Requirementsanalyse}
\label{sec:requirementsanalyse}

Om het ontwikkelingsproces van Pentimentor gericht te sturen, moet het onderzoek bestaande en haalbare MTS en ATS-technologieën in bestaande tools nagaan. Zo gebeurt het verkennen en experimenteren op ATS-technieken bij beschikbare tools door een kwalitatief onderzoek in de vorm van een requirementsanalyse. Het resultaat van deze onderzoeksfase is een moscow-schema dat de benodigde en haalbare functionaliteiten voor een ATS-toepassing definieert. Met dit kan het onderzoek een vergelijkbare toepassing ontwikkelen voor gepersonaliseerde ATS van wetenschappelijke artikelen met de kwaliteiten van gepersonaliseerde MTS. Daarnaast achterhaalt deze fase de ontbrekende MTS-functionaliteiten die tabel \ref{table:benefits-mts} in de literatuurstudie uitwees. De geteste toepassingen, opgesomd in tabel \ref{table:shortlist-tools}, beschikken over (gepersonaliseerde) ATS-technieken. Deze lijst omvat erkende toepassingen van de overheid en toepassingen die leerkrachten of scholieren kunnen gebruiken om teksten te vereenvoudigen. Met deze onderzoeksmethode kan het onderzoek een antwoord geven op de volgende twee deelvragen van het onderzoek.

\begin{itemize}
	\item Welke functies ontbreken AI-toepassingen om geautomatiseerde tekstvereenvoudiging mogelijk te maken voor scholieren met dyslexie in de derde graad middelbaar onderwijs?
	\item Welke manuele methoden voor tekstvereenvoudiging komen niet in deze tools voor?
\end{itemize}

Figuur \ref{img:flowchart-requirementsanalyse} toont de flowchart om de requirementsanalyse te kunnen uitwerken.

\begin{figure}[H]
	\includegraphics[width=\linewidth]{img/flowchart-requirementsanalyse.jpg}
	\caption{Het benodigde stappenplan bij de requirementanalyse.}
	\label{img:flowchart-requirementsanalyse}
\end{figure}

% TOOLSELECTIE 
Allereerst start het onderzoek met een toolselectie. Zoals aangewezen in sectie \ref{sec:beschikbare-tools-en-taalmodellen}, leent de overheid vijf softwarepakketten uit aan middelbare scholen. Echter neemt de requirementsanalyse drie van deze vijf in de analyse op, want hun functionaliteiten zijn passend voor deze doelgroep. De overige twee zijn minder aanwezig in het onderwijs. Daarnaast toont een zoekopdracht aan dat deze tools geen LS toepassen. Buiten deze erkende softwarepakketten, kunnen online beschikbare tools ook scholieren met dyslexie ondersteunen bij het begrijpend lezen van wetenschappelijke artikelen met ATS, zoals bewezen in \textcite{Bingel2018}. Daarom betrekt de requirementsanalyse enkel tools met onderschreven ATS-functionaliteiten en laat daarmee pure samenvattingstools erbuiten. Tabel \ref{table:shortlist-tools} toont een overzicht van de te experimenteren tools.

\begin{center}
	\begin{table}[H]
		\begin{tabular}{ | m{6cm} | m{6cm} | } 
			\hline
			\textbf{Erkende software} & \textbf{Online beschikbare tools} \\
			\hline
			Sprintplus (E1) & Simplish (O1) \\
			Kurzweil3000 (E2) & SciSpace (O2) \\ 
			AlineaSuite (E3) & Rewordify (O3) \\
			& ChatGPT (O4) \\
			& Bing Chat (O5) \\
			\hline
		\end{tabular}
		\caption{Shortlist van uit te testen tools en toepassingen voor tekstvereenvoudiging.}
		\label{table:shortlist-tools}	
	\end{table}
\end{center}

% Inclusiecritera
Vervolgens bouwt het onderzoek een lijst op van de toetsingscriteria. Zo dienen de MTS-technieken uit tabel \ref{table:scientific-paper-struggles} en tabel \ref{table:manual-simplification} als bouwstenen voor het opstellen van de toetsingscriteria. Tabel \ref{table:criteria-requirementsanalysis} geeft een opsomming van de MTS-technieken waaraan tools moeten voldoen. Daarnaast dient dit schema om een inschatting van de functionaliteiten van deze tools te maken.

\begin{center}
	\begin{table}[H]
		\begin{tabular}{ | m{4cm} | m{11cm} | } 
			\hline
			\textbf{MTS-techniek} & \textbf{Functionaliteit} \\
			\hline
			LS & Gepersonaliseerde LS, ofwel woordenschat dat niet te hooggegrepen is. Gekende woordenschat mag blijven. \\
			& Woorden met minder lettergrepen gebruiken \\
			& Extra uitleg schrijven bij zinnen \\
			& Paragrafen herschrijven zodat ze eerst uitleg geven op een high-level niveau, vervolgens lagen van complexiteit toevoegen om de lezer te begeleiden \\
			& Woordenlijst aanmaken \\
			& Idiomen vervangen door eenvoudigere synoniemen \\
			\hline
			SS & Zinnen inkorten \\
			& Verwijswoorden aanpassen \\
			& Voorzetseluitdrukkingen aanpassen \\
			& Samengestelde werkwoorden aanpassen \\
			& Actieve stem toepassen \\
			& Enkel regelmatige werkwoorden gebruiken \\
			\hline
			SA & Achtergrondkleur aanpassen \\
			& Woord- en karakterspatiëring \\
			& Consistente lay-out \\
			& Duidelijk zichtbare koppenstructuur \\
			& Huidige positie benadrukken \\
			& Waarschuwingen geven omtrent formulieren en sessies \\
			& Inhoud visueel groeperen \\
			& Tekst herschrijven als tabel \\
			& Tekst herschrijven als opsomming \\
			\hline
		\end{tabular}
		\caption{Richtlijnen waarop het onderzoek de toepassingen aftoetst in de requirementsanalyse.}
		\label{table:criteria-requirementsanalysis}	
	\end{table}
\end{center}

Als realistisch testmateriaal, maken de experimenten gebruik van twee gepubliceerde wetenschappelijke artikelen. Zo kunnen deze artikelen relevant zijn voor leerkrachten om aan scholieren in de derde graad van het middelbaar onderwijs te geven als leesvoer. Beide artikelen volgen de kenmerken van een wetenschappelijk artikel, zoals beschreven in tabel \ref{table:scientific-paper-struggles}. Daarnaast gebruiken ze vakjargon en wetenschappelijke concepten in een compact formaat. Tabel \ref{table:referentieteksten-bronvermelding} geeft een overzicht van de twee artikelen en een bijhorende bronvermelding.

\begin{center}
	\begin{table}[H]
		\begin{tabular}{ | m{10cm} | m{5cm} | } 
			\hline
			\textbf{Titel} & \textbf{Bronvermelding} \\
			\hline
			De controle op het gebruik van algoritmische surveillance- onder druk? Een exploratie door de lens van de relationele ethiek & \autocite{VanBrakel2022} \\
			\hline
			Nederland versus België: verschillen in economische dynamiek en beleid. & \autocite{Sleuwaegen2022} \\
			\hline
		\end{tabular}
		\caption{Bronvermeldingen voor de twee wetenschappelijke artikelen.}
		\label{table:referentieteksten-bronvermelding}
	\end{table}
\end{center}

% 4. 
Om een overzicht te hebben van de functionaliteiten volgens prioriteit, bouwt het onderzoek een moscow-schema vanuit de opgestelde richtlijnen. Zo komen belangrijke functionaliteiten, die nodig zijn om gepersonaliseerde tekstvereenvoudiging met ATS mogelijk te maken, in de categorie \textit{must-haves} terecht. Alle vereiste functionaliteiten om (gepersonaliseerde) ATS mogelijk te maken, moet als \textit{must-have} in het moscow-schema voorkomen. Irrelevante functionaliteiten binnen de scope van een prototype of niet toepasselijk voor de doelgroep plaatst het onderzoek als \textit{wont-have}.

\begin{center}
	\begin{table}[H]
		\begin{tabular}{ | m{2cm} | m{13cm} | } 
			\hline
			\textbf{MoSCoW-principe} & \textbf{Functionaliteit} \\
			\hline
			Must-have & Gepersonaliseerde LS, ofwel woordenschat dat niet te hooggegrepen is. Gekende woordenschat mag blijven. \\
			& Woorden met minder lettergrepen gebruiken. \\
			& Woordenlijst aanmaken na handmatige CWI. \\
			& Wetenschappelijke artikelen in PDF-vorm opladen. \\
			& SA-technieken toepassen op de oorspronkelijke tekst. \\
			& Personaliseerbare opmaakopties, waaronder lettertype -en grootte aanpassen, tekstformaat aanpassen, achtergrondkleur aanpassen. \\
			& Duidelijk zichtbare koppenstructuur. \\
			& Tekst herschrijven als opsomming. \\
			\hline
			Should-have & Tekstanalyse \\
			& Extra (in-line) uitleg schrijven bij moeilijke woordenschat. \\
			& Personaliseerbare PDF- of Word-document lay-out. \\
			& Uitvoer als pdf of docx-bestand teruggeven. Het onderzoek gebruikt geen PrintToPDF. \\
			& Wetenschappelijke artikelen in PDF-vorm opladen met OCR. \\
			& Tekstanalyse voor en na de vereenvoudiging aanbieden. \\
			\hline
			Could-have & Huidige positie benadrukken.\\
			& Woordenschat genereren na automatische CWI. \\
			& Waarschuwingen geven omtrent formulieren en sessies. \\
			& Enkel regelmatige werkwoorden gebruiken. \\
			& Extraherende samenvatting \\
			& Abstraherende samenvatting \\
			& Tekst herschrijven in tabelvorm \\
			\hline
			Wont-have & Mobiele versie of \textit{responsive design}. \\
			& Audio-uitvoer \\
			& Integratie met externe toepassingen. Het onderzoek gebruikt Grammarly als voorbeeld voor deze test. \\
			\hline
		\end{tabular}
		\caption{Het moscow-schema voor de requirementsanalyse.}
		\label{img:moscow-table}
	\end{table}
\end{center}

Vervolgens komen experimenten op functionaliteiten voor gepersonaliseerde ATS op wetenschappelijke artikelen aan bod. Allereerst moeten eindgebruikers wetenschappelijke artikelen kunnen opladen. Indien het onderzoek een wetenschappelijk artikel niet als pdf kan opladen, dan extraheert het onderzoek de tekstinhoud van het wetenschappelijk artikel. Vervolgens krijgt de toepassing deze tekst als invoer. Zo krijgen de chatbots eerst de prompt, gevolgd door een stuk van het wetenschappelijk artikel. Met zes verschillende prompts kan het onderzoek de LS, SS en SA-functionaliteiten van een promptgebaseerde toepassing achterhalen. Tabel \ref{table:tested-prompts-requirementsanalysis} vermeldt de toegepaste prompts. Toepassingen krijgen eerst een link van het wetenschappelijk artikel. Als de toepassing hier niet over beschikt, dan krijgt de chatbot de tekstinhoud van het wetenschappelijk artikel in \textit{plain-text} mee. 

\begin{center}
	\begin{table}[H]
		\begin{tabular}{ | m{2cm} | m{14cm} | } 
			\hline
			\textbf{Naam} & \textbf{Prompt} \\
			\hline
			P1 & Vereenvoudig deze tekst. \\
			\hline
			P2 & Vereenvoudig deze tekst voor studenten (16-18 jaar) door moeilijke woorden te vervangen, vakjargon te schrappen, woorden langer dan 18 letters te vervangen, acroniemen voluit te schrijven, een woord slechts eenmaal door een synoniem te vervangen, korte uitleg te geven wanneer dat nodig is, en percentages te vervangen. \\
			\hline
			P3 & Vereenvoudig een tekst door deze op te delen in kortere zinnen van maximaal tien woorden. Verander voornaamwoorden als 'zij', 'hun' of 'hij' in namen. Vervang complexe zinsconstructies en voorzetselzinnen door eenvoudiger alternatieven, maar laat ze ongewijzigd als er geen eenvoudiger optie beschikbaar is. \\
			\hline
			P4 & Schrijf de tekst als opsomming. \\
			\hline
			P5 & Schrijf de tekst in tabelformaat. \\
			\hline
			P6 & Genereer op basis van deze tekst een woorden- en synoniemenlijst. \\
			\hline
		\end{tabular}
		\caption{De toegepaste GPT-3-prompts in de requirementsanalyse.}
		\label{table:tested-prompts-requirementsanalysis}
	\end{table}
\end{center}

Daarna voert het onderzoek experimenten rond gepersonaliseerde opmaakopties uit. Kurzweil, SprintPlus en AlineaSuite bieden opmaakopties aan in het instellingenscherm. Zo kunnen eindgebruikers het lettertype, -kleur -en grootte en de achtergrondkleur aanpassen naar keuze. Als een toepassing niet over opmaakopties beschikt, stopt het experiment rond opmaakopties voor die toepassing. Tot slot test het onderzoek de mogelijkheden om het formaat van teksten aan te passen met structurele aanpassingen. Zo vragen P4 en P5 specifiek naar een structurele aanpassing, terwijl P1, P2 en P3 minstens een doorlopende tekst als resultaat verwachten. Andere beschikbare tools missen checkboxen of keuzelijsten om deze keuze aan te reiken, waardoor het testen van deze functionaliteit niet mogelijk is.

\section{Vergelijking van taalmodellen}
\label{sec:vergelijkende-studie}

Om wetenschappelijke artikelen met gepersonaliseerde ATS te vereenvoudigen, moet dergelijk toepassing gebruikmaken van een geschikt taalmodel. Zo moet Pentimentor vereenvoudigde versies van wetenschappelijke artikelen kunnen geven, specifiek volgens de noden van scholieren met dyslexie in de derde graad van het middelbaar onderwijs. Om de uitvoer van het prototype nauwkeurig af te stemmen, moet het onderzoek een antwoord op de volgende vraag geven.

\begin{itemize}
	\item Welk taalmodel kunnen ontwikkelaars inzetten voor tekstvereenvoudiging met ATS van wetenschappelijke artikelen voor scholieren met dyslexie in de derde graad van het middelbaar onderwijs, met dezelfde of gelijkaardige kwaliteiten als gepersonaliseerde tekstvereenvoudiging met MTS?
\end{itemize}

Zoals in de literatuurstudie aangegeven, beschikken ontwikkelaars over onvoldoende gespecialiseerde taalmodellen om wetenschappelijke artikelen te vereenvoudigen. Daarom vergelijkt deze onderzoeksfase alle taalmodellen uit tabel \ref{table:vergelijkende-studie-taalmodellen}. 

\begin{center}
	\begin{table}[H]
		\begin{tabular}{ | m{4cm} | m{11cm} | } 
			\hline
			\textbf{Verwijzing} & \textbf{Taalmodel} \\
			\hline
			T1 & Haining Scientific Abstract Simplification \\
			\hline
			T2 & BART-based Scientific Lay Summarizer \\
			\hline
			T3 & Keep It Simple\\
			\hline
			T4 & GPT-3 \\
			\hline
		\end{tabular}
		\caption{Gebruikte taalmodellen in de vergelijkende studie}
		\label{table:vergelijkende-studie-taalmodellen}
	\end{table}
\end{center}

Deze onderzoeksfase bestaat uit vijf deelfasen, weergegeven op figuur \ref{img:flowchart-vergelijkende-studie-metrics}. Het staat vooral stil bij de vergelijking van leesmetrieken van de oorspronkelijke wetenschappelijke artikelen, MTS-referentieteksten en ATS-teksten. Met een \textit{mixed-methods} onderzoek kan deze onderzoeksfase taalmodellen beoordelen op machinaal en menselijk niveau. Alle scripts kan u terugvinden op de GitHub-repository\footnote{https://github.com/dylancluyse/bachelorproef-nlp-tekstvereenvoudiging/tree/main/scripts}. De gebruikte wetenschappelijke artikelen, afgekort OG, zijn identiek aan de wetenschappelijke artikelen in tabel \ref{table:referentieteksten-bronvermelding}. Om realistisch referentiemateriaal te verkrijgen, schrijven twee leerkrachten (MTSL) en twee leerlingen zonder dyslexie (MTSL2) zelf een vereenvoudiging van de twee wetenschappelijke artikelen met MTS. Deze vier personen baseren zich op vooraf meegekregen richtlijnen, toegelicht in bijlage \ref{ch:referentietekst}. 

\begin{figure}
	\includegraphics[width=\linewidth]{img/flowchart-vergelijkende-studie.jpg}
	\caption{Het gevolgde stappenplan voor de vergelijking van taalmodellen.}
	\label{img:flowchart-vergelijkende-studie-metrics}
\end{figure}

% SCRIPT 1
Allereerst haalt het script de inhoud van de map met wetenschappelijke artikelen op om deze vervolgens in een tekstbestand te plaatsen, zoals weergegeven in codeblok \ref{code:verg-studie-phase-1}. Hier schrijft het script tekstinhoud van het wetenschappelijk artikel over naar een nieuw tekstbestand.

\begin{lstlisting}[language=Python, caption={Script voor de eerste fase van de vergelijkende studie.}, label={code:verg-studie-phase-1}]
def add_newline_after_dot(input_file, output_file):
	with open(input_file, 'r', encoding='utf-8') as file:
		text = file.read()
		text = re.sub(r'\d', '', text)
		modified_text = text.replace('.', '.\n')
		with open(output_file, 'w', encoding='utf-8') as file:
			file.write(modified_text)
		
		folder_path = 'scripts\pdf'
		original_scientific_papers = [f for f in os.listdir(folder_path)]
		
		for paper in original_scientific_papers:
			input_file =  folder_path + '/' + paper
			output_file = folder_path + '/' + 'RE_' + paper
			add_newline_after_dot(input_file, output_file)
\end{lstlisting} 

Vervolgens vertaalt het tweede script alle zinnen naar Engels. Eerst doorloopt het alle wetenschappelijke artikelen. Daarna vertaalt het tekstinhoud met de \textit{deep translator} python-bibliotheek. Als resultaat ontstaat er een csv-bestand met twee kolommen: alle Nederlandstalige en alle Engelstalige zinnen van één wetenschappelijk artikel. Als separator gebruikt het csv-bestand een \textit{pipe}-symbool en zo houdt het rekening met punten en komma's in zinnen.

\begin{center}
	\begin{lstlisting}[language=Python, caption={Script voor de tweede fase van de vergelijkende studie.}, label={code:verg-studie-phase-2}]
def translate_dutch_to_english(dutch_text_file):
	with open(dutch_text_file, 'r', encoding='utf-8') as file:
		dutch_sentences = file.readlines()
		dutch_sentences = [sentence.strip() for sentence in dutch_sentences]
				
		english_sentences = []
		for sentence in dutch_sentences:
			translated = GoogleTranslator(source='nl', target='en').translate(sentence)
			english_sentences.append(translated)
			df = pd.DataFrame({'Dutch': dutch_sentences, 'English': english_sentences})
			df.to_csv(str(dutch_text_file).split('.')[0] + '.csv', index=False)
				
				
		folder_path = 'scripts/pdf/'
		original_scientific_papers = [f for f in os.listdir(folder_path)]
				
		for paper in original_scientific_papers:
			if paper.startswith('RE_') and paper.endswith('.txt'):
				print(f'STARTING {paper}')
				dutch_text_file = folder_path + paper
				translate_dutch_to_english(dutch_text_file)
		\end{lstlisting}
\end{center}

In een derde fase moet het script API-calls voor iedere zin of paragraaf versturen naar ieder taalmodel. Eerst slaat het script een dictionary op van taalmodellen om de inhoud van de wetenschappelijke artikelen te vereenvoudigen, weergegeven in listing \ref{code:verg-studie-phase-3}. Daarna vindt een tokenisatiefase plaats, waarvoor het Spacy \textit{sentence tokenisation} en verwante \textit{embedding models} gebruikt. Deze modellen staan in tabel \ref{table:wordembeddings-spacy}. Na de tokenisatiefase stuurt het script API-calls naar de hosts van de taalmodellen. De request naar de HF API bestaat uit de parameters weergegeven in tabel \ref{table:huggingface-requests-parameters}. Alle HF-taalmodellen vereisen een laadfase. Daarom bevat de \textit{API-call} een extra parameter, namelijk \textit{wait\_for\_model}. Verder past dit script geen extra parameters van de taalmodellen aan. 

\begin{center}
	\begin{table}[H]
		\begin{tabular}{ | m{6cm} | m{8cm} | } 
			\hline
			\textbf{Naam parameter} & \textbf{Waarde} \\
			\hline
			Inputs & De oorspronkelijke zin. Enkel bij T1 komt 'simplify:' voor deze zin. \\
			\hline
			Max length & De lengte van de oorspronkelijke zin + 10 tokens. \\
			\hline
			Wait for model & Altijd ingesteld op \textit{True}. \\
			\hline
		\end{tabular}
		\caption{Meegegeven parameters bij HF-requests}
		\label{table:huggingface-requests-parameters}
	\end{table}
\end{center}

Zoals aangehaald door \textcite{Gooding2022} kunnen promptgebaseerde testen verschillende resultaten krijgen, afhankelijk van de gegeven input. Daarom gebruikt het script drie verschillende prompts, gebaseerd op de MTS-technieken beschreven in tabel \ref{table:manual-simplification}. Tabel \ref{table:tested-prompts} visualiseert de gebruikte prompts voor de testen met het GPT-3 model. 

\begin{center}
	\begin{table}[H]
		\begin{tabular}{ | m{2cm} | m{13cm} | } 
			\hline
			\textbf{Naam} & \textbf{Prompt} \\
			\hline
			P1 & Vereenvoudig deze tekst \\
			\hline
			P2 & Vereenvoudig deze tekst voor studenten (16-18 jaar) door moeilijke woorden te vervangen, vakjargon te schrappen, woorden langer dan 18 letters te vervangen, acroniemen voluit te schrijven, een woord slechts eenmaal door een synoniem te vervangen, korte uitleg te geven wanneer dat nodig is, en percentages te vervangen. \\
			\hline
			P3 & Vereenvoudig een tekst door deze op te delen in kortere zinnen van maximaal tien woorden. Verander voornaamwoorden als 'zij', 'hun' of 'hij' in namen. Vervang complexe zinsconstructies en voorzetselzinnen door eenvoudiger alternatieven, maar laat ze ongewijzigd als er geen eenvoudiger optie beschikbaar is. \\
			\hline
		\end{tabular}
		\caption{De GPT-3-prompts die in de vergelijkende studie aan bod komen.}
		\label{table:tested-prompts}
	\end{table}
\end{center}

Zowel T1 als T4 gebruiken een \textit{nul-temperature} en een \textit{top-p} waarde van 90\% om vertrouwde antwoorden te krijgen. Daarnaast dient de top-p om een hoge woordfrequentie te verkrijgen, zoals aangegeven in tabel \ref{table:gpt-3-parameters}. Bij T1 zijn deze twee parameters ingebakken in de functie. Nadien verwerken de taalmodellen iedere zin uit de tekst. Tot slot beantwoordt de HF of GPT API in JSON-formaat, bevattende de vereenvoudigde versie van de opgegeven zin. T1, T2 en T3 vereenvoudigen de Engelstalige zinnen, in tegenstelling tot T4 en verwante prompts die de Nederlandstalige zinnen vereenvoudigen. Om een \textit{request failure} door een te lange input te voorkomen, breekt het script de volledige input op per 1000 tokens.

\begin{center}
	\begin{table}[H]
		\begin{tabular}{ | m{7cm} | m{7cm} | } 
			\hline
			\textbf{Taal} & \textbf{Embeddingsmodel} \\
			\hline
			Nederlands & NL Core News Medium \\ 
			\hline
			Engels & EN Core Web Medium \\
			\hline
		\end{tabular}
		\caption{Gebruikte SpaCy word-embeddings}
		\label{table:wordembeddings-spacy}
	\end{table}
\end{center}

\begin{center}
	\begin{lstlisting}[language=Python, caption={Script voor de derde fase van de vergelijkende studie}, label={code:verg-studie-phase-3}]		
folder_path = 'scripts\pdf'
dutch_spacy_model = "nl_core_news_md"
english_spacy_model = "en_core_web_sm"
	
dict = {
	'nl':'nl_core_news_md',
	'en':'en_core_web_sm'
}
		
total_df = None
gt = Translator()
		
huggingfacemodels = {
	'T1':"https://api-inference.huggingface.co/models/haining/scientific_abstract_simplification",
	'T2': "https://api-inference.huggingface.co/models/sambydlo/bart-large-scientific-lay-summarisation",
	'T3': "https://api-inference.huggingface.co/models/philippelaban/keep_it_simple"
}
		
max_length = 2000
COMPLETIONS_MODEL = "text-davinci-003"
EMBEDDING_MODEL = "text-embedding-ada-002"
	
languages = {
	'nl':'nl_core_news_md',
	'en':'en_core_web_md'
}
		
class HuggingFaceModels:
	def __init__(self, key=None):
		global huggingface_api_key
		try:
			huggingface_api_key = key
		except:
			huggingface_api_key = 'not_submitted'
				
	def query(self, payload, API_URL):
		headers = {"Authorization": f"Bearer {huggingface_api_key}"}
		response = requests.post(API_URL, headers=headers, json=payload)
		return response.json()
			
	def scientific_simplify(self, text, lm_key):
		try:
			API_URL = huggingfacemodels.get(lm_key)
			translated = GoogleTranslator(source='auto', target='en').translate(str(text))
			
			if lm_key == 'T1':
				result = self.query({"inputs": str('simplify: ' + str(translated)),"parameters": {"max_length": len(sentence)+10},"options":{"wait_for_model":True}}, API_URL)
			else:
				result  = self.query({"inputs": str(translated),"parameters": {"max_length": len(sentence)+10},"options":{"wait_for_model":True}}, API_URL)
					
					
			if 'generated_text' in result[0]:
				translated = GoogleTranslator(source='auto', target='nl').translate(str(result[0]['generated_text']))
				return translated
			elif 'summary_text' in result[0]:
				translated = GoogleTranslator(source='auto', target='nl').translate(str(result[0]['summary_text']))
				return translated
			else:
				return None
		except:
			return text
		
	def get_sentence_length(sentence):
			txt_language = detect(sentence)
			dic_language = languages.get(txt_language)
			nlp = spacy.load(dic_language)
			doc = nlp(sentence)
			return len()	
			
	def tokenize_text(text):
		txt_language = detect(text)
		dic_language = languages.get(txt_language)
		nlp = spacy.load(dic_language)
		doc = nlp(text)
		return doc.sents
					
	def process_file(file_path):
		with open(folder_path + '/' + file_path, "r", encoding='utf8') as file:
			text = file.read()
			tokens = tokenize_text(text)
			return tokens
				
		hf = HuggingFaceModels(key=os.getenv('huggingface_key'))
		original_scientific_papers = [f for f in os.listdir(folder_path)]
		
		for paper in original_scientific_papers[3:]:
			sentence_tokens = process_file(paper) 
			for sentence in sentence_tokens:
				for model in huggingfacemodels.keys():
					filename = "SIMPLIFIED_"+model+'_'+paper
					with open(filename, 'a', encoding='utf-8') as f:
						output = hf.scientific_simplify(str(sentence), model)
						f.write(str(output)) 	
	\end{lstlisting}
\end{center}

Vervolgens berekent het script leesmetrieken met de \textit{readability} python-bibliotheek. Leesmetrieken dienen als objectieve maatstaf bij deze vergelijkende studie, zoals aangegeven door \textcite{Nenkova2004}. 

\begin{itemize}
	\item De vergelijkende studie neemt de FRE en FOG scores op, want deze scores kunnen de moeilijkheidsgraad van een zin of tekst machinaal meten.
	\item Het aantal complexe en lange woorden kan wijzen op de gebruikte \textit{substitution generation} van het taalmodel. De DCI bepaalt de complexiteit van een woord in deze library. Ten slotte tellen alle woorden met meer dan vier lettergrepen mee als een lang woord volgens de \textit{readability}-library.
	\item Het aantal hulpwerkwoorden en vervoegingen van 'zijn' kan aanduiden op mogelijke passieve stem, wat het onderzoek van \textcite{Ruelas2020} als 'hinderende' zinsyntax vernoemt.
\end{itemize}

Dit script resulteert in een \textit{Pandas-dataframe} met alle leesbaarheidsmetrieken uit de \textit{readability}-library. Uiteindelijk slaat het script de \textit{Pandas-dataframe} op als CSV-bestand. Listing \ref{code:verg-studie-phase-4} omvat de code voor fase 4. Hierin berekent het script leesmetrieken voor iedere zin. Allereerst laadt het script twee embeddingsmodellen in, weergegeven in tabel \ref{table:wordembeddings-spacy}. Daarna bouwt het lege dataframes op om de meetresultaten te kunnen opslaan. Vervolgens itereert het python-script door alle tekstbestanden om deze tekst in te lezen. Voor elke zin probeert het script om leesmetrieken te verkrijgen met de \textit{readability.getmeasures} functie. Hierbij specifieert het script welke taal de readability-library moet gebruiken. Als het script alle meetwaarden succesvol kan verkrijgen, dan maakt het script een nieuwe rij met machinaal berekende leesmetrieken. Tot slot voegt het alle rijen toe aan de DataFrame. Zinnen waarbij het script geen leesmetrieken kan berekenen, dan verwerpt het die zin voor dat artikel.

\begin{center}
	\begin{lstlisting}[language=Python, caption={Script voor de vierde fase van de vergelijkende studie}, label={code:verg-studie-phase-4}]	
		simplified_folder = 'scripts/vereenvoudigde_artikelen'
		original_folder = 'scripts/pdf'
		
		scientific_papers = [original_folder + "/" + f for f in os.listdir(original_folder)] + [simplified_folder + "/" + f for f in os.listdir(simplified_folder)]
		
		languages = {
			'nl':'nl_core_news_md',
			'en':'en_core_web_md'
		}
		
		df = pd.DataFrame()
		
		for paper in scientific_papers:
		with open(paper, 'r', encoding='utf-8') as file:
			text = file.read()
			nlp = spacy.load(languages.get('nl'))
			doc = nlp(text)
		
		for sent in doc.sents:
			try:
				metrics = readability.getmeasures(sent.text, lang='nl')
				row = {
					'Paper': paper.split('/')[2].split('.')[0],
					'Sentence': sent.text,
					'FRE': metrics['readability grades']['FleschReadingEase'],
					'FOG': metrics['readability grades']['GunningFogIndex'],
				}
		
		for key, value in metrics['sentence info'].items():
			row[key] = value
		
		for key, value in metrics['word usage'].items():
			row[key] = value
		
		for key, value in metrics['sentence beginnings'].items():
			row[key] = value
			
		df = df.append(row, ignore_index=True)
		except Exception as e:
			print(e)
		
		df.to_csv('result.csv', index=False)
	\end{lstlisting}
\end{center}

In de laatste fase visualiseert het onderzoek de verkregen resultaten. Hierbij gebruikt het \textit{Jupyter-notebooks} en \textit{Matplotlib} om de verkregen resultaten te visualiseren. Allereerst gebruikt de \textit{Jupyter-notebook} een kleinere dataframe met enkel de data van A1. Daarna groepeert het script de data op basis van modellen. Het aantal woorden per zin fungeert als geaggregeerd veld. Met Matplotlib kan het script vervolgens een eenvoudige boxplot genereren. De notebook herhaalt dit proces voor beide artikelen en voor alle leesmetrieken vermeldt in tabel \ref{table:verg-studie-metrieken}. Zo illlustreert \textit{listing} \ref{code:generation-boxplot} hoe het script boxplots genereert. Deze visualiseren woordgebruik van de verschillende taalmodellen bij A1. Alle code om grafieken te genereren kan de lezer op de GitHub-repository\footnote{https://github.com/dylancluyse/bachelorproef-nlp-tekstvereenvoudiging/blob main/scripts/PHASE\_5.ipynb} terugvinden. 

\begin{lstlisting}[language=Python, caption={Code om een boxplot voor het aantal woorden per zin te genereren.}, label={code:generation-boxplot}]	
artikel_1 = df[(df['paper'] == 'Artikel 1 AI') & (df['FRE'] > 0)]
data = artikel_1.groupby('model')['words_per_sentence']
data_list = [group[1].tolist() for group in data]
plt.figure(figsize=(20,10))
plt.boxplot(data_list)
plt.xticks(range(1, len(data_list) + 1), data.groups.keys())
plt.title('Woorden per zin per model')
plt.xlabel('Model')
plt.ylabel('Woorden per zin')
plt.savefig('boxplot-avg-a1.png')
\end{lstlisting}

Verder toont tabel \ref{table:verg-studie-metrieken} alle leesmetrieken, samen met de toegepaste visualisatietechniek. De boxplots dienen om de spreiding van een leesmetrieken te tonen. Zo kan het onderzoek achterhalen hoe gespreid de gebruikte woordenschat is op het vlak van de leesmetrieken FRE en FOG. De spreiding bij het aantal woorden per zinnen geeft het onderzoek ook weer als een boxplot. Zo kan het onderzoek de regelmaat van korte zinnen achterhalen uit een tekst. De \textit{violinplots} dienen om de verdeling van moeilijke of lange woorden weer te geven. Tot slot maakt het onderzoek gebruik van een staafdiagram om het aantal hulpwerkwoorden of aantal zinnen met een vervoeging van het werkwoord 'zijn' te visualiseren.

\begin{center}
	\begin{table}[H]
		\begin{tabular}{ | m{8cm} | m{7cm} | } 
			\hline
			\textbf{Leesmetriek} & \textbf{Visualisatietechniek }\\
			\hline
			FOG & Boxplot \\
			\hline
			FRE & Boxplot \\
			\hline
			Aantal woorden per zinnen & Boxplot \\
			\hline
			Aantal complexe woorden per zin volgens Dale Chall index & Violinplot \\
			\hline
			Aantal lange woorden per zin & Violinplot \\
			\hline
			Aantal gebruikte hulpwerkwoorden & Staafdiagram \\
			\hline
			Aantal zinnen met een vervoeging van het werkwoord 'zijn' & Staafdiagram \\
			\hline
		\end{tabular}
		\caption{Visualisatietechnieken voor de machinale metrieken bij de vergelijking van de vereenvoudigde teksten met de oorspronkelijke tekst en de referentieteksten.}
		\label{table:verg-studie-metrieken}
	\end{table}
\end{center}

Tenslotte komen de resultaten van de menselijke beoordeling aan bod. Deze fase van de vergelijkende studie staat stil bij aspecten die leesmetrieken niet kunnen meten, waaronder de normen vermeld in tabel \ref{table:criteria-vergelijkende-studie-human-obv}. De referentietekst dient hier als hulpmiddel om de referentietekst, ofwel het verwachte resultaat, te vergelijken met de vereenvoudigde tekst door een taalmodel. 

\begin{table}[H]
	\begin{tabular}{| m{10cm} | m{4.5cm} |}
		\hline
		\textbf{Metriek} & \textbf{Vereenvoudigings- techniek} \\ \hline
		Acroniemen behouden & LS 	\\ \hline
		Inschatting van de doelgroep & LS	\\ \hline
		Behoud van kern- en bijzaken & LS \\ \hline
		Schrijven in tabelvorm of als opsomming & SA \\ \hline
		Passieve zinconstructies herschrijven naar actieve zinconstructies & SS \\ \hline
		Bronvermelding behouden &  SA \\ \hline
		Citeren en parafraseren & SS en SA. \\ \hline
	\end{tabular}
	\caption{Criteria voor menselijke observatie bij de vergelijkende studie.}
	\label{table:criteria-vergelijkende-studie-human-obv}
\end{table}

\section{Prototype voor tekstvereenvoudiging}

Met requirements uit het moscow-schema en geschikte taalmodellen voor gepersonaliseerde ATS kan het onderzoek een volgende stap zetten om de onderzoeksvraag te beantwoorden. De volgende sectie omschrijft de ontwikkeling van Pentimentor, namelijk een prototype voor gepersonaliseerde ATS voor scholieren met dyslexie in de derde graad van het middelbaar onderwijs. Deze ontwikkeling kan ontwikkelaars helpen als rode draad bij de ontwikkeling van dergelijke toepassingen. Zo beantwoordt de ontwikkeling van \textit{Pentimentor} volgende deelvraag: 

\begin{itemize}
	\item Hoe kunnen ontwikkelaars een intuïtieve en lokale webtoepassing ontwikkelen die zowel scholieren met dyslexie als leerkrachten kan helpen bij het vereenvoudigen van wetenschappelijke artikelen met behoud van semantiek, jargon en zinsstructuren?
\end{itemize}

Voor de ontwikkeling van het prototype volgt het onderzoek figuur \ref{img:general-overview-prototype}. Deze flowchart toont zes algemene fasen. Zo start het onderzoek met een voorbereidende fase waarin het onderzoek nodige technieken en taalmodellen opsomt. Vervolgens ontwikkelt het back end en front end. Daarna ontwikkelt het twee algemene componenten: een leraren -en scholierencomponent. Hierna moet het onderzoek stilstaan bij de opzet van Pentimentor. 

\medspace

Uiteindelijk beoordeelt het onderzoek Pentimentor. Het vergelijkt dit met andere toepassingen volgens het moscow-schema. Zo moet het lerarencomponent wetenschappelijke artikelen kunnen vereenvoudigen, nadat de gebruiker ATS-technieken selecteert. Daarnaast moet het scholierencomponent ondersteuning bieden aan scholieren die de verschillen tussen origineel en vereenvoudigd willen zien. Hier moeten scholieren met dyslexie in \textit{real-time} aanpassingen kunnen maken aan de tekst en ondersteuning krijgen tijdens het begrijpend lezen van een tekst. Tot slot moeten gepersonaliseerde opmaakopties gelijk blijven over alle pagina's van Pentimentor.


\begin{sidewaysfigure}
	\begin{figure}[H]
		\includegraphics[width=\linewidth]{img/flowchart-general-development.jpg}
		\caption{Algemeen overzicht van de ontwikkeling van het prototype voor ATS van wetenschappelijke artikelen.}
		\label{img:general-overview-prototype}
	\end{figure}
\end{sidewaysfigure}

\subsubsection{Taalmodellen selecteren}

Omdat de keuze van taalmodellen verdere technologiekeuzes kan beïnvloeden, bepaalt het onderzoek eerst één of meerdere taalmodellen voor gepersonaliseerde ATS. Zo wijst de vorige onderzoeksfase uit dat GPT-3 een geschikt taalmodel is voor gepersonaliseerde ATS. Voor abstraherende samenvatting kan Pentimentor één van drie taalmodellen uit de vorige onderzoeksfase gebruiken.

\subsubsection{Programmeertalen, frameworks en verwante libraries selecteren}

Voor snelle en gratis ontwikkeling gebruikt Pentimentor \textit{open-source} pakketten. Taalmodellen kan het aanspreken via \textit{API-calls} met JS of Python. Omdat het systeem de API-sleutel moet opslaan als sessievariabele, gebeuren alle \textit{API-calls} vanuit de \textit{back end}. Verder vermeldt tabel \ref{table:technologies} alle gebruikte programmeertalen. Tot slot vermeldt tabel \ref{table:python-libraries} alle gebruikte python-libraries.

\begin{center}
	\begin{table}[H]
	\begin{tabular}{ | m{4cm} | m{11cm} | } 
		\hline
		\textbf{Technologie} 	& \textbf{Functionaliteit} \\
		\hline
		Python 					& Dit dient voor de back end die API-calls en PoS-tags verwerken. \\
		\hline
		JavaScript (JS)			& Dit maakt de toepassing gebruikersvriendelijk door commandline instructies te vervangen door eenduidige knoppen. \\
		\hline
		HTML en CSS 			& Dit past het uiterlijk aan, naargelang de gekozen parameters van de eindgebruiker. \\
		\hline
		Jinja 					& Dit geeft alle Python back end data door aan de front-end.  \\
		\hline
		Docker 					& Dit dient om Pentimentor lokaal op te kunnen starten en voorziet een lokale omgeving waarin het systeem alle software installeert. \\
		\hline
		Bash					& Script om de lokale opzet op eenduidige wijze op te starten voor Linux en Mac-systemen. \\
		\hline
		Powershell 				& Script om de lokale opzet op eenduidige wijze op te starten voor Windows-systemen. \\
		\hline
	\end{tabular}
	\caption{Gebruikte programmeertalen in het prototype voor tekstvereenvoudiging.}
	\label{table:technologies}
	\end{table}
\end{center}

\begin{center}
	\begin{table}[H]
	\begin{tabular}{ | m{4cm} | m{11cm} | } 
		\hline
		\textbf{Python-bibliotheek} & \textbf{Functionaliteit} \\
		\hline
		Flask					& Het websiteframework van Pentimentor. Dit framework combineert front end en back end. \\ 
		\hline
		PDFMiner 				& Tekstinhoud van PDF's extraheren. \\ 
		\hline
		LayoutParser			& Selectief tekstinhoud uit PDF's extraheren. \\
		\hline
		NumPy 					& De \textit{reshape}-functie vereenvoudigt de ontwikkeling om dynamisch paragrafen uit te printen. \\
		\hline		
		Spacy 					& \textit{PoS-tagging} en \textit{token lemmatization}. \\
		\hline
		OpenAI					& GPT-3 API aanspreken. \\
		\hline
	\end{tabular}
	\caption{Gebruikte Python-libraries en hun respectievelijke functie in het prototype.}
	\label{table:python-libraries}
	\end{table}
\end{center}

\subsubsection{Back end opzet}

Nadat het onderzoek taalmodellen, technologieën en frameworks uitkiest, start het met front end implementatie. Allereerst voegt het de homepagina toe. Dit dient als portaal voor de componenten. Daarna implementeert het personaliseerbare opmaakopties. Zoals aangeraden door \textcite{Galliussi2020}, moeten ontwikkelaars hiermee rekening houden. Zo gebruikt Pentimentor parameters onderzocht door \textcite{Rello2013a, Rello2013b}. Om deze instellingen aan te bieden, voegt het onderzoek een webpagina toe. Deze HTML-pagina bevat formulieren met aanpasbare parameters. Nadat de gebruiker parameters aanpast, stuurt de pagina POST-requests naar de back end. Die verwerkt dit en slaat de ontvangen dictionary op als sessievariabele, zoals weergegeven in listing \ref{code:back-end-session-personalized}. Tot slot laadt de \textit{front end} deze opmaakopties bij iedere wegpagina in. Dit gebeurt met \textit{window onload}, zoals weergegeven in listing \ref{code:window-onload-js}. Zo hoeven eindgebruikers deze parameters niet bij iedere webpagina opnieuw aan te passen.


\begin{lstlisting}[language=python, caption={De back end functie die de aanpassingen uit het formulier opslaat als sessievariabele.}, label={code:back-end-session-personalized}]
PER_SET_SESSION_NAME = 'personalized_settings'

@app.route('/get-settings-user',methods=['POST'])
def return_personal_settings_dict():
	if PER_SET_SESSION_NAME in session:
		return jsonify(session[PER_SET_SESSION_NAME])
	else:
		return jsonify(result='session does not exist')
	
@app.route('/change-settings-user', methods=['POST'])
def change_personal_settings():
	try:
		session[PER_SET_SESSION_NAME] = dict(request.form)
		msg = 'Succesvol aangepast!'
	except Exception as e:
		msg = str(e)
		flash(msg)
	return render_template('settings.html')
\end{lstlisting}

\begin{lstlisting}[language=javascript, caption={De onload-functie die de gepersonaliseerde opmaakopties regelt bij het inladen van een webpagina.}, label={code:window-onload-js}]
window.onload = async function () {
	var url = `http://localhost:5000/get-settings-user`;
	const response = await fetch(url, { method: 'POST' });
	var result = await response.json();
	document.body.style.fontSize        = result.fontSize+'px';
	document.body.style.fontFamily      = result.fontSettings;
	document.body.style.backgroundColor = result.favcolor;
	document.body.style.lineHeight      = result.lineHeight+'cm';
	document.body.style.wordSpacing     = result.wordSpacing+'cm';
	document.body.style.textAlign       = result.textAlign;
}
\end{lstlisting}

Naast personaliseerbare opmaakopties moet Pentimentor wetenschappelijke artikelen kunnen opladen en inlezen. Zo kan het met \textit{PDF-extractors} zoals \textit{PDFMiner} tekst uit de PDF extraheren. Niet alle \textit{PDF-extractors} zijn foutbestendig, want zoals opgemerkt in de literatuurstudie kunnen \textit{PDF-extractors} ook tekst verliezen tijdens dit proces. Om dit te voorkomen, biedt het prototype een OCR-optie met \textit{LayoutParser} aan. Beide manieren vereisen functies in de back end en front end.

\medspace

Aan de \textit{front end} voegt het onderzoek checkboxes, \textit{file-input} en \textit{textarea-input tags} toe. Uiteindelijk stuurt het formulier een POST-request om alle meegekregen informatie door te sturen naar de back end. Na het ontvangen van de request van de front-end, handelt de Flask back end het bestand verder af zoals aangewezen in listing \ref{code:inlezen-wetenschappelijk-artikel-front-end-back-end}. Nadat de gebruiker het formulier bevestigt, geeft het binaries van het wetenschappelijk artikel aan de back end. Die controleert het type invoer en slaat het nadien tijdelijk \textit{in-memory} op. Daarnaast krijgt het de keuze tussen normale of OCR-upload mee als boolean, waarbij \textit{true} gelijk staat aan een afhandeling met LayoutParser.

\begin{lstlisting}[language=Python, caption={Koppeling tussen front-end en back-end voor het inlezen van een wetenschappelijk artikel}, label={code:inlezen-wetenschappelijk-artikel-front-end-back-end}]
def setup_scholars_teachers(request):
	settings = request.form
	if 'fullText' in request.form:
		text = request.form['fullText']
		langs = detect_langs(text)
		reader = Reader()
		dict_text = reader.get_full_text_site(text)                
	elif 'pdf' in request.files:
		if 'advanced' not in settings:
			pdf = request.files['pdf']
			pdf_data = BytesIO(pdf.read())
			all_pages = extract_pages(pdf_data,page_numbers=None,maxpages=999)
			langs = detect_langs(str(all_pages))
			reader = Reader()
			full_text = reader.get_full_text_dict(all_pages)
			dict_text = reader.get_full_text_site(full_text)
		else:
			pdf = request.files['pdf']
			pdf_data = pdf.read()
			pages = convert_from_bytes(pdf_data)
			reader = Reader()
			img_text = reader.get_full_text_from_image(pages)
			langs = detect_langs(img_text)
			dict_text = reader.get_full_text_site(img_text)                            
			return dict_text, langs, 'voorbeeldtitel', 'voorbeeldonderwerp'
			
@app.route('/for-scholars', methods=['GET','POST'])
def teaching_tool():
	try:
		dict_text, langs, title, subject = setup_scholars_teachers(request)
		return render_template('for-scholars.html', pdf=dict_text, lang=langs, title=title, subject=subject)
	except Exception as e:
		return render_template('error.html',error=str(e))
	
@app.route('/for-teachers', methods=['GET','POST'])
def analysing_choosing_for_teachers():
	try:
		dict_text, langs, title, subject = setup_scholars_teachers(request)
		return render_template('for-teachers.html', pdf=dict_text, lang=langs, title=title, subject=subject)
	except Exception as e:
		return render_template('error.html',error=str(e))
\end{lstlisting}

\subsubsection{PDFMiner implementatie}

Bij een gewone PDF-extractie gebruikt de Reader-klasse PDFMiner met de functie in listing \ref{code:inlezen-van-pdf}. Die itereert door alle pagina's van een opgeladen wetenschappelijk artikel. Nadien extraheert het alle tekst op een pagina en concateneert dit alle opgehaalde tekst van één pagina aan een lege string. Zo herhaalt dit proces tot het geen pagina's meer kan inlezen. Tot slot resulteert dit in een string-object met alle geëxtraheerde tekst uit het wetenschappelijk artikel. Hierna moet de Reader-klasse deze tekst formatteren tot een leesbaar formaat voor het component.

\begin{lstlisting}[language=Python, caption={Een PDF inlezen met PDFMiner}, label={code:inlezen-van-pdf}]
	def get_full_text_from_pdf(self, all_pages):
		total = ""
		for page_layout in all_pages:
			for element in page_layout:
				if isinstance(element, LTTextContainer):
				for text_line in element:
					total += text_line.get_text()
					return total
\end{lstlisting}

\subsubsection{OCR implementatie}

Tekst kan ontbreken na het extraheerproces met PDFMiner. Daarnaast krijgt Pentimentor als resultaat één string van alle karakters. Tot gevolg toont het deze tekst als een niet-georganiseerde blok tekst. Daarom voorziet het prototype een tweede techniek met OCR en ML-technieken. Zo maakt Pentimentor gebruik van LayoutParser om dit probleem op te lossen. Eerst moet de back end alle pagina's van een wetenschappelijk artikel omzetten naar een afbeelding. Voor iedere afbeelding gaat het de tekst (titels en tekstblokken) markeren. Listing \ref{code:reader-ocr} illustreert het proces voor één afbeelding van een ingeladen wetenschappelijk artikel. 

\medspace

Allereerst converteert het script de afbeelding naar een array van binaire waarden die het ML-model kan interpreteren. Vervolgens geeft het deze array aan het ML-model. Dit ML-model kan tekstblokken op vier manieren classificeren, namelijk: tekst, titel, figuur of tabel. Omdat figuren en tabellen buiten de scope van dit onderzoek vallen, houdt het script enkel rekening met titels en tekst. Deze twee namen houdt het bij in een array genaamd 'valid'. Verder omkadert LayoutParser de tekstblokken en vervolgens kent het per tekstblok een id-tag toe. Deze omkadering is nodig voor de volgende stap: de OCR-fase.

\begin{lstlisting}[language=Python, caption={Een PDF inlezen met OCR}, label={code:reader-ocr}]
model = lp.Detectron2LayoutModel(
	config_path ='lp://PubLayNet/faster_rcnn_R_50_FPN_3x/config',
	label_map   ={0: "Text", 1: "Title", 2: "List", 3:"Table", 4:"Figure"}, 
	extra_config=["MODEL.ROI_HEADS.SCORE_THRESH_TEST", 0.8])

image = cv2.imread(img_folder + afbeelding)
image = image[..., ::-1]
	
layout = model.detect(image)
	
lp.draw_box(image, layout, box_width=3)
	
valid = ['Text', 'Title']
text_blocks = lp.Layout([b for b in layout if b.type in valid])
	
h, w = image.shape[:2]
	
left_interval = lp.Interval(0, w/2*1.05, axis='x').put_on_canvas(image)
	
left_blocks = text_blocks.filter_by(left_interval, center=True)
left_blocks.sort(key=lambda b:b.coordinates[1], inplace=True)
	
right_blocks = [b for b in text_blocks if b not in left_blocks]
right_blocks.sort(key=lambda b:b.coordinates[1])
	
text_blocks = lp.Layout([b.set(id=idx) for idx, b in enumerate(left_blocks+right_blocks)])
	
lp.draw_box(image, text_blocks, box_width=3, show_element_id=True)
	
\end{lstlisting}

Nu beschikt het script over een array van afbeeldingen. Zo moet het systeem deze tekst uit de afbeeldingen kunnen extraheren. Het systeem gebruikt Tesseract om de tekst, onafhankelijk van de gekozen taal, te extraheren uit de afbeeldingen. Verder haalt het enkel de tekstblokken op die Layoutparser als tekst of titel classificeert. Omdat het systeem niet achteraf bijhoudt wat titel of tekstblok is, voegt het 'titel:' toe voor iedere titelblok. Tot slot bekomt deze functie een array van alle titel- en tekstblokken.

\begin{lstlisting}[language=Python, caption={Tekst extraheren uit de geparsete inhoud.}, label={code:text-collecting}
language = language
ocr_agent = lp.TesseractAgent(languages=language)

full_text = []
	
for block in text_blocks:
	segment_image = (block.pad(left=5, right=5, top=5, bottom=5).crop_image(image))
	text          = ocr_agent.detect(segment_image)
	block.set(text=text, inplace=True)
	
for t in text_blocks:
	if(t.type == 'Title'):
		full_text.append('title:' + str(t.text))
	else:
		full_text.append(t.text)    
	return full_text
\end{lstlisting}

\subsubsection{Geëxtraheerde inhoud formatteren naar een webpagina.}

Na het extraheren van de tekstinhoud, komt een formatteerfase aan bod. Hier converteert het systeem tekst naar het gewenste formaat voor de \textit{front end}. Eerst transformeert de \textit{back end} de string van geëxtraheerde tekst naar \textit{arrays} van zinnen met Spacy \textit{word embeddings} en \textit{sentence embeddings}. Listing \ref{code:reader-formatting} omschrijft dit proces. Het prototype bundelt een dynamisch aantal zinnen met \textit{Numpy reshape}. Eindgebruikers kunnen deze parameter aanpassen in het instellingenscherm. Om de PoS-tag bij het respectievelijke woord bij te houden, gebruikt het prototype sleutelparen. Zo verwijst de sleutel naar een woord in een zin en de waarde verwijst naar de PoS-tag. Het prototype houdt rekening met enkel Nederlandstalige en Engelstalige wetenschappelijke artikelen. Daarom laadt het prototype hoogstens twee embeddingsmodellen op. Deze embeddingsmodellen staan vermeld in tabel \ref{table:wordembeddings-spacy}. Om hardcoding te vermijden, slaat het een \textit{dictionary} op met alle namen van deze embeddingsmodellen. Het prototype gebruikt Engels als standaardtaal, als de \textit{back end} de taal niet kan herkennen of als de taal niet in de lijst van de dictionary staat. 

\begin{lstlisting}[language=Python, caption={Het formatteren van de tekst naar een formaat voor de website.}, label={code:reader-formatting}]
	def get_full_text_site(self, full_text):
		try:
			lang = detect(full_text)
		except:
			lang = 'en'
		
		if lang in dict:
			nlp = spacy.load(dict.get(lang))
		else:
			nlp = spacy.load(dict.get('en'))
		
		full_text = str(full_text).replace('\n', ' ')
		
		doc = nlp(full_text)
		sentences = []
		for sentence in doc.sents:
			sentences.append(sentence)
		
		pad_size = SENTENCES_PER_PARAGRAPH - (len(sentences) % SENTENCES_PER_PARAGRAPH)
		padded_a = np.pad(sentences, (0, pad_size), mode='empty')
		paragraphs = padded_a.reshape(-1, SENTENCES_PER_PARAGRAPH)
		
		text_w_pos = []
		for paragraph in paragraphs:
			paragraph_w_pos = []
			try:
				for sentence in paragraph:
				dict_sentence = {}
				for token in sentence:
					dict_sentence[token.text] = str(token.pos_).lower()
					paragraph_w_pos.append(dict_sentence)    
					text_w_pos.append(paragraph_w_pos)
			except:
				pass
				
		return text_w_pos
\end{lstlisting}

Het leraren -en scholierencomponent gebruiken een dynamische HTML-structuur. De vorige fase van Pentimentor resulteert in een array van dictionaries. Daarom gebruikt deze fase een Jinja-iteratie waarin het iedere dictionary doorloopt. Ieder woord slaat Pentimentor op onder de PoS-tag. De PoS-tag houdt het bij als klasse. Zo kan de eindgebruiker woorden uitfilteren of specifieke woorden aanduiden om deze aan een woordenlijst toe te voegen. Iedere zin koppelt het systeem met de overkoepelende span-tag \textit{sentence}. Dan koppelt het ieder woord met de klasse \textit{nouns}, \textit{adjectives} of \textit{verbs}. Andere woorden, zoals conjuncties, koppelt het systeem met de klasse \textit{other}.

\begin{lstlisting}[language=html, caption={Het doorlopen van de PDF-tekst op de webpagina en het toekennen van de span-tags.}, label={code:html-span-tags}]

	<p class="left-side">
		
		<span class="sentence">
			
			
			<span class={{sentence[word]}}>{{word}}</span>
			
			
		</span>
	
	</p>

\end{lstlisting}


\subsubsection{Invulformulier maken voor het lerarencomponent.}

Zo kunnen leerkrachten beschikken over een tool waarin zij de geëxtraheerde tekstinhoud kunnen manipuleren, om vervolgens opties voor gepersonaliseerde ATS te selecteren. Figuur \ref{img:proto-lerarencomponent} toont een mogelijke weergave van deze HTML-pagina. Leerkrachten moeten in dit lerarencomponent kunnen beschikken over de functionaliteiten weergegeven in tabel \ref{table:functionaliteiten-leerkrachten}. Tabel \ref{table:criteria-requirementsanalysis} reikt de benodigde opties voor gepersonaliseerde ATS aan. Het prototype gebruikt deze criteria als opties om de prompts dynamisch op te bouwen.

\begin{center}
	\begin{table}[H]
		\begin{tabular}{ | m{7cm} | m{8cm} | } 
			\hline
			\textbf{Functionaliteit} & Gebruikte JS of python-techniek \\
			\hline
			Specifieke prompt meegeven per paragraaf & Naast een optie om voor het hele document één prompt te gebruiken, voegt het prototype ook een optie toe om. Hiervoor past de webinhoud sleutelparen toe. \\
			\hline
			Opties voor gepersonaliseerde ATS aanreiken. & Met behulp van een HTML-formulier kunnen leerkrachten opties aanvinken waaraan de vereenvoudigde tekst moet voldoen. \\
			\hline
			Werkwoorden, bijvoeglijke en zelfstandige naamwoorden markeren & Front-end aanpassing met \textit{eventlistener}. De tekstkleur van het aangeduide type woorden verandert naar het gekozen kleur. \\
			\hline
			Zinnen verwijderen & \textit{Front-end} filter aangesproken door een \textit{eventlistener}. \\
			\hline
			Woord toevoegen aan de woordenlijst & Een \textit{eventlistener} handelt de functionaliteit af. Het prototype slaat woorden en hun context tijdelijk op. Het formulier houdt dit bij en geeft het vervolgens mee bij het indienen. Deze woorden en hun respectievelijke zin van voorkomen dienen om de woordenlijst op te vullen in het gegenereerde PDF of Word-bestand. \\ 
			\hline 
		\end{tabular}
	\caption{Alle beschikbare functionaliteiten in het lerarencomponent.}
	\label{table:functionaliteiten-leerkrachten}
	\end{table}
\end{center}

\subsubsection{Prompts schrijven voor gepersonaliseerde ATS in het lerarencomponent}

Het onderzoek stelt de prompts op volgens de richtlijnen beschreven in tabel \ref{table:techniques-for-good-prompts}. Daarnaast past het Intent- en Contextualization prompts toe zoals aangeraden door \textcite{White2023}. Tabel \ref{table:prompts-lerarencomponent} geeft een overzicht van alle opgestelde prompts die het prototype gebruikt in het lerarencomponent. Daarnaast gebruikt iedere API-call de parameters omschreven in tabel \ref{table:gpt-3-parameters-lerarencomponent}.

\begin{center}
	
\end{center}
\begin{table}[H]
	\begin{tabular}{ | m{5cm} | m{10cm} |}
		\hline
		\textbf{Functionaliteit} & \textbf{Prompt} \\ \hline
		Synoniem opzoeken & Geef een eenvoudiger synoniem voor '{woord}'. Context {context}. \\ \hline
		Definitie opzoeken & Geef een eenvoudige definitie voor '{woord}'. Context {context}.\\ \hline
		Zin vereenvoudigen & Schrijf deze zin eenvoudiger // '{zin}'\\ \hline
		Gepersonaliseerde ATS & Schrijf deze zin eenvoudiger volgens de colgende criteria // '{zin}' // criteria: '{criteria}'\\ \hline
		Gepersonaliseerde ATS als opsomming & Herschrijf dit als een lijst van vereenvoudigde zinnen met (gepersonaliseerde opties) :return: een lijst van vereenvoudigde zinnen gesplitst door een '|' teken /// {context}\\ \hline
	\end{tabular}
	\caption{Tabel met de gebruikte prompts voor het lerarencomponent.}
	\label{table:prompts-lerarencomponent}
\end{table}

\begin{center}
	\begin{table}[H]
		\begin{tabular}{| m{5cm}| m{10cm} |}
			\hline
			\textbf{Parameter} & \textbf{Gebruikte waarde} \\ \hline
			Prompt & Zie tabel \ref{table:prompts-lerarencomponent} \\ \hline
			Temperature & 0 \\ \hline
			Max tokens & De lengte van het woord vermeerdert met tien tokens als het over het opzoeken van een definitie of synoniem gaat. \\ 
			& Of de lengte van de zin vermeerdert met twintig tokens indien het over vereenvoudiging van een zin gaat. \\
			\hline
			Model & Da Vinci 3 \\ \hline
			Top-p & 90\% \\ \hline
			Stream & False \\ \hline
		\end{tabular}
		\caption{Gebruikte parameters om definities van woorden te genereren met GPT-3.}
		\label{table:gpt-3-parameters-lerarencomponent}
	\end{table}
\end{center}


\subsubsection{Functies voor gepersonaliseerde ATS in het lerarencomponent}

Gebruikers kunnen met Pentimentor op twee manieren aan de slag gaan: het volledige artikel automatisch laten vereenvoudigen of samenvatten, of specifiek delen uit de tekst markeren om een gepersonaliseerde vereenvoudiging of samenvatting op te kunnen bouwen. Zo moet Pentimentor een nieuwe doorlopende tekst kunnen genereren, als opsomming of tabel laten herschrijven of woordenlijsten genereren. Hiervoor moet het twee zaken bijhouden: de gemarkeerde tekst en de optie die de gebruiker wenst te kiezen. Het onderzoek gebruikt hiervoor sleutelparen. Localstorage biedt een tijdelijk persistente manier om deze keuzes op te slaan. Zo moet het systeem \textit{localstorage} opvullen met gemarkeerde tekst en na gebruik opnieuw leegmaken. Als het systeem dit niet leegmaakt, dan neemt het systeem overblijfselen van vorige of irrelevante artikelen over naar artikelen. Om de eindgebruiker bewust te maken van wat het systeem met de tekst moet doen, markeert de front end deze tekst met een kleur. Verder maakt de front end de eindgebruiker bewust welke kleur bij welke transformatie hoort. Tot slot toont listing \ref{listing:localstorage} de werking van deze methode.

\begin{lstlisting}[language=javascript, caption={Implementatie rond localstorage en tekst markeren.}, label={listing:localstorage}]
function addTextWithParagraph() {
	text = window.getSelection().toString();
	if (text == "") {
		return;
	} else {
		var fieldset = document.querySelector(".personalized");
		var inputs = fieldset.querySelectorAll("input");
		var values = [];
		for (var i = 0; i < inputs.length; i++) {
			if (inputs[i].type === "checkbox" && inputs[i].checked) {
				values.push(inputs[i].value);
			}
		}
		
		
		var selection = window.getSelection();
		if (selection.rangeCount > 0) {
			var range = selection.getRangeAt(0);
			var commonAncestor = range.commonAncestorContainer;
			var selectedElements = commonAncestor.getElementsByClassName("sentence");
			var selectedTags = [];
			for (var i = 0; i < selectedElements.length; i++) {
				var elementRange = document.createRange();
				elementRange.selectNodeContents(selectedElements[i]);
				
				if (range.intersectsNode(selectedElements[i])) {
					selectedTags.push(selectedElements[i]);
				}
			}
		}
		
		fullText = "";
		
		for (var i = 0; i < selectedTags.length; i++) {
			var tag = selectedTags[i];
			if (values.includes('summation')) {
				tag.style.backgroundColor = "#F0FFF0";
			} else if (values.includes('glossary')){
				tag.style.backgroundColor = "#F5DEB3";
			} else if (values.includes('table')) {
				tag.style.backgroundColor = "#FAFAD2";
			} else {
				tag.style.backgroundColor = "#F5F5DC";
			}
			
			var tagText = tag.textContent;
			tagText = tagText.split('\n').join('').replace(/:/g, "");
			fullText += tagText;
		}
		localStorage.setItem(fullText, values);
	}
}

function getAllLocalStorageValues() {
	var localStorageValues = {};
	for (var i = 0; i < localStorage.length; i++) {
		var key = localStorage.key(i);
		var value = localStorage.getItem(key);
		localStorageValues[key] = value;
	}
	return localStorageValues;
}
\end{lstlisting}

\subsubsection{Verbinding met GPT-3}

Om de gemarkeerde tekst met opties te kunnen verwerken, moet het systeem de inhoud van de \textit{localstorage} doorsturen naar de back end. Vervolgens moet het ieder sleutelpaar overlopen. Iedere sleutel stelt een zin of paragraaf voor dat het systeem moet vereenvoudigen of samenvatten. Als de tekst het maximum aantal tokens overschrijdt, dan splitst de back end deze tekst op in een aantal delen dat gelijk staat aan een gelijke deling van het maximum aantal tokens, zoals weergegeven in listing \ref{code:localstorage-iteration}.

\begin{lstlisting}[language=Python, caption={Alle gemarkeerde tekstblokken uit de localstorage aflopen en doorsturen naar het markdown document.}, label={code:localstorage-iteration}]
for key, value in full_text.items():
	new_sentence = []
	for word in key.split(" "):
		if word != '':
			new_sentence.append(word)
	sent            = " ".join(new_sentence)
	aantalDelingen  = (len(sent)  // 1000) + 1
	perHoeveel      = (len(sent)) // aantalDelingen
	sent = [sent[i:i+perHoeveel] for i in range(0, len(sent), perHoeveel)]
	
	for s in sent:
		new_sent, prompt = gpt.personalised_simplify(s, str(value).split(','))
		if 'summation' in str(value).split(','):
			doc_creator.generate_summary_w_summation(new_sent)
		elif 'table' in str(value).split(',') or 'glossary' in str(value).split(','):
			doc_creator.generate_simplification(new_sent)
		else:
			doc_creator.generate_simplification(new_sent)
\end{lstlisting}

Vervolgens moet de \textit{back end} deze prompts naar de GPT-3 API sturen. Zo itereert het over ieder sleutelpaar en bouwt het daarmee prompts op. Deze bestaat uit gekozen opties en gemarkeerde tekst van de gebruiker. Naast de prompt dient de back end ook parameters mee te geven. Deze omvatten alle waarden opgelijst in tabel \ref{table:gpt-3-parameters-lerarencomponent}. Omdat formaatwijzigingen specifieke prompts vereisen, bestaat de prompt uit extra klemtonen om het resultaat in markdowncode terug te krijgen. Als de \textit{back end} geen formaatwijziging moet uitvoeren, dan stuurt het generieke prompts met daarin een opsomming van alle kenmerken die de eindgebruiker wilt zien in de vereenvoudigde tekst. Listing \ref{code:gpt-api-calls} toont de code die de back end voor één gemarkeerd tekstblok uitvoert. Deze instructies herhaalt het voor ieder tekstblok dat de gebruiker heeft gemarkeerd.

\begin{lstlisting}[language=Python, caption={API-calls versturen naar de GPT-3 API.}, label={code:gpt-api-calls}]
def personalised_simplify(self, sentence, personalisation):
	if 'table' in personalisation:
	prompt = f"""
		Herschrijf de tekstinhoud maar in een tabel, gebruik twee kolommen naar keuze; schrijf dit in markdowncode.
		///
		{sentence}
		"""
	elif 'glossary' in personalisation:
		prompt = f"""
		Maak een woordenlijst (max 5 woorden) in tabelvorm van het gebruikte jargon uit deze tekst; schrijf dit in markdowncode. ///
		{sentence}
		"""
	else:
		prompt = f"""
		Vereenvoudig de zinnen met de volgende kenmerken: {", ".join(personalisation)}
		///
		{sentence}
		"""
	
	try:
		result = openai.Completion.create(prompt=prompt,temperature=0,max_tokens=len(prompt),model=COMPLETIONS_MODEL,top_p=0.9,stream=False)["choices"][0]["text"].strip(" \n")
		if 'summation' in personalisation:
			result = result.split('.')
		elif 'table' in personalisation or 'glossary' in personalisation:
			result = result
		else:
			result = result
		return result, prompt
	except Exception as e:
		return str(e), prompt
\end{lstlisting}

\subsubsection{Pandoc implementatie}

Ten slotte moet het prototype de \textit{plain-text} van vereenvoudigde tekstinhoud en de woordenlijsten in een pdf of docx-bestand gieten. Zo genereert de Creator-klasse pdf en docx-documenten volgens de meegegeven opmaakopties. Het genereren van pdf en docx-documenten gebeurt met Pandoc via python.  Pandoc gebruikt een tweestapsbeweging, waarbij het eerst \textit{plain-text} naar een markdownformaat omzet en vervolgens het Markdown-bestand naar een pdf of docx-document converteert. Daarvoor is een YAML-header nodig die de elementen, beschreven in tabel \ref{table:personalized-pdf-word-document-with-pandoc}, moet bevatten.

\begin{table}[H]
	\begin{tabular}{ | m{5cm}| m{10cm} | }
		\hline
		\textbf{Label in YAML-header} & \textbf{Voorbeeldwaarde} \\ \hline
		Title & Surveillance met artificiële intelligentie. \\ \hline
		Mainfont & Arial \\ \hline 
		Titlefont & Arial Black \\ \hline
		Date & 14-06-2023 \\ \hline 
		Document & Article \\ \hline
		Margin & 3cm \\ \hline
		Word-spacing & 0.3cm \\ \hline 
		Lineheight & singleheight \\ \hline
	\end{tabular}
	\caption{Benodigde labels voor een gepersonaliseerd document met Pandoc.}
	\label{table:personalized-pdf-word-document-with-pandoc}
\end{table}

Allereerst bouwt het prototype een markdown-bestand met daarin de YAML-header. Zo bouwt listing \ref{code:yaml-header-function} de YAML-header op. Deze header is volledig parameteriseerbaar.

\begin{lstlisting}[language=Python, caption={Writer-klasse omvattende de code om dynamische PDF- en Word-documenten te genereren.}, label={code:yaml-header-function}]
	markdown_file = "saved_files/file.md"
	DATE_NOW = str(date.today())
	
	class Creator():
	def create_header(self, title, margin, fontsize, chosen_font, chosen_title_font, word_spacing, type_spacing):
		with open(markdown_file, 'w', encoding='utf-8') as f:
			f.write("---\n")
			f.write(f"title: {title}\n") 
			f.write(f"mainfont: {chosen_font}.ttf\n")
			f.write(f"titlefont: {chosen_title_font}.ttf\n")
			f.write(f'date: {DATE_NOW}\n')
			f.write(f'document: article\n')
			f.write(f'geometry: margin={margin}cm\n')
			f.write(f'fontsize: {fontsize}pt\n')
			f.write('header-includes:\n')
			f.write(f'- \spaceskip={word_spacing}cm\n')
			f.write(f'- \\usepackage{{setspace}}\n')
			f.write(f'- \{type_spacing}\n')
			f.write("---\n")
\end{lstlisting}

Om de woordenlijst aan het markdown-bestand toe te voegen, bouwt het prototype vervolgens een \textit{dictionary}-structuur op met de positie van het woord als key en als values de woord, de PoS-tag en de opgehaalde gepersonaliseerde betekenis. Listing \ref{code:writer-glossary-klasse} toont deze functie. Het prototype moet rekening houden met homoniemen en daarom kan de key hier niet het woord zijn. Bij een lege woordenlijst komen deze bewerkingen niet aan bod. 

\begin{lstlisting}[language=Python, caption={Een woordenlijst genereren met de Writer-klasse.}, label={code:writer-glossary-klasse}]
def generate\_glossary(self, list):
	with open(markdown_file, 'a', encoding='utf-8') as f:
		f.write("---\n")
		f.write("# Woordenlijst\n")
		f.write("| Woord | Soort | Definitie |\n")
		f.write("| --- | --- | --- |\n")
		for word in list.keys(): 
			f.write(f"| {word} | {list[word]['type']} | {list[word]['definition']} |\n")
\end{lstlisting}

Daarna vult het script het markdownbestand op met de vereenvoudigde tekstinhoud, zoals weergegeven in listing \ref{code:writer-doorlopende-klasse}. Deze tekst is gesplitst door de titels die de leerkracht heeft gekozen. Daarna slaat het script de vereenvoudigde tekst op in een \textit{dictionary}-structuur. Vervolgens print het script de vereenvoudigde tekst uit naar het markdownbestand door alle titels van de \textit{dictionary}-structuur te doorlopen. Een titel uitprinten in markdown syntax moet voorafgaan aan twee \textit{hashtags}, gevolgd door een \textit{breakline}.

\begin{lstlisting}[language=Python, caption={Een doorlopende tekst toevoegen aan het markdownbestand met de Writer-klasse.}, label={code:writer-doorlopende-klasse}]
def generate_simplification(self, full_text):
	with open(markdown_file,'a', encoding="utf-8", errors="surrogateescape") as f:
		for key in full_text.keys():
			title = str(key).replace('\n',' ')
			text = full_text[key]
			f.write('\n\n')
			f.write(f'## {title}')
			f.write('\n\n')
			f.write(" ".join(text))
			f.write('\n\n')
\end{lstlisting}

Het prototype voert een andere functie uit als de tekst een opsomming moet zijn in het uitvoerbestand. Listing \ref{code:writer-summation-klasse} toont deze functie. Als de leerkracht een opsomming wenst, dan dient de titel nog steeds al separator. Enkel zal het script de zinnen uitprinten als opsomming conform aan de markdownsyntax. Na de titel print het script de vereenvoudigde tekst per paragraaf uit. Bij een opsomming gaat een asterisk-symbool vooraf. Vervolgens converteert Pandoc het Markdown-bestand naar een PDF-bestand gebouwd met de XeLateX engine of een Word-bestand met de meegekregen binaries. 

\begin{lstlisting}[language=Python, caption={Een opsomming toevoegen aan het markdownbestand met de Writer-klasse.}, label={code:writer-summation-klasse}]
def generate_simplification_w_summation(self, full_text):
	with open(markdown_file,'a', encoding="utf-8", errors="surrogateescape") as f:
	for key in full_text.keys():
		title = str(key).replace('\n',' ')
		text = full_text[key][0].split('|')
		f.write('\n\n')
		f.write(f'## {title}')
		for sentence in text:    
		f.write('\n\n')
		f.write(f'* {sentence}')
		f.write('\n\n')
\end{lstlisting}

Tenslotte maakt het script de pdf en docx-bestanden van de vereenvoudigde teksten. Beide bestanden maken gebruik van dezelfde YAML-header. Listing \ref{code:writer-create-pdf} gebruikt alle bovenstaande functies indien nodig. Daarna genereert Pandoc de pdf en docx-bestanden en zal de functie deze twee bestanden comprimeren tot één zip-bestand. Hoewel Flask maar één bestand kan teruggeven, comprimeert het script met de \textit{zipfile} bibliotheek deze twee bestanden tot één bestand. Zo krijgt de eindgebruiker alsnog zowel het docx als het pdf-document. 

\begin{lstlisting}[language=Python, caption={Een zip-bestand aanmaken met daarin een docx en pdf bestand van de vereenvoudigde tekst.}, label={code:writer-create-pdf}]
def create_pdf(self, title, margin, list, full_text, fonts, word_spacing, type_spacing, summation):
	if title is not None:
		self.create_header(title=title, margin=margin, fontsize=14, chosen_font=fonts[0], chosen_title_font=fonts[1], word_spacing=word_spacing, type_spacing=type_spacing)
	else:
		self.create_header(title='Vereenvoudigde tekst', margin=0.5, fontsize=14, chosen_font=fonts[0], chosen_title_font=fonts[1], word_spacing=word_spacing, type_spacing=type_spacing)
	
	if len(list) != 0:
		self.generate_glossary(list=list)
	
	if summation:
		self.generate_summary_w_summation(full_text=full_text)
	else:
		self.generate_summary(full_text=full_text)
	
	pypandoc.convert_file(source_file=markdown_file, to='docx', outputfile=docx_file,   extra_args=["-M2GB", "+RTS", "-K64m", "-RTS"])
	pypandoc.convert_file(source_file=markdown_file, to='pdf',  outputfile=pdf_file,    extra_args=['--pdf-engine=xelatex'])
	with zipfile.ZipFile(zip_filename, 'w') as myzip:
		myzip.write(pdf_file)
		myzip.write(docx_file)
\end{lstlisting}

De functionaliteiten van het lerarencomponent stoppen hier. Vervolgens komt het scholierencomponent aan bod, waarbij de nadruk ligt op het ontwikkelen van een ondersteunende tool.

\subsection{De ontwikkeling van het scholierencomponent.}

Tabel \ref{table:beschikbare-functionaliteiten-scholierencomponent} geeft een overzicht van alle functionaliteiten die in het scholierencomponent moeten zitten.

\begin{table}
	\begin{tabular}{| m{10cm} | m{5cm} |}
		\hline
		\textbf{Functionaliteit} & \textbf{JS/GPT-3} \\ \hline
		Zinnen verwijderen & JS \\ \hline
		Zinnen vereenvoudigen met gepersonaliseerde keuzes & GPT-3 en JS \\ \hline
		Zinnen vereenvoudigen met prompt & GPT-3 en JS \\ \hline
		Woord aan woordenlijst toevoegen & GPT-3 en JS \\ \hline
		Woorden (werkwoorden, bijvoeglijke en zelfstandige naamwoorden) markeren & JS \\ \hline
	\end{tabular}
	\caption{Beschikbare functionaliteiten in het scholierencomponent.}
	\label{table:beschikbare-functionaliteiten-scholierencomponent}
\end{table}

\subsubsection{Display maken voor scholierencomponent}

Net zoals bij het lerarencomponent, moet het prototype een oorspronkelijke weergave van het wetenschappelijk artikel kunnen tonen. Het onderzoek baseert de lay-out op dat van de erkende softwarepakketten, alsook de uitgeteste chatbots. 

\subsubsection{Prompts schrijven voor gepersonaliseerde ATS}

De prompts in tabel \ref{table:prompts-lerarencomponent} kan het scholierencomponent overnemen van het lerarencomponent. Aanvullend hierop kunnen scholieren zelf een prompt schrijven. Het systeem gebruikt deze prompt met de gemarkeerde tekst als input voor GPT-3.

\subsubsection{Back end functies schrijven voor gepersonaliseerde ATS}

Om zinnen in de doorlopende tekst te verwijderen, moet de front end over de nodige span-tags beschikken. Bij het inlezen van de PDF voert de back end dit al uit, zoals verwezen in listing \ref{code:reader-formatting}. Daarom hoeft het onderzoek geen nieuwe functie in de back end te schrijven. Vervolgens moet de back end over een functie beschikken om een zin te vereenvoudigen. Het lerarencomponent gebruikt al dergelijk functie. Daarmee hergebruikt de \textit{back end} de functie in listing \ref{listing:gpt-personalised-simplify}. 

\medspace

Zinnen baseren op een zelfgemaakte prompt moet de \textit{back end} echter wel toevoegen. Zo gebruikt de back end de functie in listing \ref{code:custom-prompt} om een \textit{custom prompt} te verwerken. Deze functie stuurt de oorspronkelijke prompt, samen met de context, door naar de GPT-3 API.

\begin{lstlisting}[language=python, caption={Een API-call sturen naar GPT-3 met een custom prompt.}, label={code:custom-prompt}]
def personalised_simplify_w_prompt(self, sentences, personalisation):
	try:
		result = openai.Completion.create(
			prompt=personalisation,
			temperature=0,
			max_tokens=len(personalisation)+len(sentences),
			model=COMPLETIONS_MODEL,
			top_p=0.9,
			stream=False
		)["choices"][0]["text"].strip(" \n")
		return result, personalisation
	except Exception as e:
		return str(e), personalisation
\end{lstlisting}

% 4.

Vervolgens heeft de back end als een functie die een woord opzoekt met GPT-3. Deze functie staat beschreven in listing \ref{listing:gpt-look-up-word} en de back end kan deze functie zonder problemen hergebruiken. Tot slot hoeft de \textit{back end} geen nieuwe functie te krijgen om woordmarkering af te handelen.

\subsubsection{Front end implementatie voor ATS functionaliteiten}

% 1.

Allereerst kan de \textit{front end} zonder hulp van de \textit{back end} zinnen met JS verwijderen. Alle woorden in een zin bundelt de front end in een span-tag van de klasse 'sentence'. Als de gebruiker dergelijk span-tag aanklikt, dan verwijdert de \textit{front end} de gekozen span-tag.

% 2.

Eerst slaat JS de gemarkeerde tekst en meegekregen ATS-opties op en geeft deze door aan de back end met een API-call. Vervolgens verwerkt de back end deze aanvraag door een nieuwe aanvraag te sturen naar GPT-3, met daarin een prompt die de tekst en de gekozen ATS-technieken bevat. De prompt specifieert het formaat waaraan de uitvoer van het taalmodel moet voldoen.  Als de tekst doorlopend is, dan verwerkt JS dit resultaat als een p-tag.  Als het taalmodel een opsomming moest genereren, dan zal de front-end alle zinnen doorlopen en uitprinten tussen twee li-tags.

% 3.

Listing \ref{code:frontend-add-word-to-glossary} toont de aanpak om via de front end een woord, pos-tag en definitie toe te voegen aan de woordenlijst tabel. De front end handelt enkel aanvragen voor werkwoorden, adjectieven, hulpwerkwoorden en zelfsandige naamwoorden af. Eenmaal de scholier op een woord drukt, stuurt de front end een aanvraag naar de back end. Hiervoor stuurt de front end de context en het woord naar de back end. De front end voegt nadien een nieuwe rij aan de tabel toe met daar in het woord, de PoS-tag en de definitie.

\begin{lstlisting}[language=javascript, caption={Een woord aan de woordenlijst toevoegen in het scholierencomponent.}, label={code:frontend-add-word-to-glossary}]
document.addEventListener("DOMContentLoaded", () => {
	const spans = document.querySelectorAll(".verb, .adj, .noun, .aux");
	spans.forEach((span) => {
		span.addEventListener("click", async (event) => {
			const radioButton = document.querySelector("#explainWords");
			if (radioButton && !radioButton.checked) {
				return;
			}
			let leftSideTag = span.closest("p");
			let rightSideTag = leftSideTag.nextElementSibling;
			sentence_of_origin = span.closest(".sentence");
			
			var context = "";
			for (const child of sentence_of_origin.children) {
				context = context + " " + child.textContent;
			}
			const word = event.target.textContent;
			const response = await fetch(`http://localhost:5000/look-up-word`, {
				method: "POST",
				headers: { "Content-Type": "application/json" },
				body: JSON.stringify({ word: word, sentence: context }),
			});
			result = await response.json();
			
			if (result.result == "error") {
				alert("Incorrect API key provided: " + result.word);
			} else {
				var pos_tag = result.result.split('|')[0];
				var definition = result.result.split('|')[1];
				let table = document.querySelector(".table-glossary");
				let newRow = table.insertRow(-1);
				let cell1 = newRow.insertCell(0);
				let cell2 = newRow.insertCell(1);
				let cell3 = newRow.insertCell(2);
				cell1.innerHTML = result.word;
				cell2.innerHTML = pos_tag;
				cell3.innerHTML = definition;
			}
		});
	});
});
\end{lstlisting}

% 4.

Tot slot moet de front end specifieke woorden kunnen markeren. De front-end beschikt al over de PoS-tags. Zo kan de front end, met de JS-functie in listing \ref{code:frontend-mark-pos-tag}, deze woorden een specifieke kleur geven als markering. De listing toont enkel het markeren van zelfstandige naamwoorden. Daarnaast kan de front end ook adjectieven uit de tekst verwijderen zonder taalmodel. Zo hoeft de JS-functie enkel de span-tags van de klasse 'adj' verwijderen uit de document.

\begin{lstlisting}[language=javascript, caption={Zelfstandige naamwoorden in het scholierencomponent markeren.}, label={code:frontend-mark-pos-tag}]
const nouns = document.getElementById('noun-show');

nouns.addEventListener('change', function () {
	if (this.checked) {
		const color = document.getElementById('colorForNouns').value;
		const elements = document.querySelectorAll("span.noun");
		elements.forEach(function (element) {
			element.style.color = color;
		});
	} else {
		const elements = document.querySelectorAll("span.noun");
		elements.forEach(function (element) {
			element.style.color = "black";
		});
	}
});
\end{lstlisting}

\subsection{De opzet voor een lokale webtoepassing.}

Eindgebruikers kunnen voorlopig enkel Pentimentor via \textit{commandline} opstarten. Het online plaatsen valt buiten de \textit{scope} van dit onderzoek, maar Docker en zelfgeschreven scripts bieden een eenduidige opstart. Hierbij moet de eindgebruiker geen technische vaardigheden gebruiken. Omdat Pentimentor enkel API's aanspreekt, werkt het met één Docker-container. Listing \ref{code:dockerfile} toont de zelfgeschreven code van de Dockerfile. Om de installatie van Python soepel te laten verlopen, bouwt het onderzoek een lijst van nodige python-bibliotheken op. Met Pipreq kan dit automatisch. Daarnaast moet het systeem de nodige Spacy word-embeddings en LayoutParser pakketten installeren. Dat gebeurt met aparte commando's in de Dockerfile. Dit proces als eerste laten verlopen kan de versies tegen het einde van de ontwikkeling gedateerd maken. Ten laatste moeten ook alle lettertypen zich in de lokale map bevinden, want Pandoc moet over alle lettertypen beschikken om een docx-bestand te maken. Een scriptbestand in Powershell, zoals weergegeven in \ref{code:shell-boot}, of Bash zoals weergegeven in \ref{code:bash-boot}, maakt de opstart van deze webapplicatie intuïtiever dan via commandline. Hiermee kunnen MacOS, Linux en Windows-gebruikers Pentimentor installeren. De automatische installatie van Docker valt buiten het doel van het script, maar het systeem prompt gebruikers om eerst Docker te installeren.

\begin{lstlisting}[language=Dockerfile, caption={Dockerfile voor Pentimentor.}, label={code:dockerfile}]
FROM python:3.8-slim-buster
WORKDIR /app
COPY requirements.txt requirements.txt
RUN apt-get update && apt-get install -y pandoc texlive-xetex texlive poppler-utils
RUN pip3 install -r requirements.txt \
	&& python3 -m spacy download nl_core_news_md \
	&& python3 -m spacy download en_core_web_md
COPY . .	
CMD [ "python3", "-m" , "flask", "run", "--host=0.0.0.0", "--port=5000"]
\end{lstlisting}


\begin{lstlisting}[language=Powershell, caption={Script voor het opstarten van de Docker-container voor Windows-gebruikers}, label={code:shell-boot}]
@echo off
cd web-app
docker stop text-application-prototype
docker rm text-application-prototype
docker rmi text-app
docker build -t text-app .
docker run --name text-application-prototype --network webapp_simplification -d -p 5000:5000 text-app
\end{lstlisting}

\begin{lstlisting}[language=Bash, caption={Script voor het opstarten van de Docker-container voor Unix-gebruikers}, label={code:bash-boot}]
#!/bin/sh	
cd web-app || exit
docker stop text-application-prototype
docker rm text-application-prototype
docker rmi text-app
docker build -t text-app .
docker run --name text-application-prototype --network webapp_simplification -d -p 5000:5000 text-app
\end{lstlisting}


\subsection{Experimenten met Pentimentor en vergelijkingen met bestaande toepassingen.}

Na de ontwikkeling van Pentimentor voert het onderzoek twee testen uit. Zo kan het achterhalen of Pentimentor voldoet aan de opgestelde functionaliteiten uit de requirementsanalyse, weergegeven in tabel \ref{img:moscow-table}. Alle experimenten gebruiken de wetenschappelijke artikelen uit tabel \ref{table:referentieteksten-bronvermelding}. Daarnaast gebruiken ze de parameters beschreven in tabel \ref{table:chosen-parameters-experiment}. Tot slot vergelijkt het onderzoek de uitvoer van Pentimentor met de oorspronkelijke wetenschappelijke artikelen en referentieteksten door MTS. 

\begin{table}
	\begin{tabular}{| m{5cm} | m{5cm} |}
	\hline
	\textbf{Parameter} & \textbf{Gekozen variabele} \\ \hline
	Standaardlettertype & Arial \\ \hline
	Lettertype voor titel & Arial \\ \hline
	Regeleinde & Anderhalve \\ \hline
	Woordspatiëring (in cm) & 0.5  \\ \hline
	Documentmarge (in cm) & 3 \\ \hline
	Schrijven als & Opsomming \\ \hline
	\end{tabular}
	\caption{Gekozen parameter voor experimenten.}
	\label{table:chosen-parameters-experiment}
\end{table}
\chapter{\IfLanguageName{dutch}{Resultaten}{Results}}%
\label{ch:resultaten}

In dit hoofdstuk overloopt het onderzoek de resultaten van alle onderzoeksmethoden. Deze omvatten de requirementsanalyse, de vergelijking van taalmodellen en de ontwikkeling van \textit{Pentimentor}. Allereerst bespreekt dit hoofdstuk de resultaten van de requirementsanalyse door het moscow-schema op de uitgeteste tools toe te passen. Daarna staat het onderzoek stil bij de verkregen machinale en menselijke resultaten bij de vergelijking van vier taalmodellen. Met deze resultaten kan het onderzoek een geschikt taalmodel voor personaliseerbare \textit{automatic text simplification} (ATS) uitkiezen. Tot slot bespreekt het onderzoek het Pentimentor en vergelijkt het dit met vergelijkbare tools op basis van functionaliteiten en verschillende uitvoer. 

\medspace

Verder bevatten de resultaten enkel de belangrijkste grafieken. Alle resultaten slaat het onderzoek op in csv-bestanden. Deze bestanden kan de lezer terugvinden op de bijhorende GitHub-repository\footnote{https://github.com/dylancluyse/bachelorproef-nlp-tekstvereenvoudiging/blob/main/STATISTIEKEN\_BACHELORPROEF\_ATS.csv}.

\section{Ontbrekende functies bij AI-toepassingen}

Tabel \ref{table:afgetoetste-criteria} toont de resultaten van de requirementsanalyse.

\begin{table}[H]
	\centering
	\begin{tabular}{ | m{8cm} | m{0.5cm} | m{0.5cm} | m{0.5cm} | m{0.5cm} | m{0.5cm} | m{0.5cm} | m{1cm} | m{1cm} | }
		\hline
		\textbf{Richtlijn} & \textbf{E1} & \textbf{E2} & \textbf{E3} & \textbf{O1} & \textbf{O2} & \textbf{O3} & \textbf{O4} & \textbf{O5} \\ \hline
		Rekening houden met doelgroep & - & - & - & - & - & - & P2 & P2 \\ \hline
		Woorden met minder lettergrepen gebruiken & - & X & - & X & - & - & P1-6 & P1-6 \\ \hline
		Extra uitleg schrijven bij zinnen & - & X & - & - & - & - & P1-3 & P1-3 \\ \hline
		Paragrafen herschrijven zodat ze eerst uitleg geven op een high-level niveau & - & - & - & - & - & - & P2 & P2 \\ \hline
		Woordenlijst aanmaken & X & X & X & X & - & - & P6 & P6 \\ \hline
		Synoniemenlijst aanmaken & - & X & - & - & - & - & P6 & P6 \\ \hline
		Idiomen vervangen door eenvoudigere synoniemen & - & - & - & X & - & - & P1-3,6 & P1-3,6 \\ \hline
		Zinnen inkorten & - & - & - & X & X & X & P3-5 & P3-5 \\ \hline
		Verwijswoorden aanpassen & - & - & - & X & - & X & P3 & P3 \\ \hline
		Voorzetseluitdrukkingen aanpassen & - & - & - & - & - & - & P3 & P3 \\ \hline
		Samengestelde werkwoorden aanpassen & - & - & - & X & - & X & P3 & P3 \\ \hline
		Actieve stem toepassen & - & - & - & - & - & - & - & - \\ \hline
		Enkel regelmatige werkwoorden gebruiken & - & - & - & - & - & - & P3 & P3 \\ \hline
		Achtergrondkleur aanpassen & X & X & X & - & - & - & - & - \\ \hline
		Woord- en karakterspatiëring aanpassen & - & X & X & - & - & - & - & - \\ \hline
		Consistente lay-out & X & X & X & - & - & - & P1-6 & P1-6 \\ \hline
		Duidelijk zichtbare koppenstructuur & X & X & X & - & - & - & X & X \\ \hline
		Huidige positie benadrukken & X & X & X & - & - & - & - & - \\ \hline
		Waarschuwingen geven omtrent formulieren en sessies & - & - & - & X & - & X & - & - \\ \hline
		Inhoud visueel groeperen & - & X & X & - & - & - & - & - \\ \hline
		Tekst herschrijven als tabel & - & - & - & - & - & - & P4, P6 & P4, P6 \\ \hline
		Tekst herschrijven als opsomming & - & - & - & - & - & - & P5 & P5 \\ \hline
		Artikel opladen als pdf & X & X & X & - & X & X & - & - \\ \hline
		Artikel opladen als \textit{plain-text} & - & - & - & X & - & X & P1-6 & P1-6 \\ \hline
		Artikel opladen via link & - & - & - & - & X* & - & P1-6* & - \\ \hline
	\end{tabular}
	\caption{Afgetoetste criteria volgens de experimenten.}
	\label{table:afgetoetste-criteria}
\end{table}

\subsubsection{Experimenten op de must-have functionaliteiten}

Allereerst wijzen de experimenten uit dat niet alle toepassingen de vermelde personaliseerbare opmaakopties aanreiken. Zo beschikken enkel E1, E2 en E3 over deze opties. Andere tools bieden enkel een statische webweergave aan. Verder bieden alle uitgeteste tools een methode aan om eenduidig pdf-bestanden op te slaan, met uitzondering op O1, O4 en O5. Deze uitvoerbestanden ontbreken echter een duidelijke titelstructuur en deze kunnen eindgebruikers niet eenduidig als word-bestand raadplegen. Verder kunnen gebruikers met de geteste toepassingen een wetenschappelijk artikel als pdf opladen. Uitzonderlijk laten O4 en O5 dit niet toe. Zij werken enkel met tekstinvoer. Figuur \ref{img:scispace-example} toont de werking van toepassing O2. Deze figuur toont hoe een eindgebruiker de tekst van het oorspronkelijk pdf-bestand kan selecteren. Vervolgens kunnen zij dit eenvoudig laten samenvatten door de toepassing. Tot slot kunnen zij uitzonderlijk met O2 en O5 online wetenschappelijke artikelen via URL opladen.

\begin{figure}[H]
	\includegraphics[width=\linewidth]{img/typeset-example.png}
	\caption{Informatie opvragen van een wetenschappelijk artikel met SciSpace}
	\label{img:scispace-example}
\end{figure}

Verder moeten toepassingen een moeilijk woord in een doorlopende tekst kunnen aanpassen. Specifiek kunnen O1, O3, O4 en O5 een annotatie toevoegen aan moeilijke woorden, maar dit gebeurt alleen als er daarvoor geen geschikt synoniem beschikbaar is. Zo illustreert figuur \ref{img:simplish-output} hoe O1 een extra definitie als annotatie kan geven. Hoewel dit het taalniveau niet kan aanpassen, toch laat figuur \ref{img:scholarcy} zien hoe O3 dit probeert in te schatten met een \textit{rewordifying level}. Bovendien kunnen O4 en O5 de doelgroep aanpassen afhankelijk van de \textit{prompts}. Andere uitgeteste tools tonen niet hoe zij de doelgroepinschatting maken. Geen tool laat gebruikers toe om een vooraf opgestelde woordenlijst met moeilijke woorden mee te geven waarop de tool zich kan baseren.

\begin{figure}[H]
	\includegraphics[width=\linewidth]{img/simplish-output.png}
	\caption{Schermafbeelding van de tekstanalyse bij Simplish na een tekstvereenvoudiging.}
	\label{img:simplish-output}
\end{figure}

Daarnaast kunnen toepassingen ook handmatig een woordenlijst maken. Zo kunnen E1, E2, E3 en O1 een woorden- of synoniemenlijst opstellen. Bij deze toepassingen selecteren gebruikers handmatig moeilijke woorden. Die moet de toepassing vereenvoudigen of een definitie ervan ophalen. Uitzonderlijk O1 geeft de woordsoort mee. Verder kunnen gebruikers niet bepalen uit welke bron de definitie moet komen, bijvoorbeeld een online woordenboek. Enkel E1 laat dit toe.

\medspace

Verder kunnen E1, E2 en E3 geen syntactische vereenvoudiging toepassen op de oorspronkelijke tekst. Overige uitgeteste tools kunnen zinnen inkorten door deze te splitsen. Geen van de uitgeteste tools kunnen de geselecteerde tekst automatisch naar een actieve vorm schrijven. In tegenstelling tot O4 en O5 die dit wel kunnen doen, maar enkel als de tool een onderwerp in de prompt meekrijgt. 

\medspace

Tot slot slagen O2, O4 en O5 erin om de tekst te herschrijven als opsomming. Zo doet O2 dit automatisch, vergeleken met O4 en O5 die deze vraag expliciet in hun prompt moeten krijgen. De andere uitgeteste tools kunnen dit niet automatisch doen. 

\subsubsection{Should-haves}

Allereerst tonen O1 en O3 automatisch leesgraadmetrieken nadat die een vereenvoudiging of samenvatting maken van de oorspronkelijke tekst. Zo tonen figuren \ref{img:simplish-output} en \ref{img:scholarcy} een voorbeeld van deze analyse. Deze analyse toont het aantal zinnen en complexe/lange woorden voor zowel het oorspronkelijk als het vereenvoudigde artikel. Andere uitgeteste tools tonen dit niet.

\begin{figure}[H]
	\includegraphics[width=\linewidth]{img/scholarcy-attempt.png}
	\caption{Tekstanalyse met \textit{Rewordify}.}
	\label{img:scholarcy}
\end{figure}

Verder kan het onderzoek niet afleiden of de uitgeteste tools een OCR-techniek gebruiken. Wel gebruikt O2 een andere inleestechniek dan de andere tools. Zo kan het de twee gebruikte wetenschappelijke artikelen inlezen met een identieke lay-out als het oorspronkelijk artikel. Daarna markeert de gebruiker enkel aanpassingen in het artikel, terwijl alle uitvoer van het taalmodel rechts in beeld komt. Echter, de gebruiker kan deze aanpassing niet afleiden uit de oorspronkelijke tekst, in tegenstelling tot O1 en O3. Die tonen wel de verschillen tussen het oorspronkelijk en het vereenvoudigd artikel aan de eindgebruiker.

\subsubsection{Could-haves}

Verschillende geteste tools maken gebruik van gebruikerfeedbacktechnieken in de vorm van \textit{pop-ups}, zoals bij een aanpassing van de tekst. O4 en O5 vormen hierop een uitzondering, omdat zij deze technieken niet bevatten. Uitzonderlijk O4 en O5 kunnen tekst omzetten naar een tabelformaat, maar dit gebeurt alleen na een expliciete prompt. Ze hebben de mogelijkheid om tekst te interpreteren en deze in een tabel te gieten. Zo maken ze gebruik van een 2 op 2 tabel, maar de gebruiker kan het aantal kolommen en rijen instellen via de prompt.

\medspace

Geen van de geteste tools kan automatisch moeilijke woorden of vakterminologie extraheren uit een tekst, behalve O4 en O5, die dit wel kunnen met behulp van een expliciete prompt.

\medspace

Wat betreft samenvattingen kunnen O1, O2, O3, O4 en O5 zowel extraherende als abstraherende samenvattingen maken van de oorspronkelijke tekst. E1, E2 en E3 kunnen alleen een extraherende samenvatting maken, maar dit gebeurt pas na een handmatige selectie van de zinnen.

\medspace

Alleen O4 en O5 hebben de mogelijkheid om een gegeven tekst te herschrijven. Dezelfde prompt kan leiden tot een ontelbaar aantal resultaten. Alle andere toepassingen slagen hier niet in en alle versies van een herschreven artikel leiden tot hetzelfde resultaat. Tot slot kunnen enkel O1, O2, O4 en O5 wel onregelmatige werkwoorden corrigeren.

\subsubsection{Wont-haves}

Op het vlak van toegankelijkheid beschikken O4 en O5 over een mobiele versie. Daarnaast kan een gebruiker O1, O2 en O3 wel via een mobiel apparaat bekijken, maar deze lenen geen speciale interface toe. Vervolgens bieden E1, E2 en E3 geen mobiele versie aan. 

\medspace

Daarnaast beschikken enkel E1, E2 en E3 over luistersoftware. Hoewel browsers over ingebouwde luistertools beschikken, toch bieden de andere tools geen personaliseerbare \textit{text-to-speech} techniek aan. Tot slot beschikken de geteste toepassingen over geen integratie met andere spelcheckers. Wel werkt de browserextensie van \textit{Grammarly} bij zowel O1, O2, O3, O4 en O5. 

\section{Geschikte taalmodel voor gepersonaliseerde tekstvereenvoudiging met ATS}

De vergelijkende studie evalueert de uitvoer van de uitgeteste taallmodellen, opgesomd in \ref{table:vergelijkende-studie-taalmodellen}, met een machinale en een menselijke beoordeling. Zo achterhaalt deze onderzoeksmethode welk taalmodel of LLM beter aansluit bij het aanbieden van gepersonaliseerde ATS voor scholieren met dyslexie in de derde graad van het middelbaar onderwijs. 

\subsubsection{Machinale beoordeling van de vereenvoudigde teksten}

Tabel \ref{table:resultaten-aantal-zinnen} geeft het aantal zinnen per (vereenvoudigd) artikel. De MTS-referentieteksten bevatten minder zinnen dan het oorspronkelijk artikel. Het aantal zinnen na ATS met T1, T2 en T3 is gehalveerd tot minder dan een kwart van oorspronkelijke hoeveelheid zinnen. Enkel T4P2 genereert meer zinnen dan de oorspronkelijke versie van A1 na ATS. T4P2 genereert bij zowel A1 als A2 meer zinnen vergeleken met de andere geteste taalmodellen. T2 daarentegen genereert bij beide artikelen het minst aantal zinnen. Figuren \ref{img:boxplot-min-max-avg-words-a1} en \ref{img:boxplot-min-max-avg-words-a2} illustreren deze verschillen tussen de taalmodellen.

\begin{table}[h]
	\centering
	\begin{tabular}{ | m{3cm} | m{3cm} | m{3cm} | } 
		\hline
		\textbf{Bron} & \textbf{#Zinnen in A1} & \textbf{#Zinnen in A2} \\
		\hline
		Oorspronkelijk & 78  & 159 \\ 
		\hline
		MTS (door leerkracht) & 43 & 45 \\
		\hline
		MTS (door leerling) & n.v.t. & 50 \\
		\hline
		T1 & 26 & 24 \\
		\hline
		T2 & 11 & 7 \\
		\hline
		T3 & 67 & 130 \\
		\hline
		T4 P1 & 61 & 98 \\
		\hline
		T4 P2 & 89 & 133 \\
		\hline
		T4 P3 & 39 & 55 \\
		\hline
	\end{tabular}
	\caption{Aantal zinnen (gemeten met Spacy sentence embeddings) per tekst.}
	\label{table:resultaten-aantal-zinnen}
\end{table}

\begin{figure}[H]
	\includegraphics[width=\linewidth]{img/boxplot-avg-a1.png}
	\caption{Overzicht van het minimum, maximum en gemiddeld aantal woorden per zin per model in A1.}
	\label{img:boxplot-min-max-avg-words-a1}
\end{figure}

\begin{figure}[H]
	\includegraphics[width=\linewidth]{img/boxplot-avg-a2.png}
	\caption{Overzicht van het minimum, maximum en gemiddeld aantal woorden per zin per model in A2.}
	\label{img:boxplot-min-max-avg-words-a2}
\end{figure}

Verder vergelijkt het onderzoek de verkregen leesbaarheidsscores per zin dat ieder taalmodel kan genereren. Allereerst de FRE-scores die de leesgraad van een zin aanduidt. Alle geteste taalmodellen genereren zinnen waarvan de FRE-scores niet opmerkelijk hoger of lager zijn ten opzichte van het oorspronkelijk artikel. Figuren \ref{img:boxplot-fre-a1} en \ref{img:boxplot-fre-a2} tonen deze verschillen. Gemiddeld bevinden alle versies van het wetenschappelijk artikel zich tussen 20 en 50. Zonder \textit{outliers} beperkt T3 de FRE van alle zinnen tot hoogstens 40. T3, T4P1, T4P2 en T4P3 genereren zinnen met een hogere FRE dan OG en MTSL. 

\begin{figure}[H]
	\includegraphics[width=\linewidth]{img/boxplot-fre-a1.png}
	\caption{Boxplot van de FRE-scores voor A1.}
	\label{img:boxplot-fre-a1}
\end{figure}

\begin{figure}[H]
	\includegraphics[width=\linewidth]{img/boxplot-fre-a2.png}
	\caption{Boxplot van de FRE-scores voor A2.}
	\label{img:boxplot-fre-a2}
\end{figure}

De FOG-scores van alle geteste taalmodellen en MTS-referentieteksten zijn niet opmerkelijk hoger of lager bij de vereenvoudigde wetenschappelijke artikelen, zoals weergegeven in figuren \ref{img:boxplot-fog-a1} en \ref{img:boxplot-fog-a2}. De zinnen van MTSL2 en T2 scoren gemiddeld lagere FOG-scores dan OG. Daarnaast scoort T2 een lager gemiddelde dan andere taalmodellen. Dit gemiddelde ligt tussen 10 en 12. Tot slot scoren MTSL en andere taalmodellen gemiddeld hoger dan OG. Tot slot genereren de taalmodellen geen zinnen met een hogere FOG-score dan OG.

\begin{figure}[H]
	\includegraphics[width=\linewidth]{img/boxplot-fog-a1.png}
	\caption{Boxplot van de FOG-scores voor A1.}
	\label{img:boxplot-fog-a1}
\end{figure}

\begin{figure}[H]
	\includegraphics[width=\linewidth]{img/boxplot-fog-a2.png}
	\caption{Boxplot van de FOG-scores voor A2.}
	\label{img:boxplot-fog-a2}
\end{figure}

Vervolgens meet het onderzoek het aantal lange en complexe woorden die de taalmodellen genereren. Deze resultaten geeft figuur \ref{img:boxplot-poster.png} weer. Zo genereren T1, T2 en T3 meer complexe woorden vergeleken met T4, MTSL en OG. Bij A1 genereert T4P3 opmerkelijk minder complexe woorden per zin dan de andere taalmodellen. Verder volgt A2 een gelijke trend met de andere taalmodellen. Het gebruikte script ziet een woord als 'lang' wanneer dit minstens vier lettergrepen heeft. Tot slot genereren MTSL, T2, T4P1, T4P2 en T4P3 minder lange woorden per zin dan OG.

\begin{figure}[H]
	\includegraphics[width=\linewidth]{img/boxplot-poster.png}
	\caption{Een boxplot van het aantal lange en complexe woorden per zin, gegroepeerd op model voor A1.}
	\label{img:long-complex-words}
\end{figure}


Vervolgens tonen figuren \ref{img:histplot-aux-a1} en \ref{img:histplot-aux-a2} het aantal hulpwerkwoorden in de tekst. Deze figuren zijn geen absolute percentages en houden geen rekening met het aantal zinnen. Ten slotte tonen \ref{img:histplot-aux-a1} en \ref{img:histplot-aux-a2} het aantal vervoegingen van het werkwoord 'zijn' aan. 

\begin{figure}[H]
	\includegraphics[width=\linewidth]{img/boxplot-aux-a1.png}
	\caption{Een staafdiagram van het aantal gebruikte hulpwerkwoorden in de tekst, gegroepeerd op model voor A1.}
	\label{img:histplot-aux-a1}
\end{figure}

\begin{figure}[H]
	\includegraphics[width=\linewidth]{img/boxplot-aux-a2.png}
	\caption{Een staafdiagram van het aantal gebruikte hulpwerkwoorden in de tekst, gegroepeerd op model voor A2.}
	\label{img:histplot-aux-a2}
\end{figure}

\begin{figure}[H]
	\includegraphics[width=\linewidth]{img/boxplot-tobe-a1.png}
	\caption{Het aantal vervoegingen van het werkwoord 'zijn', gegroepeerd op model voor A1.}
	\label{img:histplot-tobe-a1}
\end{figure}

\begin{figure}[H]
	\includegraphics[width=\linewidth]{img/boxplot-tobe-a2.png}
	\caption{Het aantal vervoegingen van het werkwoord 'zijn', gegroepeerd op model voor A2.}
	\label{img:histplot-tobe-a2}
\end{figure}

\subsubsection{Menselijke beoordeling van de referentieteksten.}

In het volgende deel bespreekt het onderzoek de menselijke beoordeling van de resultaten. Allereerst kunnen T4P1 en T4P2 Engelstalige vaktermen vertalen naar het Nederlands. Zo blijft de afkorting voor 'DPKIA' intact, maar vertaalt T4P1 hetzelfde woord naar het Nederlands.  T1, T2, T3 en T4P3 houden hier echter geen rekening mee en behouden de oorspronkelijke versie van de tekst. De auteurs schrijven alle afkortingen voluit, zoals beschreven in de richtlijnen. Zo toont figuur \ref{img:vergelijking-taalmodellen} deze verschillen.

\begin{figure}[H]
	\includegraphics[width=\linewidth]{img/vergelijking.png}
	\caption{De verschillen tussen de oorspronkelijke tekst, T4P1 en T4P3 bij één uitgekozen paragraaf.}
	\label{img:vergelijking-taalmodellen}
\end{figure}

Alle taalmodellen kunnen LS toepassen. De handmatig vereenvoudigde referentieteksten bevatten zinnen die vakjargon gebruiken op het niveau van 15 tot 18 jarige studenten. T4P1 kan uitleg tussen ronde haakjes schrijven, wanneer het geen eenvoudiger synoniem kan vinden. T4P1, T1, T2 en T3 passen woorden aan, maar schrijven geen extra uitleg. T4P3 past deze techniek minder toe dan de vooraf vermelde taalmodellen. T4P3 verkort lange zinnen door deze op te splitsen. T1, T2 en T3 behalen een gelijke zinslengte als dat van de oorspronkelijke zin. T4P1 en T4P2 kunnen langere zinnen genereren, maar smelten geen twee zinnen met elkaar samen. 

\medspace

Geen taalmodel wijkt af van de hoofdgedachte van het oorspronkelijke wetenschappelijk artikel. Hoewel T1, T2 en T3 deels afgebroken zinnen kan genereren, bevatten deze zinnen de hoofdgedachte. T2 bevat minder dan 10\% van het oorspronkelijk artikel en ontbreekt daarbij bijzaken die nodig zijn om alle vragen in \ref{ch:referentietekst} te kunnen begrijpen en te beantwoorden. Tenslotte verwerken T1, T2 en T3 de APA- en California bronvermeldingen niet in de vereenvoudigde teksten. Hoewel T4 deze wel verwerkt, bevat de tekst na een vereenvoudiging deze bronvermeldingen niet meer.

\medspace

Ter conclusie van de resultaten scoren de drie prompts van T4 beter bij de menselijke beoordeling van de resultaten. Het taalmodel en de verwante drie prompts genereren coherente teksten met een verlaagde lexicale complexiteit. Echter houden de geteste taalmodellen weinig tot geen rekening met afkortingen of bronvermeldingen.

\section{Pentimentor vergelijken met \textit{top-of-the-line} tools.}

Gebruikers kunnen vanuit de homepagina drie schermen kiezen: het lerarencomponent, het scholierencomponent en een instellingenpagina. Op de instellingenpagina kunnen eindgebruikers hun persoonlijke opmaak toevoegen. Zo toont figuur \ref{img:website-instellingen} alle mogelijke opmaakopties die Pentimentor aanreikt.

\begin{center}
	\begin{figure}[H]
		\includegraphics[width=\linewidth]{img/website-instellingen.png}
		\caption{Voorbeeldweergave van de instellingenpagina.}
		\label{img:website-instellingen}
	\end{figure}
\end{center}

Bovendien stelt Pentimentor gebruikers in staat om op basis van gekregen parameters automatisch personaliseerbare docx-documenten te genereren. Zo toont figuur \ref{img:proto-pos-tagging-scholieren} een voorbeeldweergave van deze functionaliteit.

\begin{center}
	\begin{figure}[H]
		\includegraphics[width=\linewidth]{img/proto-pos-tagging.png}
		\caption{Een voorbeeldweergave van de toepassing van PoS-tagging bij het scholierencomponent.}
		\label{img:proto-pos-tagging-scholieren}
	\end{figure}
\end{center}

Scholieren kunnen zinnen selecteren om daarna deze tekst te laten vereenvoudigen met gepersonaliseerde ATS. Figuren \ref{img:proto-scholieren-step-1} en \ref{img:proto-scholieren-step-3} tonen hoe gebruikers met Pentimentor de gemarkeerde doorlopende tekst kunnen laten herschrijven naar een opsomming. Eerst markeren zij een stuk tekst om hiermee opties aan het mee te geven. Daarnaast kan het ook tekst herschrijven in een tabelformaat.

\begin{center}
	\begin{figure}[H]
		\includegraphics[width=\linewidth]{img/proto-opsomming-1.png}
		\caption{Stap 1 van een gepersonaliseerde tekstvereenvoudiging in het scholierencomponent.}
		\label{img:proto-scholieren-step-1}
	\end{figure}
\end{center}

\begin{center}
	\begin{figure}[H]
		\includegraphics[width=\linewidth]{img/proto-opsomming-3.png}
		\caption{Stap 2 van een gepersonaliseerde tekstvereenvoudiging in het scholierencomponent.}
		\label{img:proto-scholieren-step-3}
	\end{figure}
\end{center}

Verder tonen figuren \ref{img:step-1-proto-vraagstelling} en \ref{img:step-2-proto-vraagstelling} een tweede functionaliteit. Zo kunnen scholieren specifieke vragen stellen aan Pentimentor door middel van een gecentreerd invoerscherm.

\begin{center}
	\begin{figure}[H]
		\includegraphics[width=\linewidth]{img/proto-vraagstelling-1.png}
		\caption{Stap 1 bij het stellen van een specifieke vraag bij gemarkeerde tekst.}
		\label{img:step-1-proto-vraagstelling}
	\end{figure}
\end{center}

\begin{center}
	\begin{figure}[H]
		\includegraphics[width=\linewidth]{img/proto-vraagstelling-2.png}
		\caption{Stap 2 bij het stellen van een specifieke vraag bij gemarkeerde tekst.}
		\label{img:step-2-proto-vraagstelling}
	\end{figure}
\end{center}

Volgens de evaluatie en experimenten blijkt Pentimentor te voldoen aan de \textit{must-haves} uit de requirementsanalyse. Tabel \ref{img:moscow-table} geeft hier een overzicht van. Het biedt twee methoden aan om pdf-bestanden in te lezen. Dit via een pipeline van machineleer- en OCR-technieken. Bovendien kan het de tekst ophalen met behulp van PDFMiner, wat overeenkomt met de verwachte functionaliteiten voor pdf-upload. Daarnaast hebben gebruikers de vrijheid om te kiezen welke tekstinhoud ze willen vereenvoudigen met behulp van gepersonaliseerde ATS. Na de vereenvoudiging of samenvatting van een wetenschappelijk artikel kunnen eindgebruikers alle vragen beantwoorden met behulp van de inhoud van het vereenvoudigde artikel, zoals aangegeven door Hollenkamp (2020) als een absolute vereiste.

\medspace

Verder kunnen gebruikers de opmaak van Pentimentor aanpassen naargelang hun voorkeur. Zo past het systeem deze voorkeuren toe op de digitale weergave in de webtool, maar ook de opmaak van het uitvoerbestand. Figuur \ref{img:screenshot-pdf-attempt} toont de vereenvoudigde versie van het wetenschappelijke artikel met de parameters uit tabel \ref{table:chosen-parameters-experiment}. Daarnaast houdt Pentimentor rekening met de gekozen regeleindes, woord- en karakterspatiëring, lettertype -en grootte, koppenstructuur en marges van het uitvoerbestand. Pentimentor houdt hier rekening mee, in tegenstelling tot de andere uitgeteste toools. Enkel E1 en E2 kunnen het lettertype -en grootte aanpassen. Tot slot toont figuur \ref{img:screenshot-docx-attempt} hoe een volledig personaliseerbaar docx-bestand er uit kan zien.

\begin{figure}[H]
	\includegraphics[width=\linewidth]{img/screenshot-prototype-word.png}
	\caption{De uitvoer na een vereenvoudiging met Pentimentor. De tekst is een vereenvoudigde versie van het artikel van \textcite{VanBrakel2022}.}
	\label{img:screenshot-docx-attempt}
\end{figure}

Verder bevat Pentimentor enkele \textit{should-haves}. Allereerst kunnen gebruikers een tekst op een duidelijke manier markeren. Hiermee kunnen zij annotaties toevoegen, de tekst aanpassen door \textit{in-line} definities toe te voegen. Bovendien kan Pentimentor abstraherende samenvattingen genereren in verschillende formaten, zoals opsommingen, tabellen of doorlopende tekst. Zo toont figuur \ref{img:screenshot-docx-attempt} een voorbeeld van een gegenereerde opsommingssamenvatting.

\medspace

Pentimentor zorgt voor een duidelijke gebruikerservaring door meldingsschermen te tonen wanneer het iets van de gebruiker verwacht. Figuur \ref{img:step-1-proto-vraagstelling} illustreert deze werking. Bovendien toont het waarschuwingen in formulieren, zoals getoond in figuur \ref{img:proto-lerarencomponent}.

\begin{figure}[H]
    \includegraphics[width=\linewidth]{img/proto-lerarencomponent.png}
    \caption{Een mogelijke weergave van het lerarencomponent met het wetenschappelijk artikel van \textcite{VanBrakel2022} als input.}
    \label{img:proto-lerarencomponent}
\end{figure}

Echter voldoet Pentimentor niet aan alle \textit{should-haves}. Zo ontbreekt de mogelijkheid om automatisch een woordenlijst met moeilijke woorden of vakjargon te genereren. Daarnaast mist Pentimentor analytische functionaliteiten, zoals het tonen van tekstanalyse aan de eindgebruiker.

\medspace

Tot slot bevat Pentimentor geen \textit{wont-haves}. Zo ontbreekt het een luistercomponent waarmee scholieren de vereenvoudigde tekst kunnen beluisteren. Deze functionaliteit is wel aanwezig bij E1, E2 en E3. Andere uitgeteste tools beschikken hier ook niet over. Bovendien kunnen gebruikers Pentimentor alleen raadplegen in een lokale omgeving met Docker, terwijl zij wel andere geteste toepassingen zonder installatie kunnen raadplegen. Daarnaast heeft Pentimentor geen browserextensie, terwijl O5 dit als enige toepassing wel kan.
%%=============================================================================
%% Conclusie
%%=============================================================================

\chapter{Conclusie}%
\label{ch:conclusie}

Deze scriptie tracht een antwoord te bieden op de volgende onderzoeksvraag:

\begin{itemize}
	\item Hoe kan een wetenschappelijk artikel automatisch vereenvoudigd worden, gericht op de unieke noden van scholieren met dyslexie in de derde graad middelbaar onderwijs?
\end{itemize}

Eerst geeft de requirementsanalyse nieuwe inzichten in huidige toepassingen voor \textit{automatic text simplification} (ATS). Zo ontbreekt er personaliseerbaarheid in bestaande online tools. Verder beschikken tools die enkel lexicale vereenvoudiging toepassen over onvoldoende opmaakopties om de leeservaring van scholieren met dyslexie tijdens het begrijpend lezen van een wetenschappelijk artikel te bevorderen. Daarnaast kunnen eindgebruikers met deze toepassingen geen gepersonaliseerde vereenvoudiging of samenvatting maken. Ten slotte kunnen zij geen wetenschappelijke artikelen inladen in de toepassingen die wel gepersonaliseerde ATS kunnen uitvoeren. 

\medspace

Bing Chat en ChatGPT bieden mogelijkheden voor ATS aan. Ze vereisen echter een uitgebreide informaticakennis, ofwel een vaardigheid waarover de meeste scholieren en leraren niet beschikken. Ontwikkelaars kunnen de achterliggende taalmodellen gebruiken om toepassingen te maken, maar zij richten zich hoofdzakelijk op samenvattingstools. In hoofdzaak resulteert deze bijdrage niet in meer begrijpelijke teksten. Het komt het leerproces van scholieren vaak niet ten goede. Huidige toepassingen bewijzen nochtans dat ontwikkelaars toepassingen voor personaliseerbare tekstvereenvoudiging kunnen ontwikkelen. De opgestelde requirementsanalyse benadrukt de noodzaak van een gebruiksvriendelijke toepassing in het onderwijs, waarmee scholieren en leerkrachten wetenschappelijke teksten op een efficiënte manier kunnen vereenvoudigen.

\medspace

Vervolgens wijst de vergelijking van taalmodellen uit dat HuggingFace (HF) taalmodellen, specifiek getraind op vereenvoudigingsopdrachten, lexicale vereenvoudiging mogelijk maken. Het geavanceerde taalmodel GPT-3 doet het beter door bovendien syntactische vereenvoudiging aan te bieden, samen met formaatwijzigingen. Dit is ongezien bij huidige toepassingen. Zo produceert GPT-3 ook teksten met minder lange en complexe woorden. Dit taalmodel kan doelgroepen in grote lijn inschatten, waartoe andere tools niet in staat zijn. Geteste HF taalmodellen genereren minder coherente teksten en lopen het risico op samengesmolten zinnen. Daarnaast vereisen zij een extra vertaalfase, wat GPT-3 niet hoeft te doen. Zo moeten ontwikkelaars geen extra vertaalfase uitwerken wanneer zij teksten met GPT-3 willen vereenvoudigen. Daarbij moet het prototype specifieke prompts en technieken aangegeven door \textcite{McFarland2023, White2023} gebruiken.

\medspace

Tot slot toont de vorming van Pentimentor aan dat ontwikkelaars ATS-software kunnen ontwikkelen met \textit{open-source} AI- en NLP-technologieën. Zo kunnen zij PDFMiner en Layoutparser gebruiken om tekstinhoud uit wetenschappelijke artikelen te extraheren, met of zonder behoud van de oorspronkelijke titelstructuur. Bovendien kunnen ze de API van OpenAI's GPT-4 benutten voor gepersonaliseerde ATS-toepassingen door middel van geschikte prompts. Vervolgens kunnen zij met Pandoc gepersonaliseerde documenten in docx-formaat automatisch genereren. Ontwikkelaars kunnen basis Javascript toepassen om eenduidige handelingen voor eindgebruikers te onwikkelen, die voordien enkel per commandline mogelijk waren. Zo kunnen zij webpagina's opbouwen die voldoen aan de noden beschreven in \textcite{Rello2012a}.  Hoewel het prototype niet aan alle \textit{should-haves} en \textit{could-haves} voldoet, kunnen ontwikkelaars een volledig functionele toepassing uitwerken met de genoemde softwarepakketten.

\medspace

Dit onderzoek legt de nadruk op de aanwezigheid van geavanceerde taalmodellen en tools voor potentiële ATS-toepassingen die voldoen aan de behoeften van scholieren met dyslexie. GPT-3 kan dienen als een geschikt taalmodel, vanwege zijn sterke prestaties in toegepaste leesmetrieken, waaronder criteria zoals het aantal complexe en lange woorden per zin. Daarnaast kan dit taalmodel korte annotaties genereren voor vakjargon of onbekende woordenschat voor scholieren. Verder hebben ontwikkelaars de mogelijkheid om zich te richten op specifieke doelgroepen via aanpasbare parameters. Hierdoor kunnen teksten op maat worden gemaakt die aansluiten bij de individuele behoeften van de gebruiker. Ontwikkelaars moeten echter voorzichtig zijn met dergelijke toepassingen, omdat taalmodellen geen gegarandeerd correcte inschatting bieden voor de doelgroepen. Extra trainingsdata kan het model helpen bij deze inschatting door middel van leerstof op leesniveau van de doelgroep, zoals aangeraden door \textcite{Gooding2022}.
%%=============================================================================
%% Discussie
%%=============================================================================

\chapter{\IfLanguageName{dutch}{Discussie}{Discussie}}%
\label{ch:discussie}

Dit onderzoek gebruikt drie onderzoeksmethoden om te bepalen hoe ontwikkelaars een optimale vorm van gepersonaliseerde ATS kunnen bieden aan scholieren met dyslexie in de derde graad van het middelbaar onderwijs.

\medspace

Uit de resultaten van de requirementsanalyse blijkt dat zowel erkende toepassingen als online tools onvoldoende functionaliteiten bieden voor gepersonaliseerde ATS. Daarnaast bieden deze tools onvoldoende gepersonaliseerde opmaakopties. Dit resultaat komt overeen met de verwachting dat bestaande tools niet specifiek gericht zijn op gepersonaliseerde ATS voor scholieren met dyslexie in de derde graad van het middelbaar onderwijs. Mogelijke verklaringen hiervoor zijn de complexiteit die gepaard gaat met de ontwikkeling van dergelijk toepassing, het gebrek aan iniatief binnen dit vakgebied en de populariteit van pure samenvattingstools, zoals in de literatuurstudie aangegeven door \textcite{Gooding2022}.

\medspace

Hoewel ChatGPT en Bing Chatbot functionaliteiten bieden voor gepersonaliseerde ATS, ontbreken eenduidige handelingen waardoor gebruikers moeite kunnen hebben met het vereenvoudigen van wetenschappelijke artikelen. Daarnaast houdt het model van ChatGPT geen rekening met verwijzingen of artikelen buiten de getrainde data, wat problemen kan veroorzaken voor data-integriteit. Hiertegenover staat Bing Chat, dat wel rekening houdt met externe referenties en daarmee een goede basis vormt voor ontwikkelaars om referentiemateriaal aan te bieden in ondersteunende onderwijstoepassingen. Verder onderzoek naar de toepassing van deze AI via een API is noodzakelijk en kan baanbrekend zijn voor de onderwijssector, ondersteund door \textcite{Roose2023, Garg2022}. Anderzijds is er de mogelijkheid om bestaande toepassingen, zoals Kurzweil, uit te breiden met functionaliteiten die gepersonaliseerde ATS aanbieden aan scholieren met dyslexie in de derde graad van het middelbaar onderwijs.

\medspace

Vervolgens wijzen de resultaten van de vergelijkende studie uit dat het GPT-3 model geschikter is voor gepersonaliseerde ATS. Het geteste GPT-3-model gebruikt de davinci-engine en finetuned alleen API-parameters. Zo bevat het ook geen extra \textit{pre-trained} data van wetenschappelijke artikelen. De vergelijkende studie wijst verder uit dat de drie geteste HF-modellen via API en het geteste GPT-3-model via API beschikken over CWI-functionaliteiten en substitution generation. Hoewel de vrij beschikbare HF-taalmodellen LS mogelijk kunnen maken, staan ze in de schaduw van GPT-3, dat als API vrij beschikbaar is voor ontwikkelaars. Het GPT-3-model kan een baanbrekende oplossing bieden voor gepersonaliseerde ATS van wetenschappelijke artikelen, want het taalmodel kan snel en efficiënt moeilijke woorden herkennen in doorlopende tekst en structurele aanpassingen maken aan de oorspronkelijke tekst.

\medspace

Dit resultaat bevestigt de verwachting dat GPT-3 beter in staat is om gepersonaliseerde ATS aan te bieden in vergelijking met vrij beschikbare HF-taalmodellen. Een verklaring hiervoor is de complexiteit van het taalmodel. De geteste taalmodellen zijn getraind op data van wetenschappelijke artikelen. Meer onderzoek is echter nodig om deze verschillen beter te begrijpen binnen de context van wetenschappelijke artikelen. LLM's, waaronder GPT-3, kunnen vragen beantwoorden en een eenduidige oplossing voor gepersonaliseerde ATS aan ontwikkelaars aanbieden. Er is onderzoek nodig naar de verschillen op taalgebied in relatie tot de toename van parameters bij grotere taalmodellen, zoals aangewezen in \textcite{Simon2021}. Er is behoefte aan onderzoek naar het gebruik van nieuwe modellen zoals GPT-4 en Bing Chat in het onderwijs. De scriptie kon geen gebruikmaken van GPT-4. Een opvolgend onderzoek met dit taalmodel is vereist om te testen of dit taalmodel over voldoende data beschikt om wetenschappelijke artikelen te vereenvoudigen op maat van scholieren met dyslexie in de derde graad van het middelbaar onderwijs. Verder onderzoek naar doelgroepinschattingen via prompts is ook nodig. Daarnaast zou toekomstig onderzoek zich kunnen richten op het potentieel van de combinatie van GPT-3 en textit{full-text-search}-technologieën. Onderzoek is nodig naar de verschillen tussen taalmodellen, die getraind zijn op wetenschappelijke artikelen, en taalmodellen, die getraind zijn op algemene data. 

\medspace

De vergelijkende studie bevatte minieme verschillen tussen de taalmodellen bij de leesgraadscores FRE en FOG. Verder wijst de vergelijkende studie uit dat de \textit{readability}-library geen directe manier heeft om de actieve stem van een zin te achterhalen. Zo kan het onderzoek geen vaststelling maken of de uitgeteste taalmodellen in staat zijn om passief naar actief te schrijven. Spacy textit{word embeddings} kunnen een alternatieve manier aanreiken om hulpwerkwoorden en vervoegingen van het werkwoord zijn te achterhalen. Verder onderzoek is nodig om de bruikbaarheid van leesgraadscores te bepalen en te begrijpen hoe ze zich verhouden tot de kwaliteit van de vereenvoudigde tekst. Toepassingen zoals TextInspector meer metrieken dan de uitgeteste leesgraadscores aan. Daarom is er meer onderzoek nodig naar een optimale om geautomatiseerde tekstanalyse uit te kunnen voeren.

\medspace

Verworven kennis en aangeleerde tools uit alle richtingen Toegepaste Informatica aan Vlaamse Hogescholen, lieten toe om het prototype voor de webtool te ontwikkelen. Dit prototype dient slechts als een haalbaarheidstoetsing voor ontwikkelaars bij het ontwikkelen van dergelijke toepassing. Het is belangrijk dat de lezer zich bewust is van het feit dat de webtool zich baseert op onderzochte kenmerken en technieken die de impact van tekstvereenvoudiging met MTS hebben aangetoond bij scholieren met dyslexie. Daarnaast gebeurde de ontwikkeling van het prototype met het oog op een snelle en eenduidige implementatie van technieken die voordien enkel beschikbaar waren via CLI. Tijdens de ontwikkeling van het prototype is gebleken dat ontwikkelaars met vrij beschikbare middelen en API's in staat zijn om gepersonaliseerde ATS-toepassingen te bieden aan scholieren met dyslexie in de derde graad van het middelbaar onderwijs. Zo kunnen ontwikkelaars het stappenplan volgen om een vergelijkbaar resultaat te behalen en optioneel de taken in een projectteam parallel laten uitvoeren volgens de flowchart.

\medspace

Dit resultaat komt overeen met de verwachting dat ontwikkelaars over de benodigde tools beschikken om een dergelijk prototype voor gepersonaliseerde ATS te maken. Een verklaring hiervoor is de beschikbaarheid van textit{open-source} tools en python-bibliotheken die ontwikkelaars in staat stellen complexe taken eenvoudig uit te voeren. Toch moet de lezer zich ervan bewust zijn dat het prototype niet getest is bij het doelpubliek tijdens het begrijpend lezen van een wetenschappelijk artikel. Daarom kan het alleen dienen als een meting van de haalbaarheid voor ontwikkelaars. Er is een gebrek aan wetenschappelijke vakliteratuur over tekstvereenvoudiging met ATS voor deze doelgroep.

\medspace

Tot slot kunnen logopedisten of studenten in een logopedische studierichting dit prototype gebruiken om onderzoek uit te voeren naar het effect van dit prototype op leesbegrip bij scholieren met dyslexie in de derde graad van het middelbaar onderwijs. Dit stemt overeen met de implicaties waar \textcite{Gooding2022} op wijst. Onderzoekers binnen het vakdomein secundair onderwijs kunnen de effecten en voor -of nadelen van deze tool observeren bij leerlingen en leerkrachten in het middelbaar onderwijs. Er is echter meer onderzoek nodig om de inzet van gepersonaliseerde ATS-toepassingen en browserextensies voor tekstvereenvoudiging in het onderwijs te verbeteren. Zo is er behoefte aan een toepassing die alle functionaliteiten kan combineren. Bovendien is er behoefte aan meer onderzoek naar tekstvereenvoudiging met ATS voor de specifieke doelgroep van scholieren met dyslexie.



%---------- Bijlagen -----------------------------------------------------------

\appendix

\chapter{Onderzoeksvoorstel}

\section*{Samenvatting}

% Kopieer en plak hier de samenvatting (abstract) van je onderzoeksvoorstel.
Ingewikkelde woordenschat en zinsbouw hinderen scholieren met dyslexie in de derde graad van het middelbaar onderwijs bij het begrijpend lezen van wetenschappelijke artikelen. Gepersonaliseerde \textit{automated text simplification} (ATS) helpt deze scholieren bij hun leesbegrip. Daarnaast kan artificiële intelligentie (AI) dit proces automatiseren om de werkdruk bij leraren en scholieren te verminderen. Dit onderzoek achterhaalt met welke technologische en logopedische aspecten AI-ontwikkelaars rekening moeten houden bij de ontwikkeling van een AI-toepassing voor geautomatiseerde en gepersonaliseerde tekstvereenvoudiging. Hiervoor is de volgende onderzoeksvraag opgesteld: "Hoe kan een wetenschappelijk artikel automatisch worden vereenvoudigd, gericht op de unieke noden van scholieren met dyslexie in het derde graad middelbaar onderwijs?" . Een requirementsanalyse achterhaalt de benodigde functionaliteiten om gepersonaliseerde en geautomatiseerde tekstvereenvoudiging mogelijk te maken. Vervolgens wijst de vergelijkende studie uit welk taalmodel ontwikkelaars kunnen inzetten om de taak van gepersonaliseerde en geautomatiseerde tekstvereenvoudiging mogelijk te maken. De requirementsanalyse wijst uit dat toepassingen om wetenschappelijke artikelen te vereenvoudigen, gemaakt zijn voor een centrale doelgroep en geen rekening houden met de unieke noden van een scholier met dyslexie in het derde graad middelbaar onderwijs. Toepassingen voor gepersonaliseerde ATS zijn mogelijk, maar ontwikkelaars moeten meer inzetten op de unieke noden van deze scholieren.

% Verwijzing naar het bestand met de inhoud van het onderzoeksvoorstel
%---------- Inleiding ---------------------------------------------------------

\section{Introductie}%
\label{sec:introductie}

% Met een jaarlijks budget van 32 miljoen in het vakgebied kunstmatige intelligentie (AI) op de werkvloer is België een pionier \autocite{Crevits2022}.  Zo zijn er verschillende projecten, om taalgerelateerde AI-ontwikkelingen op te starten, uit de grond gestampt. Het amai!-project \footnote{https://amai.vlaanderen/}  verenigt AI-softwarebedrijven uit verschillende domeinen om zo met AI-toepassingen te maken die processen automatiseren om de werkdruk te verminderen, zoals binnen het onderwijs \textit{real-time} ondertiteling en een taalassistent voor leerkrachten in meertalige klasgroepen.

Het Vlaams middelbaar onderwijs staat op barsten. Werkdruk en stress overspoelen leraren en scholieren. Bovendien is de derde graad van het middelbaar onderwijs een belangrijke mijlpaal voor de verdere loopbaan van scholieren, al hebben zij volgens \textcite{Dapaah2022} dan moeite om grip te krijgen op de vakliteratuur bij STEM-vakken. De STEM-agenda\footnote{https://www.vlaanderen.be/publicaties/stem-agenda-2030-stem-competenties-voor-een-toekomst-en-missiegericht-beleid} van de Vlaamse overheid moet het STEM-onderwijs tegen 2030 aantrekkelijker te maken, door de ondersteuning voor zowel leerkrachten als scholieren te verbeteren. Toch neemt deze agenda de aanpak van steeds complexere wetenschappelijke taal, zoals beschreven in \textcite{Barnett2020}, niet op. Wetenschappelijke artikelen vereenvoudigen, op maat van de noden van een scholier met dyslexie in het middelbaar onderwijs, is tijds- en energie-intensief voor leerkrachten en scholieren. Automatische en adaptieve tekstvereenvoudiging biedt hier een baanbrekende oplossing om de werkdruk in het middelbaar onderwijs te verminderen.

Het doel van dit onderzoek is om te achterhalen met welke technologische en logopedische aspecten AI-ontwikkelaars rekening moeten houden bij de ontwikkeling van een adaptieve AI-toepassing voor geautomatiseerde tekstvereenvoudiging. De volgende onderzoeksvraag is opgesteld: "Hoe kan een wetenschappelijk artikel automatisch vereenvoudigd worden, gericht op de verschillende behoeften van scholieren met dyslexie in de derde graad middelbaar onderwijs?". Een antwoord op volgende deelvragen kan de onderzoeksvraag vereenvoudigen. Eerst geeft de literatuurstudie een antwoord op de eerste vier deelvragen. Daarna vormt het veldonderzoek een antwoord op de vijfde deelvraag. Ten slotte beantwoordt de vergelijkende studie de zesde en laatste deelvraag. De resultaten van dit onderzoek zetten AI-ontwikkelaars aan om een toepassing te maken om scholieren met dyslexie te kunnen ondersteunen in de derde graad middelbaar onderwijs.

% wat wordt manueel gedaan? --> voor de doelgroep
 % --> kleiner prototype
  
% aantal deelvragen doet er niet toe, maak gewoon dat je concrete en doelgerichte deelvragen hebt

\begin{enumerate}
	\item Welke aanpakken zijn er voor geautomatiseerde tekstvereenvoudiging? Aansluitende vraag: "Hoe worden teksten handmatig vereenvoudigd voor scholieren met dyslexie?"
	\item Welke specifieke noden hebben scholieren van de derde graad middelbaar onderwijs bij het begrijpen van complexere teksten?
	\item Wat zijn de specifieke kenmerken van wetenschappelijke artikelen? 
	\item Met welke valkuilen bij taalverwerking met AI moeten ontwikkelaars rekening houden?
	\item Welke toepassingen, tools en modellen zijn er beschikbaar om Nederlandstalige geautomatiseerde tekstvereenvoudiging met AI mogelijk te maken?
	\item Welke functies ontbreken AI-toepassingen om geautomatiseerde én adaptieve tekstvereenvoudiging mogelijk te maken voor \newline scholieren met dyslexie in de derde graad \newline middelbaar onderwijs? Aansluitende vraag: "Welke manuele methoden voor tekstvereenvoudiging komen niet in deze tools voor?"
\end{enumerate}

%---------- Stand van zaken ---------------------------------------------------

\section{State-of-the-art}%
\label{sec:state-of-the-art}

\subsection{Tekstvereenvoudiging}

% Deelvraag: Wat is tekstsimplificatie
De voorbije tien jaar is artificiële intelligentie (AI) sterk verder ontwikkeld. \textcite{Vasista2022} benadrukt dat de toename in kennis voor nieuwe toepassingen zorgde. Tekstvereenvoudiging vloeide hier uit voort. Momenteel bestaan er al robuuste toepassingen die teksten kunnen vereenvoudigen, zoals Resoomer\footnote{https://resoomer.com/nl/}, Paraphraser\footnote{https://www.paraphraser.io/nl/tekst-samenvatting} en Prepostseo\footnote{https://www.prepostseo.com/tool/nl/text-summarizer}. Binnen het kader van tekstvereenvoudiging is er bestaande documentatie beschikbaar waar onderzoekers het voordeel van toegankelijkheid aanhalen, maar volgens \textcite{Gooding2022} ontbreken deze toepassingen de extra noden die scholieren met dyslexie in de derde graad middelbaar onderwijs vereisen.

\textcite{Shardlow2014} haalt aan dat het algemene doel van tekstvereenvoudiging is om ingewikkelde bronnen toegankelijker te maken. Het zorgt voor verkorte teksten zonder de kernboodschap te verliezen. \textcite{Siddharthan2014} haalt verder aan dat tekstvereenvoudiging op één van drie manieren gebeurt. Er is conceptuele vereenvoudiging waarbij documenten naar een compacter formaat worden getransformeerd. Daarnaast is er uitgebreide modificatie die kernwoorden aanduidt door gebruik van redundantie. Als laatste is er samenvatting die documenten verandert in kortere teksten met alleen de topische zinnen. Met deze concepten zijn ontwikkelaars volgens \textcite{Siddharthan2014} in staat om ingewikkelde woorden te vervangen door eenvoudigere synoniemen of zinnen te verkorten zodat ze sneller leesbaar zijn.

Tekstvereenvoudiging behoort tot de zijtak van \textit{Natural Language Processing} (NLP) in AI. NLP omvat methodes om menselijke teksten om te zetten in tekst voor machines. Documenten vereenvoudigen met NLP kan volgens \textcite{Chowdhary2020} op twee manieren: extraherend of abstraherend. Bij extraherende vereenvoudiging worden zinnen gelezen zoals ze zijn neergeschreven. Vervolgens bewaart een document de belangrijkste taalelementen om de tekst te kunnen hervormen. Deze vorm van tekstvereenvoudiging komt volgens \autocite{Sciforce2020} het meeste voor. Daarnaast is er abstraherende vereenvoudiging waarbij de kernboodschap wordt bewaard. Met de kernboodschap wordt er een nieuwe zin opgebouwd. Volgens het onderzoek van \textcite{Chowdhary2020} heeft deze vorm potentieel, maar het zit nog in de kinderschoenen.

\subsection{Noden van scholieren met dyslexie}

% Deelvraag 2: Bewezen voordelen van tekstsimplificatie bij scholieren met dyslexie
Het experiment van Franse wetenschappers \newline \textcite{Gala2016} illustreert dat manuele tekstvereenvoudiging schoolteksten toegankelijker \newline maakt voor kinderen met dyslexie. Dit deden ze door simpelere synoniemen en zinsstructuren te gebruiken. Tien kinderen werden opgenomen in het experiment, variërend van 8 tot 12 jaar oud. Verwijswoorden werden vermeden en woorden kort gehouden. De resultaten waren veelbelovend. Het leestempo lag hoger en de kinderen maakten minder leesfouten. Ook bleek er geen verlies van begrip in de tekst bij geteste kinderen. Resultaten van de studie werden gebundeld voor de mogelijke ontwikkeling van een AI-tool.

% doelgroep concreter maken hierboven
% semantisch vlak --> zinsstructuur
% welke criteria vonden zij 'geslaagd'?

De visuele weergave van tekst beïnvloedt de leessnelheid bij scholieren met dyslexie. Zo haalt het onderzoek van \textcite{Rello2012} tips aan waarmee teksten en documenten rekening moeten houden bij scholieren met dyslexie in de derde graad middelbaar onderwijs. Het gaat over speciale lettertypes, spreiding tussen woorden en het gebruik van inzoomen op aparte zinnen. Het onderzoek haalt verder aan dat teksten voor deze unieke noden aanpassen tijdrovend is en daarmee tekstvereenvoudiging door AI een revolutionaire oplossing kan bieden. De Universiteit van Kopenhagen is met bovenstaande idee aan de slag gegaan. Onderzoekers \textcite{Bingel2018} hebben gratis software ontwikkeld, genaamd Hero\footnote{https://beta.heroapp.ai/}, om tekstvereenvoudiging voor scholieren in het middelbaar onderwijs met dyslexie te automatiseren. De software bestudeert met welke woorden de gebruiker moeite heeft, en vervangt die door simpelere alternatieven. Hero bevindt zich nu in beta-vorm en wordt enkel in het Engels en Deens ondersteund. Als alternatief is er Readable\footnote{https://readable.com/}. Dit is een Engelstalige AI-toepassing dat zinnen beoordeeld met leesbaarheidsformules.

% Deelvraag: Uitdagingen van AI-software met tekstsimplificatie
\textcite{Roldos2020} haalt aan dat NLP in de laatste decennia volop in ontwikkeling is, maar ontwikkelaars botsen nog op uitdagingen. Het gaat om zowel interpretatie- als dataproblemen bij AI-modellen. Het onderzoek haalt twee punten aan. Allereerst is het voor een machine moeilijk om de context van homoniemen te achterhalen. Bijvoorbeeld bij het woord ‘bank’ is het niet duidelijk voor de machine of het gaat over de geldinstelling of het meubel. Daarnaast zijn synoniemen een probleem voor tekstverwerking.

Het onderzoek van \textcite{Sciforce2020} haalt aan dat het merendeel van NLP-toepassingen Engelstalige invoer gebruikt. Niet-Engelstalige toepassingen zijn zeldzaam. De opkomst van AI technologieën die twee datasets gebruiken, biedt een oplossing voor dit probleem. De software vertaalt eerst de oorspronkelijke tekst naar de gewenste taal, voordat de tekst wordt herwerkt. Hetzelfde onderzoek bewijst dat het vertalen van gelijkaardige talen, zoals Duits en Nederlands, een minimaal verschil opleverd.

% Deelvraag: Stand van zaken bij Belgische secundaire scholen
% Voor scholieren met dyslexie in het derde graad middelbaar onderwijs bestaan digitale hulpmiddelen die voor een betere visuele presentatie zorgen van teksten. De Vlaamse overheid leent gratis abonnementen\footnote{https://onderwijs.vlaanderen.be/nl/onderwijspersoneel/van-basis-tot-volwassenenonderwijs/lespraktijk/ict-in-de-klas/voorleessoftware-voor-leerlingen-met-leesbeperkingen} uit voor voorlees- en schrijfsoftware. De voornaamste zijn SprintPlus\footnote{https://www.sprintplus.be/}, Alinea\footnote{https://sensotec.be/product/alinea-suite/} en Kurzweil3000\footnote{https://sensotec.be/product/kurzweil-3000/}. Vlaamse scholieren met dyslexie in het middelbaar onderwijs kunnen voor deze software een gratis abonnement of licentie aanvragen. Al bieden de vijf softwarepakketten elk een samenvattingsfunctie aan, echter ligt de focus op spreek- en luisterfuncties waarbij het samenvatten en markeren van tekst als extra wordt gehouden.

% \subsection{Wetenschappelijke artikelen}

Volgens \textcite{PlavenSigray2017} houden onderzoekers zich vaak in hun eigen taalbubbel, wat negatieve gevolgen heeft voor de leesbaarheid van een wetenschappelijk artikel. Bovendien vormt de stijgende trend van het gebruik aan acroniemen \textcite{Barnett2020} een extra hindernis. \textcite{Donato2022} haalt aan dat onbegrijpelijke literatuur, waaronder studiemateriaal geschreven door de docent en online wetenschappelijke artikelen, één van de redenen is waarom scholieren met dyslexie in het middelbaar onderwijs van richting veranderen.

\subsection{Huidige toepassingen}

Vlaanderen heeft weinig zicht op de geïmplementeerde AI software in scholen. Dit werd vastgesteld door \autocite{Martens2021}, een samenwerking tussen de Vlaamse universiteiten en overheid voor AI. Vergeleken met andere Europese landen, maakt België het minst gebruik van leerling-georiënteerde hulpmiddelen. Degenen die wel gebruikt worden, zijn vooral online leerplatformen voor zelfstandig werken. Ook maakt België amper gebruik van beschikbare software die de leermethoden en -noden van leerlingen evalueert \autocite{Martens2021a}. 

\textcite{Verhoeven2023} haalt aan dat AI-toepassingen zoals ChatGPT, Google Bard en Bing AI kunnen helpen om routinematig werk te verminderen in het onderwijs. Echter haalt \textcite{Deckmyn2021} aan dat GPT-3, het model van ChatGPT, sterker staat in het maken van Engelstalige teksten vergeleken met Nederlandstalige teksten. De databank waar het GPT-3 model mee is getraind, bestaat uit 92\% Engelstalige woorden, terwijl er 0,35\% Nederlandse woorden aanwezig zijn in dezelfde databank. Ontwikkelaars moeten de evolutie van deze modellen opvolgen, voordat er Nederlandstalige toepassingen mee worden gemaakt.

% eventueel Bing AI bekijken
% --> artikel: 

% Deelvraag: Wat is er nodig voor tekstsimplificatie? 

\subsection{Ontwikkelen met AI}

Python staat bovenaan de lijst van programmeertalen voor NLP-toepassingen. Volgens het onderzoek van \textcite{Thangarajah2019} is dit te wijten aan de eenvoudige syntax, kleine leercurve en grote beschikbaarheid van kant-en-klare bibliotheken. Wiskundige berekeningen of statistische analyses kunnen worden uitgevoerd met één lijn code. \textcite{Malik2022} haalt de twee meest voorkomende aan, namelijk NLTK\footnote{https://www.nltk.org/} en Spacy\footnote{https://spacy.io/}. \textit{Deep Martin}\footnote{https://github.com/chrislemke/deep-martin} bouwt verder op het onderzoek van \textcite{Shardlow2014} naar een pipeline voor lexicale vereenvoudiging. \textit{Deep Martin} maakt gebruik van \textit{custom transformers} om invoertekst te converteren naar een vereenvoudigde versie van de tekstinhoud.

Voor Germaanse talen zijn er enkele datasets en word embeddings beschikbaar die de complexiteit van woorden bijhouden. Zo zijn er in de Duitse taal Klexicon\footnote{https://github.com/dennlinger/klexikon} en TextComplexityDE\footnote{https://github.com/babaknaderi/TextComplexityDE}. Een onderzoek van \textcite{Suter2016} bouwde een rule-based NLP-model met 'Leichte Sprache', wat een dataset is met eenvoudige Duitstalige zinsconstructies. Nederlandstalige datasets zijn in schaarse hoeveelheden beschikbaar, waardoor het vertalen uit een Germaanse taal is hier een optie.

Volgens \textcite{Garbacea2021} is het belangrijk dat AI-ontwikkelaars niet alleen aandacht besteden aan het aanpassen van woorden en zinnen, maar ook aan de gebruiker meegeven waarom iets is aangepast. De onderzoekers wijzen op twee ethische aspecten. Eerst moet de toepassing duidelijk aangeven waarom een woord of zin is aangepast. Het model moet de moeilijkheidsgraad van de woorden of zinnen bewijzen. \textcite{Iavarone2021} beschrijft een methode met regressiemodellen om de moeilijkheidsgraad te bepalen door een gemiddeld moeilijkheidspercentage per zin te berekenen. Daarnaast benadrukt \textcite{Garbacea2021} het belang van het markeren van de complexere delen van een tekst. Hiervoor haalt hetzelfde onderzoek methoden aan zoals \textit{lexical} of \textit{deep learning}.

Er is een tactvolle aanpak nodig om een vereenvoudigde tekst met AI te beoordelen. De studie van \textcite{Swayamdipta2019} haalt aan dat er extra nood is aan NLP-modellen waarbij de tekst zijn kernboodschap behoudt. Samen met Microsoft Research bouwden ze NLP-modellen die gericht waren op de bewaring van zinsstructuur en -context door \emph{scaffolded learning}. Hiervoor maakten de onderzoekers gebruik van een voorspellingsmethode die de positie van woorden en zinnen in een document beoordeelde. De Flesch-Kincaid leesbaarheidstest is volgens \newline \textcite{Readable2021} een alternatieve manier om vereenvoudigde tekstinhoud te beoordelen, zonder de nood aan \textit{pre-trained} modellen. Figuur \ref{img:readable-scheme} geeft de indeling per doelgroep weer. Deze score kan eenvoudig worden berekend met de \textit{Python-library} \textit{textstat}\footnote{https://pypi.org/project/textstat/}. 

\begin{figure}
	\includegraphics[width=\linewidth]{img/Screenshot_302.png}
	\caption{De indeling van leesgraadscores per doelgroep. Bron: \autocite{Readable2021}}
	\label{img:readable-scheme}
\end{figure}

%---------- Methodologie ------------------------------------------------------
\section{Methodologie}%
\label{sec:methodologie}
Een \textit{mixed-methods} onderzoek toont aan hoe toepassingen automatisch een wetenschappelijke artikel kunnen vereenvoudigen, gericht op scholieren met dyslexie in de derde graad middelbaar onderwijs. Het onderzoek houdt vijf grote fases in. De eerste fase is het proces van geautomatiseerde tekstvereenvoudiging beschrijven. Dit gebeurt via een grondige studie van vakliteratuur en wetenschappelijke teksten. Ook blogs van experten komen hier aan bod. Na het verwerven van de nodige inzichten wordt er een verklarende tekst opgesteld.

De tweede fase bestaat uit het analyseren van wetenschappelijke werken over de bewezen voordelen van tekstvereenvoudiging bij scholieren met dyslexie van de derde graad middelbaar onderwijs. Hiervoor zijn geringe thesissen beschikbaar, die zorgvuldigheid vragen tijdens interpretatie. De resulterende tekst bevat de voordelen samen met hun wetenschappelijke onderbouwing.

De derde fase is opnieuw een beschrijving. Hier worden de valkuilen bij taalverwerking met AI-software nagegaan. Deze fase van het onderzoek brengt mogelijke nadelen en tekortkomingen van AI-software bij tekstvereenvoudiging aan het licht. Dit gebeurt aan de hand van een technische uitleg.

De vierde fase omvat een toelichting over beschikbare AI toepassingen voor tekstvereenvoudiging. Aan de hand van een veldonderzoek op het internet en bij bedrijven wordt een longlist opgesteld van beschikbare toepassingen voor tekstvereenvoudiging in het middelbaar onderwijs. Met een requirementsanalyse wordt er een shortlist opgesteld van software. Het toetsen van verschillende tools wordt ook betrokken in deze fase. De shortlist vormt de basis voor de ontwikkeling van een prototype voor geautomatiseerde en adaptieve tekstvereenvoudiging.

% bestaande technologie gebruiken
% multilinguale modellen --> Nederlands als brugtaal
% hoe testen? --> output van specifieke tools
% trials --> selectie van maken --> als ik 'dit' toepas, dan krijg ik 'dit'

% tool van nieuwsbrieven --> hoe ver staan zij?
% --> interessant om te zien waar de parallellen liggen: complexe zinnen, woorden

De vijfde en laatste fase van het onderzoek bestaat uit het testen en beoordelen van gekozen AI-toepassingen voor tekstvereenvoudiging. In dit experiment proberen scholieren met dyslexie in de derde graad middelbaar onderwijs de shortlisted AI toepassingen en het prototype uit. Het doel van het experiment is om de effectiviteit en gebruikersvriendelijkheid van deze toepassingen te beoordelen. Na een grondige analyse wordt er met de resultaten bepaalt of de toepassingen aan de unieke noden van een scholier met dyslexie in de derde graad middelbaar onderwijs voldoen om wetenschappelijke artikelen te vereenvoudigen voor scholieren in het middelbaar onderwijs.

%---------- Verwachte resultaten ----------------------------------------------
\section{Verwacht resultaat, conclusie}
\label{sec:verwachte_resultaten}

% exclameren dat de tools goed automatiseren, maar weinig tot geen keuze aanreiken

Er wordt verwacht dat de huidige softwareoplossingen voor tekstvereenvoudiging onvoldoende aansluiten bij de noden van scholieren met dyslexie in de derde graad middelbaar onderwijs. Het prototype is moeilijk af te stemmen op de specifieke noden van deze doelgroep. Ontwikkelaars die werken met bestaande modellen moeten \textit{custom transformers} inzetten om bevredigende resultaten te krijgen. Bovendien ontbreken er Nederlandstalige word embeddings die de complexiteit van elk woord bijhouden en aan kant-en-klare modellen die de inhoud van wetenschappelijke artikelen kunnen vereenvoudigen. Word embeddings uit een Germaanse taal gebruiken, gevolgd door vertaling naar het Nederlands is wel een aanvaardbaar alternatief. 

% Er zijn te weinig kant-en-klare algoritmen en modellen beschikbaar om een pipeline voor tekstvereenvoudiging op te zetten, gericht op scholieren met dyslexie in het middelbaar onderwijs. 



%%---------- Andere bijlagen --------------------------------------------------
\input{referentietekst}


%%---------- Backmatter, referentielijst ---------------------------------------

\backmatter{}

\setlength\bibitemsep{2pt} %% Add Some space between the bibliograpy entries
\printbibliography[heading=bibintoc]

\end{document}
