%%=============================================================================
%% Conclusie
%%=============================================================================

\chapter{Conclusie}%
\label{ch:conclusie}

Deze scriptie tracht een antwoord te bieden op de volgende onderzoeksvraag:

\begin{itemize}
	\item Hoe kan een wetenschappelijk artikel automatisch vereenvoudigd worden, gericht op de unieke noden van scholieren met dyslexie in de derde graad middelbaar onderwijs?
\end{itemize}

Eerst geeft de requirementsanalyse nieuwe inzichten in huidige toepassingen voor \textit{automatic text simplification} (ATS). Zo ontbreekt er personaliseerbaarheid in bestaande online tools. Verder beschikken tools die enkel lexicale vereenvoudiging toepassen over onvoldoende opmaakopties om de leeservaring van scholieren met dyslexie tijdens het begrijpend lezen van een wetenschappelijk artikel te bevorderen. Daarnaast kunnen eindgebruikers met deze toepassingen geen gepersonaliseerde vereenvoudiging of samenvatting maken. Ten slotte kunnen zij geen wetenschappelijke artikelen inladen in de toepassingen die wel gepersonaliseerde ATS kunnen uitvoeren. 

\medspace

Bing Chat en ChatGPT bieden mogelijkheden voor ATS aan. Ze vereisen echter een uitgebreide informaticakennis, ofwel een vaardigheid waarover de meeste scholieren en leraren niet beschikken. Ontwikkelaars kunnen de achterliggende taalmodellen gebruiken om toepassingen te maken, maar zij richten zich hoofdzakelijk op samenvattingstools. In hoofdzaak resulteert deze bijdrage niet in meer begrijpelijke teksten. Het komt het leerproces van scholieren vaak niet ten goede. Huidige toepassingen bewijzen nochtans dat ontwikkelaars toepassingen voor personaliseerbare tekstvereenvoudiging kunnen ontwikkelen. De opgestelde requirementsanalyse benadrukt de noodzaak van een gebruiksvriendelijke toepassing in het onderwijs, waarmee scholieren en leerkrachten wetenschappelijke teksten op een efficiënte manier kunnen vereenvoudigen.

\medspace

Vervolgens wijst de vergelijking van taalmodellen uit dat HuggingFace (HF) taalmodellen, specifiek getraind op vereenvoudigingsopdrachten, lexicale vereenvoudiging mogelijk maken. Het geavanceerde taalmodel GPT-3 doet het beter door bovendien syntactische vereenvoudiging aan te bieden, samen met formaatwijzigingen. Dit is ongezien bij huidige toepassingen. Zo produceert GPT-3 ook teksten met minder lange en complexe woorden. Dit taalmodel kan doelgroepen in grote lijn inschatten, waartoe andere tools niet in staat zijn. Geteste HF taalmodellen genereren minder coherente teksten en lopen het risico op samengesmolten zinnen. Daarnaast vereisen zij een extra vertaalfase, wat GPT-3 niet hoeft te doen. Zo moeten ontwikkelaars geen extra vertaalfase uitwerken wanneer zij teksten met GPT-3 willen vereenvoudigen. Daarbij moet het prototype specifieke prompts en technieken aangegeven door \textcite{McFarland2023, White2023} gebruiken.

\medspace

Tot slot toont de vorming van Pentimentor aan dat ontwikkelaars ATS-software kunnen ontwikkelen met \textit{open-source} AI- en NLP-technologieën. Zo kunnen zij PDFMiner en Layoutparser gebruiken om tekstinhoud uit wetenschappelijke artikelen te extraheren, met of zonder behoud van de oorspronkelijke titelstructuur. Bovendien kunnen ze de API van OpenAI's GPT-4 benutten voor gepersonaliseerde ATS-toepassingen door middel van geschikte prompts. Vervolgens kunnen zij met Pandoc gepersonaliseerde documenten in docx-formaat automatisch genereren. Ontwikkelaars kunnen basis Javascript toepassen om eenduidige handelingen voor eindgebruikers te onwikkelen, die voordien enkel per commandline mogelijk waren. Zo kunnen zij webpagina's opbouwen die voldoen aan de noden beschreven in \textcite{Rello2012a}.  Hoewel het prototype niet aan alle \textit{should-haves} en \textit{could-haves} voldoet, kunnen ontwikkelaars een volledig functionele toepassing uitwerken met de genoemde softwarepakketten.

\medspace

Dit onderzoek legt de nadruk op de aanwezigheid van geavanceerde taalmodellen en tools voor potentiële ATS-toepassingen die voldoen aan de behoeften van scholieren met dyslexie. GPT-3 kan dienen als een geschikt taalmodel, vanwege zijn sterke prestaties in toegepaste leesmetrieken, waaronder criteria zoals het aantal complexe en lange woorden per zin. Daarnaast kan dit taalmodel korte annotaties genereren voor vakjargon of onbekende woordenschat voor scholieren. Verder hebben ontwikkelaars de mogelijkheid om zich te richten op specifieke doelgroepen via aanpasbare parameters. Hierdoor kunnen teksten op maat worden gemaakt die aansluiten bij de individuele behoeften van de gebruiker. Ontwikkelaars moeten echter voorzichtig zijn met dergelijke toepassingen, omdat taalmodellen geen gegarandeerd correcte inschatting bieden voor de doelgroepen. Extra trainingsdata kan het model helpen bij deze inschatting door middel van leerstof op leesniveau van de doelgroep, zoals aangeraden door \textcite{Gooding2022}.